\chapter{Crypto Journalism}

\index{trade journalism}

Tech journalism is a small part of a larger media landscape undergoing a shift
in its distribution modes. In the internet era, journalism is in a precarious
state. Moreover, this state requires a reimagining of the role of journalism in
the public sphere while finding a way to remain profitable. The line between
journalism and commerce has become increasingly blurred with the rise of native
advertising. This practice has become commonplace and is offered to commercial
entities to place direct advertisements inside of publications that match the
form and appearance of their surrounding content. Nowhere is this conflict of
interest more evident than in the coverage of cryptocurrencies and the yellow
journalism surrounding it.

\index{journalism!trade}
\index{journalism!crypto}

We are entering a world where anyone with a keyboard and internet connection can
spread misinformation faster than experts can effectively refute disinformation.
There is always a fundamental asymmetry to disinformation, as humorously noted
by the popular cliche: ``It takes ten times as much energy to refute bullshit as
it does to produce it.''

However, this trend blurs lines between journalism, and promotional content has
been the norm in most trade journalism, which is journalism that addresses a
niche business sector with articles relevant to the commercial interests of
those employed in the field. There is trade journalism in interior decorating,
real estate, and private wealth management. In the financial services sector,
respectable and established trade publications such as The Wall Street Journal
and the Financial Times cover the sizeable overall state of the market and
stories in the general public interest. These institutions typically follow
extremely high journalistic standards and may separate reporting and editorial
content. There is a high ethical standard to disclose conflicts of interest
between the personal holdings of a specific journalist. Generally, reporters
cannot hold the individual financial products in their articles.

However, cryptocurrency and its associated trade journalism have few—if
any—restrictions on outright touting or promotion. Nor does crypto journalism
have an established culture of ethics, standards, or obligations to disclose
conflicts of interest. This context has resulted in a contemporary form of trade
journalism that thrives in this wild west environment of volatility predictions,
constant scandals, and blatant self-promotion.

\section{Checkbook Journalism}

Across trade journalism, so-called \textit{pay-to-play} articles are an
increasingly common practice where commercial parties will outright pay a
journalist for a story written in their interests in a purely commercial
transaction. The transaction is simple, the journalist exchanges their reach,
influence, credibility, and expertise in the domain, and in exchange, the other
party receives positive press about their product or offering. There is
absolutely nothing intrinsically wrong with this practice as publications are
commercial enterprises and have no legal obligation to be objective or even do
reporting.  However, blurring the line between outlets that practice objective
journalism and pay-to-play outlets is a matter of public concern and should be
disclosed when citing trade journalism.

\index{journalism!pay to play}

The confusion about trade journalism as a reliable source is unfortunately
common in the absence of authoritative mainstream reporting on cryptocurrency.
Government bodies and financial institutions such as the International Monetary
Fund, United States Securities and Exchange Commission, and FinCEN regularly
cite cryptocurrency trade journalism as the basis for public policy.

\index{FinCEN}

During the height of the initial coin offering (ICO) bubble, there was a massive
explosion in pay-to-play publications where, for a price, the outlet would
spread the merits of the latest token offering to prospective investors and
mirror the same pitch and investment opportunity claims as if they were
statements of fact. This process of credibility purchasing, exploitation of
transitive trust, and stoking a "fear of missing out" was a core part of the
engine that drove the ICO bubble and was a lucrative enterprise for those
participating in it. Some unethical publications have silently pulled their
articles touting tokens that were the subject of lawsuits or criminal
investigations after the fact.

\index{ICO}

The articles pushed by these outlets vary from the mundane to the bizarre, but
several trends are apparent headline trends across most outlets. The first
narrative is an almost pending corporate adoption of bitcoin or blockchain
technology. This narrative ties into the legitimacy story of technology and
feeds on the public repute of large corporations such as central banks,
investors, and governments. These headlines are generally derivatives of the
standard form "X Bank looking to Tokenize Y" or "Central Bank Z is looking into
bitcoin." The content of the articles will cherry-pick quotes from seemingly
mundane internal reports on emerging trends in financial services to support
whatever position the outlet is looking to promote. The contents of these
reports rarely ever support any one technology or even denote any action on
their behalf other than continuing research and hesitation. Given that most
readers will rarely read anything more than the title, the actual coherence of
the content is often irrelevant. These articles exist purely as fodder to be
shared on social media for sensationalist clickbait titles.

The second increasingly common narrative is the use of cryptocurrency as a
vehicle for economic progress in developing nations. Nations such as Venezuela
and Zimbabwe have suffered horribly under economic mismanagement and corrupt
politicians. The currencies of these countries have experienced hyperinflation
and instability. The narrative pushed by cryptocurrency outlets is that the
citizens of these nations are fleeing their domestic currencies in favor of
digital currencies as a flight to safety. While it is true that there are some
users of cryptocurrencies in these nations, as there are in most
internet-connected countries, there is absolutely no macro trend
\cite{ellsworth_special_2018} of citizens toward bitcoin as a means of exchange
there.  In Venezuela, Bloomberg \cite{vasquez_there_2019} reports a growing
trend in the use of the dollar in the region over the bolivar. There is a small
percentage of bitcoin usage in this region, but relative to the total economic
activity in the bolivar, it represents a minute fraction of transactions.

\index{Venezuela}
\index{bolivar}

\section{Disclosures}

Outlets such as Bitcoinist, CoinTelegraph, and CoinRepublic regularly practice a
form of biased trade journalism in which the specific holdings of their partners
or parent companies are routinely touted as exciting or revolutionary
investments. The largest trade journalism outlet, Coindesk, is operated by the
cryptocurrency hedge fund Digital Currency Group.

During the height of the ICO bubble, investigative journalists looked into the
prices for journalists to promote a given ICO project at various cryptocurrency
outlets. Shockingly the investigation found the prices of an article between
\$240 to a high of \$4500 \cite{faife_we_2018}. Most outlets had no ethical
standard for which to turn away pay-to-post offers. Conversely, these
publications will openly condemn any position that threatens their portfolios or
raises concerns about their commercial partners' specific cryptocurrency
projects.

\index{Coindesk}
\index{Digital Currency Group}

The trade journalism of cryptocurrency is overwhelmingly economically motivated
and does not have the higher standards of established outlets. In the absence of
higher standards, we should neither treat it as a reliable source nor as a
factual representation of events.
