\chapter{Technical Shortcomings}

The success of an emerging technology is not an inevitability, and not all
technical innovation is unqualifiedly good. The software industry's history is
like any other engineering discipline---full of dead ends, false starts, and
wrong turns. Software is unconstrained by many traditional factors; however, it
is ultimately constrained by the economics of its applications and the
limitations of computer science.

The fundamental technical shortcomings of cryptocurrency stem from four major
categories: scalability, privacy, security, recentralization, and
incompatibility with existing infrastructure and legal structures.

\index{recentralization}

\section{Scalability}

The bitcoin technical program carries the specific seed of its own destruction
by virtue of being tied to a political ideology. This ideology opposes any
technical centralization, and this single fact limits the technical avenues the
technology could pursue in scaling. As noted in the second chapter, bitcoin is
inherently an anarchist project with an anti-state mentality that runs deep
within its development community. This ideology informed the initial development
of bitcoin to pursue censorship resistance as a core feature at any cost,
including performance and transaction throughput. By design, the bitcoin network
should allegedly be immune to payment interdiction or law enforcement who wishes
to restrict funds' movement. This guiding principle is the central proposition
any proposed scaling solutions must conform to to be considered acceptable in
the cryptocurrency community, but it is also its technical undoing.

\begin{infobox}
 \textbf{Bitcoin does not scale to transaction counts used by standard payment systems.}
\end{infobox}

The bitcoin scalability problem arises from the consensus model it uses to
confirm blocks of pending transactions. In the consensus model, the batches of
committed transactions are limited in size and frequency and tied to a
proof-of-work model in which miners must perform bulk computations to confirm
and commit the block to the global chain. The protocol constrains a bitcoin
block to be no more than 1MB in size and a single block is committed only every
10 minutes. For comparison, the size of an average 3-minute song encoded in the
MP3 file format is roughly 3.5MB. Doing the arithmetic on the throughput results
in the shockingly low figure that the bitcoin network is only able to do 3-7
transactions per second.

The transaction throughput of bitcoin is very low by traditional
database standards. It is a common marketing tactic in the database industry to
inflate benchmarks or use synthetic workloads that advertise inflated write
speeds for databases. Nevertheless, there are mature and open source databases
such as Microsoft SQL Server, Postgres, and Redis for which we can gather very
accurate information about their write throughputs.  Postgres is a classical
relational SQL database and is capable of doing 200 to 300 updates per second,
or 12,000 transactions per minute. Redis is a key-value store and can do 110,000
writes per second or 6 million transactions per minute. However, since all core
banking solutions use traditional databases as their storage engine, these
numbers represent a baseline figure throughput that should be stated for base
comparison. \cite{katz_bitcoin_2017}

\index{relational database}

An appropriate comparison would be the Visa credit card network, whose
self-reported figures are 3,526 transactions per second. Most credit card
transactions can be confirmed in less than a minute, and the network handles
\$11 trillion of exchange yearly. Credit cards and contactless payments are such
a success story for digital finance that they have become a transparent part of
everyday life that most of us take for granted. The comparison between bitcoin
and Visa is not perfect, as Visa can achieve this level of transaction
throughput by centralizing transaction handling through its own servers and has
spent thirty years building services to handle this kind of load. The slow part
of transaction handling is always compliance, ensuring parties are solvent, and
detecting patterns of fraudulent activity. However, for the advocates proposing
that bitcoin can handle retail transaction loads on a global scale, this is the
definitive benchmark that must be reached for technical parity.

\index{Visa}
\index{block size}
\index{bitcoin!transaction speed}
\index{bitcoin!transaction throughput}

\begin{infobox}
 \textbf{
    There is a mistaken conflation between success as a digital currency and
    success as a speculative investment. These two classes of financial
    products have opposite properties. Currencies require stability and
    reliability in price.
  }
\end{infobox}

\index{bitcoin!as speculative asset}

Building these payment processing systems can be viewed through a lens of
compromise between three factors: scalability, decentralization, and security.
The design choices of bitcoin favor decentralization and security while making a
sacrifice in scalability. The database infrastructure behind standard money
transmitter services is designed to be scalable and secure.

\index{money transmitter}

The scalability issues of the bitcoin protocol are universally recognized, and
there have been many proposed solutions that alter the protocol itself.
\cite{hinzen_proof--works_2019} Bitcoin development is a collaboration between three
spheres of influence: the exchanges who onboard users and issue the bulk of
transactions, the core developers who maintain the official clients and define
the protocol in software, Furthermore, the miners who purchase the physical
hardware and mine blocks. The economic incentives of all of these groups are
different, and a change to the protocol would shift the profit centers for each
of the groups. For example, while the exchanges would be interested in larger
block sizes (i.e., more transactions) the miners (who prioritize fee-per-byte)
would have to purchase new hardware and receive less in mining rewards for more
computational work and thus more significant electricity cost. This stalemate of
incentives has led to mass technical sclerosis of the base protocol and a
situation in which core developers are afraid of major changes to the protocol
for fear of upsetting the economic order they are profiting from.

\index{miner}
\index{bitcoin!scalability problem}

It is a common joke in software development that the answer to any difficult
technical problem is simply to add another level of indirection to the problem.
This leads us to our new problem: the lightning network. Since the base protocol
is unscalable, the seemingly natural solution to adapting this network is to add
yet another network on top of the bitcoin network. This proposed design of this
system would batch settlements between peers into bidirectional state channels.
These state channels are managed by a smart contract but must be monitored by
the two parties on both ends to efficiently close the channel when a batch of
transactions is finalized. This design opens a small but non-zero time window
for fraud in this system in which one party will broadcast an old state to the
contract and can extract the remaining bitcoin locked in the contract before the
other party has finalized. The proposed solution is either a central registry in
which lightning network participants would suffer reputational damage for this
kind of fraudulent transaction or yet another level of indirection known as a
watchtower contract. That watchtower is another smart contract that monitors the
first contracts looking for mismatches between the main network and the channel
states.

\index{lightning network}

The lightning network itself introduces a whole new set of attack vectors for
double spends and frauds as outlined in many cybersecurity papers such as the
"Flood and Loot" attack \cite{harris_flood_2020}. This attack effectively allows
attackers to make specific bulk attacks on state channels to drain users' funds.
The lightning network is an experiment and untested approach to scaling, and
progress on this scaling approach has stagnated since 2018. According to
self-reported lightning network statistics, less than 0.001\% of circulating
bitcoin were being managed by the network, and transaction volume has remained
relatively flat after 2019. No merchants operate with the Lightning network for
payments and as of today is nothing more than a prototype. There is little
evidence supporting this scaling model even works and would not introduce
implicit custodial requirements, novel attack vectors, or new mechanisms for
fraud. The perpetual narrative around the Lightning Network is that it has
always been 18 months away from adoption. Moreover, this narrative is updated
every 18 months it fails to deliver. In software parlance, the term vapourware
is used to describe software perpetually in the works of being developed but
never materializing into a usable form. The bitcoin lightning network is pure
vapourware. \cite{rosenthal_stanford_nodate}

\begin{infobox}
 \textbf{Bitcoin transactions are considerably more expensive than any other means of payment.}
\end{infobox}

Outside of the bitcoin network, there are similar problems in other
cryptocurrencies. The bitcoin meme of technical indirection through Layer 2
solutions have been translated to other systems and their development
philosophies. This perspective views the base protocol as being only a
settlement layer for larger bulk transfers between parties, and those smaller
individual payments should be handled by secondary systems with different
transaction throughputs and consistency guarantees. The ethereum network has
taken a different set of economic incentives in its initial design, and at the
time of writing, this network is still only capable of doing roughly 15
transactions per second. There is a proposed drastic protocol upgrade to this
network known as ethereum 2.0, which includes a fundamental shift in the
consensus algorithm. This project has been in development for five years and has
consistently failed to meet all its launch deadlines, and it remains unclear
when or if this new network will launch. Since this new network would alter the
economics of mining the protocol, it is unclear if there will be community
consensus between miners and developers that the protocol will go live or
whether they will see the same economic stalemate and sclerosis that the bitcoin
ecosystem observes. This rollout may result in a fork of the ethereum chain and
a fracturing of the ecosystem and its incentive structure divisions.

\begin{infobox}
 \textbf{ Because of slow transaction speeds, cryptocurrencies are almost
  impossible to use for legitimate commercial transactions. }
\end{infobox}

The broader cryptocurrency community has seen a zoo of alternative proposed
scaling solutions, these proposals going by technical names such as sidechains,
sharding, DAG networks, zero-knowledge rollups and a variety of proprietary
solutions which make miraculous transaction throughput claims. However, the
tested Nakamoto consensus remains the dominant technology. At the time of
writing, there is little empirical evidence for the viability of new scaling
solutions as evidenced by live deployments with active users. Central to the
cryptocurrency ideology is a belief that this technical problem must be
tractable, and for many users, it is a matter of faith that a future
decentralized network can scale to Visa levels while maintaining censorship
resistance and without centralization. Independent of this faith, the
fundamental reality is that cryptocurrency currently does not scale and cannot
adapt itself to fit the existing realities of how the world transacts.
\cite{kharkharif_bitcoins_2019}

\section{Privacy}

\index{privacy}

Bitcoin wallet addresses are a unique global addressing system derived from the
use of hash functions. In a nutshell, a bitcoin wallet address is generated from
an elliptic curve private key which is a unique number generated randomly when a
bitcoin wallet is created. This number is inconceivably large by everyday
standards and will have hundreds of digits. If generated by a proper random
number generator, the probability of that specific set of digits ever being
generated again during the universe's lifetime is infinitesimally small. This
number satisfies the necessary properties of being a good secret value that the
user holds private and uses to control access to their funds.  There are
$21^{60}$ total possible addresses in the bitcoin protocol. An example wallet
address:

$$
\texttt{1A1zP1eP5QGefi2DMPTfTL5SLmv7DivfNa}
$$

\index{bitcoin!wallet}

The public address generated associated with a wallet is encoded in a format
where uppercase and lowercase letters stand for numerical values. This sequence
of letters and numbers uniquely identifies the endpoint for other users to send
funds to and can be shared publicly in textual format or in a graphical format
such as a QR code. However, no information about the user is contained within
this representation. From base assumptions, the number is essentially anonymous.
In the traditional banking system, a coding known as IBAN (International Bank
Account Number) is the standard numbering system used to identify accounts and
associated financial institutions. When issuing an international wire transfer,
a bank account will ask for the receiving IBAN as part of the transfer. This
number is then mapped internally to the account holder's account and is stored
within the bank's core banking software and routing system.

A bitcoin address is, however, not fully anonymous. The bitcoin ledger itself is
a fully public list of transactions that have ever occurred since the network's
inception. It contains the very first transactions allegedly by Satoshi to the
most recent transactions conducted in the last 10 minutes. The full provenance
of a bitcoin can be traced back to its creation and through every address it
passed through.

This feature means that while accounts are anonymous, the global transaction
data can be used to infer specific properties about when, with whom, and in what
amounts an address is transacting. This kind of information is traditionally
called metadata. For instance, metadata about your text messaging habits may not
contain the direct messages you send. However, given sufficiently large sample
size, it is possible to deduce a person's social network, their life partner,
and coworkers from the frequency and timing of messages. Likewise, a great deal
of information can be deduced from tracing the provenance of a bitcoin address,
and thus bitcoin addresses are not entirely anonymous but partially anonymous or
\textit{pseudonymous}.

\index{pseudonymous}

The tracking and tracing of bitcoin involved in criminal activities has emerged
as a standard practice domain in law enforcement and emerging companies like
ChainAnalysis have been able to deduce quite a bit of implied information simply
from public information. Unlike with bank accounts, law enforcement, does not
require a subpoena of public information for an ongoing investigation.
Notoriously many users of darknet services such as the Silk Road were caught
because of a misunderstanding about the transparency of the bitcoin ledger used
by these actors.

\index{Silk Road}
\index{darknet}

Acquiring bitcoin has always had a bootstrapping problem for new users. In the
early days of the protocol, one could use a home computer to mine small amounts
by devoting spare CPU cycles to generate small amounts. However, for the last
nine years, this has been economically unviable. These days, the traditional
onramp is to go through a domestic exchange or one of the offshore services. In
the case of exchanges domiciled in the United States or in Europe, the
onboarding process for accounts requires the account holder to present a
government-issued identification and proof of address. This process is similar
to opening a bank account and provides a mechanism for the institution to
contact you and alert law enforcement of any suspicious or criminal activity
associated with the account opened. This is a legal requirement known as
\textit{Know Your Customer} or \textit{KYC} is the legal requirement to maintain
an audit log of the account holder's personal information and account activity.

\index{KYC}
\index{United States}
\index{United Kingdom}

Since the exchanges themselves operate accounts with massive inflows and
outflows of transactions, their wallet accounts are massive hubs of activity
that can easily be observed in the global ledger. If the exchanges are operating
in a compliant manner, then every transaction they process should internally be
mapped to metadata about the account holders and their respective information.
If an account was associated with criminal activity, law enforcement could
subpoena the exchange and demand the information required to trace the account
back to an individual. This information chasing through account metadata is the
mechanism by which money laundering and wire fraud cases can be prosecuted.

\index{money laundering}

This is in contrast to how the traditional banking system works, where bank
secrecy laws are a central part of the obligation between a bank and its
customers. Banks cannot use the transaction flows of their customers as part of
their investments or share this information with other parties unless required
by the courts. Bank transactions are required to be secure, private, and
generally confidential information. When a wire transfer is issued by a company
whose corporate account is at HSBC in London to Morgan Stanely in New York City,
the metadata contained within that transaction could contain commercially
sensitive information. For example, if a British company is sending large
amounts of funds to a newly created American division, it may indicate the
intent for the company to expand into the American market. There are cases where
the constellation of transactions between known entities could be used to deduce
confidential information about the parties. However, this fact poses an
existential question about the efficacy of cryptocurrency networks as an
international payment system if pseudonymous accounts leak information.

A retail bank account held by individuals is usually a simple structure
periodically collects deposits from an employer and frequent small debits for
everyday activities such as groceries, rent, and buying coffee.  Corporate
banking, especially for large multinational corporations, can be quite complex
and span a significant number of accounts and institutions.

However, a corporation that wanted to transact in cryptocurrency would have to
address the fundamental issue that inflows and outflows from their accounts are
commercially sensitive information. The amount of money that a corporation pays
in payroll correlates with staffing and their operating expenses in specific
regions and divisions in the company. The accounts receivable correlates with
invoices it collects, its commercial interactions with its clients, and its
lines of business. Public metadata of any of these transactions is intensely
private information that is usually some of the most protected information
inside a corporation. It is usually only shared with auditors and its direct
banking relationships. Both of these parties are professionally and legally
bound to confidentiality. A company electing to transact in cryptocurrency would
leak confidential information like a sieve by choosing this public mode of
payment.

The technical answer that one might propose to this problem is that the
corporate should create a network of wallets and shuffle the payments between
the wallets in random amounts and times to obscure the provenance of funds. This
solution is needlessly complicated for a traditional corporate treasurer who
should not perform this level of financial obfuscation and needless overhead for
their normal daily activities. This solution is also indistinguishable from the
money laundering process used by criminals. For cryptocurrency to pose any value
to the commercial banking sector, this question requires a good answer.

\index{money laundering}

If we step back, this conundrum begs a more profound question. Why are we making
what was once a non-problem into a complex problem? Since banking was invented
in Florence in the 13th century, the privacy problem has been solved. A
mixed-visibility network with some access to authorities and privacy otherwise
works very well. What does cryptocurrency offer except creating new problems?

\section{Security}

The standard advice around the custodianship of cryptocurrency is that one
should "be your own bank" and "if you do not hold your keys, they are not your
coins". These idioms are related to the fact that cryptocurrency is a
\textit{bearer instrument}, and if one holds the private keys to a set of funds,
they are effectively in control of the assets, just as if one physically held
euros or dollars in a wallet. The problem becomes when these funds are held by
an exchange account which holds funds before they withdraw. These exchanges are
not banks, are not legally bound to hold deposits, and most likely are not
colocated in the customer's jurisdiction. Most cryptocurrency exchanges provide
no legal recourse for lost funds, and the funds held are not insured under any
deposit insurance scheme.

\begin{infobox}
 \textbf{Lost private keys account have resulted in 20\% of the supply of
  bitcoins being irretrievably lost.}
\end{infobox}

In addition, these exchanges are some of the most targeted entities on the
planet for hackers. In 2019 twelve major exchanges were hacked and \$292 million
equivalent was stolen in these attacks. These events have only increased in
severity and frequency over time and in conjunction with bubble economics
involved.

While some best practices can mitigate this risk, the fundamental design of
bitcoin-style systems is that end-user is responsible for their own keys and
wallets by safeguarding their cryptographic secrets. This can be done through
several strategies. So-called cold wallets are wallet keys stored in physical
objects such as paper and not connected to electronic devices. Other systems
such as hardware wallets allow users to secure and encrypt their keys on a
dedicated hardware device.

Cybersecurity is one of our era's biggest problems, and companies with
significant information security budgets and dedicated teams regularly fail. A
system that requires every depositor to have the same level of security as a
chief information security officer and constantly be aware of threat vectors and
potential attacks on key storage is an enormous cognitive overhead. At face
value, this seems like an unnecessary burden on an average user who simply wants
to hold funds and be protected against fraud in their daily transactions. The
ask of individuals to supply their own banking institution-level information
security is highly unreasonable.

In the course of human life, many situations occur which require third parties
to be able to access or reset our accounts. If one forgets a PIN or loses a
credit card, there is a simple mechanism by which one can go to the banking
branch, prove their identity, and restore access to their funds. In a more
extreme case of an untimely death, a person's funds will be passed along to
their spouse or children through inheritance and wills. The successors can
petition the bank for access to the funds by presenting a death certificate and
gaining control of the deceased accounts. Being one's own bank makes both cases
either impossible or needlessly complex. The human mind is fragile and subject
to decay, mental disorders, and memory loss. If one forgets the passwords to
their hardware wallet or is physically destroyed, they will lose access to their
accounts. These events have already occurred to even some of the most
sophisticated investors. An elaborate setup of data backups could mitigate this,
safety deposit boxes or multiparty wallet setups, but these technical solutions
are an unnecessary complexity burden for most users.

There are many news stories of ransom, kidnapping, and murder of crypto asset
holders who attempted to safeguard their wallets personally. In cybersecurity,
the term rubber-hose cryptanalysis satirically refers to extracting
cryptographic secrets from a person by coercion or torture. A digital attack
vector is unnecessary if criminals could extract the keys by kidnapping and
torturing the owner and then laundering the funds from anywhere in the world.

Of course, the natural solution to this would simply be that most users should
not be their own bank; instead, they should use a "cryptobank" which holds their
funds and provides them access. However, this ultimately is just recreating the
same centralized authority system which cryptocurrency advocates attempted to
replace. Providing cryptocurrency security for the masses either introduces more
social problems the technology has no answer to or results in a recentralization
that undermines its own ideological goals. After all, we already have
centralized banks and existing payment systems that work just fine.

\section{Compliance}

The fundamental reality of international commerce is that money management has
been vital to a nation's sovereignty and its ability to manage its economic
growth and security. The movement, storage, and handling of money are regulated,
and most countries have laws on the international movement of funds. Showing up
at an airport in Berlin with undeclared cash above €10,000 will land one in
quite a bit of trouble. If the value proposition of cryptocurrency is
international money movement or extranational stores of value, then the
technology will have to conform to existing regulations at entry points and exit
points.

As a point of reference, it is helpful to consider how money transfers currently
work. Nations with advanced economies will have a domestic settlement system
that allows banks within a regulatory regime to transfer funds between entities
quickly. The United Kingdom has FasterPayments, Australia has BPAY, and the
United States has CHIPS (Clearing House Interbank Payments System). These
systems act as netting engines between the banks where trades are netted against
each other instead of the total amount of all trades being cleared on every
transaction. A trade goes through two steps. The first is *clearing* which is
the confirmation of information between the payer and payee, and the second is
*settlement*, which is the actual transfer of funds. For financial institutions
to transfer funds, they will have what is called a *Vostro account* of the other
bank, which records the funds held by the current bank on behalf of the other.
Conversely, the other bank will have its Nostro account, which is an account
held by the other bank which holds the current bank's money. Transfers between
the banks will be debited and credited in their respective Vostro accounts, thus
allowing them to transfer money.

\index{vostro account}
\index{CHIPS}

International wire transfers are done on the SWIFT (Society for Worldwide
Interbank Financial Telecommunication) network, which forms the messaging
systems by which banks communicate messages about international transfers. The
SWIFT network does not move money itself but simply is a messaging protocol for
institutions to communicate the intent of transfers to happen. In addition,
banks can only work directly with overseas banks with whom it holds accounts;
this is known as a *correspondent account*. If a bank does not have a
correspondent banking relationship, it will have to route the wire through a
third-party bank. This process entails having a Vostro account of a foreign bank
or going through a chain of correspondent banks that do.

\index{SWIFT}
\index{wire transfers}

Every hop along this chain incurs compliance checks with domestic laws and often
involves multiple human and technical touchpoints inside or the organization to
complete the wire. At each step along this process each party involved in the
transaction chain is legally required to carry out AML (anti-money laundering),
and sanctions check to ensure that the transfer complies with domestic laws and
international treaties. If these checks are complete, the transfer is completed
and the money will be credited to the target account. The fees associated with
These transfers are deducted from the total amount and represent the operational
costs of performing all of these compliance checks along the way.

\index{sanctions}
\index{anti-money laundering}

\begin{infobox}
 \textbf{
   Cryptocurrency is purposefully built to evade regulation and make compliance impossible and is thus incompatible with cryptocurrency existing under the rule of law.
  }
\end{infobox}

The bottleneck along this process is never the technical transmission of the
messages. Like any modern electronic messaging system, they are almost
instantaneous. Any human touchpoint will be subject to the bank's operating
hours and days on which they are open for business, which is often only business
hours and workdays.

Regular fintech companies like Wise (previously known as TransferWise) have come
up with alternative solutions to international payments for small amounts that
everyday customers send often. Since most retail transactions are small (less
than \$5000 per day). Wise's internal system matches users attempting to send
small amounts in one currency block with corresponding users sending amounts in
the opposite currency block.  Wise uses these pools of funds to net out
aggregate transactions via local bank transfers.

The inability to move money from a country is ultimately one of domestic
internal infrastructure development and external international relations, not
technical limitations. Moreover, the proposed use case for cryptocurrency as a
mode of international remittances is fundamentally limited because of a lack of
a coherent compliance story. Even if we were to use cryptocurrency as a
hypothetical international settlement medium, this system has not removed
financial institutions from the equation. The system's entry and exit points
would have to perform the same checks of outgoing and incoming money flow
required by many international agreements.

\index{remittances}

In this hypothetical scenario, we have simply shifted the custodial, compliance,
and identity management responsibilities to a different centralized entity that
performs precisely the same activities and ultimately is subject to the same
legal liabilities. In this setup, instead of settling in a national currency
pair, there are now two currency pairs with a useless and volatile intermediary
step in between. Using cryptocurrency for remittance has not disintermediated
anything; it has simply shifted the intermediaries and introduced another level
of indirection for no reason.

A system that aimed to replace the existing international transfers would be
subject to a similar set of rules regarding international transfers and capital
controls, and it is naive to think hundreds of treaties would be renegotiated on
behalf of digital currencies. Of course, like all cryptocurrency arguments, the
counterargument is an ideological one. Compliance is a non-issue because
nation-states should not exist and should not have capital controls. This
ideological goal is inexorably embedded in cryptocurrency's design, making it
unscalable and untenable technology for any real-world application where
sanctions, laws, and compliance are an inescapable part of doing business in
financial services.

\index{anarchy}
