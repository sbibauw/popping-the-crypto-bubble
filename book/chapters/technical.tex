\chapter{Technical Shortcomings}

The success of an emerging technology is not an inevitability. The history of
the software industry is like any other discipline of engineering -- full of
dead ends, false starts and wrong turns. Software is unconstrained by many
traditional factors however it is ultimately constrained by the economics of its
applications and the limitations of computer science.

The fundamental technical shortcomings of cryptocurrency stem from four
categories: scalability, privacy, security, and incompatibility with existing
infrastructure and legal structures.

\section{Scalability}

The bitcoin technical program carries the specific seed of its own destruction
by virtue of being tied to a political ideology. This ideology opposes any
technical centralization, and this single fact limited the technical avenues the
technology could pursue in scaling. As noted in the second chapter, bitcoin is
inherently an anarchist project with an anti-state mentality that runs deep
within its development community. This ideology informed the initial development
of bitcoin to pursue censorship resistance as a core feature at any cost,
including performance and transaction throughput. By design the bitcoin network
should allegedly be immune to payment interdiction or any law enforcement who
would wish to restrict the movement of funds. To many in the community this is
the central proposition on which any proposed scaling solutions must conform.

\begin{infobox}
 \textbf{Bitcoin does not scale to transaction counts used by normal payment systems.}
\end{infobox}

The bitcoin scalability problem arises from the consensus model it uses to
confirm blocks of pending transactions. In the consensus model the batches of
transactions that are committed are limited in size and frequency, and tied to a
proof-of-work model in which miners must perform bulk computations to confirm
and commit the block to the global chain. A bitcoin block is constrained by the
protocol to be no more than 1MB in size and a single block is committed every 10
minutes.  For comparison the size of an average 3 minute song encoded in the MP3
file format is roughly 3.5MB. Doing the arithmetic on the throughput results in
the shockingly low figure that the bitcoin network is only able to do 3-7
transactions per second.

The transaction throughput of bitcoin is very low by traditional database
standards. It is a common marketing tactic in the database industry to inflate
one's benchmarks or use synthetic workloads which advertise inflated write
speeds for databases. Nevertheless there are mature and open source databases
such as Postgres and Redis for which we can gather very accurate information
about their write throughputs.  Postgres is a classical relational SQL database
and is capable of doing 200 to 300 updates per second, or 12,000 transactions
per minute. Redis is a key-value store and is capable of doing 110,000 writes
per second or 6 million transactions per minute. However since all core banking
solutions use traditional databases as their storage engine, these numbers
represent a baseline figure throughput that should be stated for base
comparison. \cite{katz_2017}

An appropriate comparison would be the Visa credit card network whose
self-reported figures are 3,526 transactions per second.  Most credit card
transactions can be confirmed in less than a minute and the network handles \$11
trillion dollars of exchange yearly.  Credit cards and contactless payments are
such a success story for digital finance that they have become a transparent
part of everyday life that most of us take for granted. The comparison between
bitcoin and Visa is not a perfect comparison as Visa is able to achieve this
level of transaction throughput by centralising transaction handling through its
own servers and has spent thirty years building services to handle this kind of
load. The slow part of transaction handling is always compliance, ensuring
parties are solvent and detecting patterns of fraudulent activity. However
for the advocates proposing that bitcoin is capable of handling retail
transaction loads on a global scale, this is the definitive benchmark that
must be reached for technical parity.

\index{Visa}
\index{bitcoin!transaction speed}

\begin{infobox}
 \textbf{
    There is mistaken conflation between success as a digital currency and
    success as speculative investment. These two classes of financial products
    have opposite properties. Currencies require stability and reliability in
    price.
  }
\end{infobox}


Building these payment processing systems can be viewed through a lens of
compromise between three factors: scalability, decentralization and security.
The design choices of bitcoin favour decentralization and security, while making
a sacrifice in scalability. The database infrastructure behind normal money
transmitter services are designed to be scalable and secure.

\index{money transmitter}

The scalability issues of the bitcoin protocol are universally recognized and
there have been many proposed solutions which alter the protocol itself.
\cite{hinzen2019proof} Bitcoin development is a collaboration between three
spheres of influence: the exchanges who onboard users and issue the bulk of
transactions, the core developers who maintain the official clients and define
the protocol in software, and the miners who purchase the physical hardware and
mine blocks. The economic incentives of all of these groups are different and a
change to the protocol would shift the profit centers for each of the groups.
For example, while the exchanges would be interested in larger block sizes
(i.e.~more transactions) the miners (who prioritize fee-per-byte) would have to
purchase new hardware and receive less in mining rewards for more computational
work and thus greater electricity cost. This stalemate of incentives has led to
mass technical sclerosis of the base protocol and a situation in which core
developers are afraid of major changes to the protocol for fear of upsetting the
economic order they are profiting from.

\index{miner}
\index{bitcoin!scalability problem}

It is a common joke in software development that the answer to any difficult
technical problem is simply to add another level of indirection to the problem.
This leads us to our new problem: Lightning Network. Since the base protocol is
unscalable the seemingly natural solution to adapting this network is to add yet
another network on top of the bitcoin network. This proposed design of this
system would batch settlements between peers into bidirectional state channels.
These state channels are managed by a smart contract but must be monitored by
the two parties on both ends to efficiently close the channel when a batch of
transactions are finalized. This opens a small, but non-zero, time window for
fraud in this system in which one party will broadcast an old state to the
contract and is able to extract remaining bitcoin locked in the contract before
the other party has finalized. The proposed solution is either a central
registry in which lightning network participants would suffer reputational
damage for this kind of fraudulent transaction or yet another level of
indirection known as watchtower contract. That watchtower is another smart
contract that monitors the first contracts looking for mismatches between the
main network and the channel states.

\index{Lightning network}

The lightning network itself introduces a whole new set of attack vectors for
double spends and frauds as outlined in many cybersecurity papers such as the
``Flood and Loot'' attack \cite{harris2020flood}.  This attack effectively
allows attackers to do specific bulk attacks on state channels to drain user's
funds. The lightning network is an experiment and untested approach to scaling
and progress on this scaling approach has stagnated since 2018. According to
self-reported lightning network statistics less than 0.001\% of circulating
bitcoin were being managed by the network and transaction volume has remained
relatively flat after 2019. No merchants operate with the Lightning network for
payments and as of today is nothing more than a prototype. There is little
evidence to support that this scaling even works and would not introduce
implicit custodial requirements, novel attack vectors, or new mechanisms for
fraud. The perpetual narrative around the Lightning network is that it has
always been 18 months away from adoption and has been on this timeline for quite
some time. In software parlance the term vapourware is used to describe software
which is perpetually in the process of being developed but not materialises into
a usable form, the bitcoin lightning network is a vapourware.

\begin{infobox}
 \textbf{ Bitcoin transactions are considerably more expensive than any other means of payment.  }
\end{infobox}

Outside of the bitcoin network there are similar problems in other
cryptocurrencies. The bitcoin meme of technical indirection through Layer 2
solutions has been translated to other systems and their development
philosophies. This perspective views the base protocol as being only a
settlement layer for larger bulk transfers between parties and that smaller
individual payments should be handled by secondary systems with different
transaction throughputs and consistency guarantees. The ethereum network has
taken a different set of economic incentives in their initial design and at the
time of writing this network is still only capable of doing roughly 15
transactions per second. There is a proposed drastic protocol upgrade to this
network known as ethereum 2.0 which includes a fundamental shift in the
consensus algorithm. This project has been in development for five years and has
consistently failed to meet all its launch deadlines, and it remains unclear when or
if this new network will launch. Since this new network would alter the
economics of mining the protocol, it is unclear if there will be community
consensus between miners and developers that the protocol will go live or
whether they will see the same economic stalemate and sclerosis that the bitcoin
ecosystem observes. This rollout may result in fork of the ethereum chain and a
fracturing of the ecosystem along its incentive structure divisions.

\begin{infobox}
 \textbf{ Because of slow transaction speeds, cryptocurrencies are almost
  impossible to use for legitimate commercial transactions. }
\end{infobox}

The wider cryptocurrency community has seen a zoo of alternative proposed
scaling solutions, these proposals going by technical names such as: sidechains,
sharding, DAG networks, zero knowledge rollups and a variety of proprietary
solutions which make miraculous transaction throughput claims.  However the
tested Nakamoto consensus remains the dominant technology. At the time of
writing there is little empirical evidence for the viability of new scaling
solutions as evidenced by live deployments with active users. Central to the
cryptocurrency ideology is a belief that this technical problem must be
tractable, and for many users it is a matter of faith that a future
decentralized network can scale to Visa levels while maintaining censorship
resistance and without centralization. Independent of this faith, the
fundamental reality is that cryptocurrency currently does not scale and cannot
adapt itself to fit the existing realities of how the world transacts.

\section{Privacy}

\index{privacy}

Bitcoin wallet addresses are a global unique addressing system that is derived
from the use of hash functions. In a nutshell a bitcoin wallet address is
generated from an elliptic curve private key which is a unique number generated
randomly when a bitcoin wallet is created. This number is inconceivably large by
everday standards and will have hundreds of digits. If generated by a proper
random number generator the probability of that specific set of digits ever
being generated again during the lifetime of the universe is infinitesimally
small. This number satisfies the necessary properties of being a good secret
value which the user holds private and uses to control access to their funds.
There are $21^{60}$ total possible addresses in the bitcoin protocol. An example
wallet address:

$$
\texttt{1A1zP1eP5QGefi2DMPTfTL5SLmv7DivfNa}
$$

\index{bitcoin!wallet}

The public address generated associated with a wallet is encoded in a format in
which uppercase and lowercase letters stand for numerical values. This sequence
of letters and numbers uniquely identifies the endpoint for other users to send
funds to and can be shared publicly in textual format or in a graphical format
such as QR code. However no information about the user is contained within this
representation, from first assumptions the number is anonymous. In the normal
banking system a coding known as IBAN (International Bank Account Number) is the
standard numbering system used to identify accounts and their associated
financial institutions. When issuing an international wire transfer a bank
account will ask for the receiving IBAN number as part of the transfer.  This
number is then mapped internally to the account holders account and is stored
within the bank's core banking software and routing system.

A bitcoin address is however not fully anonymous. The bitcoin ledger
itself is a fully public list of transactions that have ever occurred
since the inception of the network. It contains the very first
transactions allegedly by Satoshi all the way to the most recent
transactions conducted in the last 10 minutes. The entire provenance of
every bitcoin can be traced back to its creation and through every
address it has passed through.

This feature means that while accounts are anonymous, the global
transaction data can be used to infer specific properties about when,
with whom, and in what amounts an address is transacting with. This kind
of information is traditionally called metadata. For instance metadata
about your text messaging habits may not contain the direct messages you
send. However given a sufficiently large sample size it is possible to
deduce a person's social network, their partner, and coworkers from the
frequency and timing of messages. Likewise a great deal of information
can be deduced from tracing the provenance of a bitcoin address and thus
bitcoin addresses are not fully anonymous but partially anonymous or
\textit{pseudonymous}.

\index{pseudonymous}

The tracking and tracing of bitcoin involved in criminal activities has
emerged as a common practice domain in law enforcement and emerging
companies like ChainAnalysis have been able to deduce quite a bit of
implied information simply from public information. Unlike with bank
accounts, law enforcement does not require a
subpoena of public information for an ongoing investigation. Notoriously
many users of darknet services such as the Silk Road were caught because
of a misunderstanding about the transparency of the bitcoin ledger used
by these actors.

\index{Silk Road}
\index{darknet}

The process of acquiring bitcoin has always had a bootstrapping problem for new
users. In the early days of the protocol one could simply use a home computer to
mine small amounts by devoting spare CPU cycles to generate small amounts.
However for the last nine years this has been economically unviable. The
traditional onramp these days is to go through a domestic exchange or one of the
offshore services. In the case of exchanges domiciled in the United States or in
Europe the onboarding process for accounts requires the account holder to
present a government issued identification and proof of address. This is similar
to opening a bank account and provides a mechanism for the institution to
contact you and alert law enforcement of any suspicious or criminal activity
associated with the account opened. This is a legal requirement known as
\textit{Know Your Customer} (KYC) and is the legal requirement to maintain an
audit log of account holder personal information and account activity.

\index{KYC}
\index{United States}
\index{United Kingdom}

Since the exchanges themselves operate accounts with massive inflows and
outflows of transactions, their wallet accounts are massive hubs of activity
that can easily be observed in the global ledger. If the exchanges are operating
in a compliant manner then every transaction they process should internally be
mapped to metadata about the account holders and their respective information.
In the case where an account was associated with criminal activity then law
enforcement could subpoena the exchange and demand the information required to
trace the account back to an individual. This information chasing through
account metadata is the mechanism by which money laundering and wire fraud cases
can be prosecuted.

\index{money laundering}

This is in contrast to how the traditional banking system works, where
bank secrecy laws are a central part of the obligation between a bank
and its customers. Normally banks cannot use the transactions flows of
their customers as part of their investments or share this information
with other parties unless required to by the courts. Bank transactions
are required to be secure, private, and generally confidential
information. When a wire transfer is issued by a company whose corporate
account is at HSBC in London to Morgan Stanely in New York City, the
metadata contained within that transaction could contain commercially
sensitive information. For example if a British company is sending large
amounts of funds to a newly created American division it may indicate
the intent for the company to expand into the American market. There are
any number of cases where the constellation of transactions between
known entities could be used to deduce confidential information about
the parties. This however poses an existential question about the
efficacy of cryptocurrency networks as an international payment system
if pseudonymous accounts leak information.

A retail bank account held by individuals is usually a simple structure which
infrequently collects deposits (payroll, etc) and frequently issues small debits
for everyday activities (groceries, rent, etc).  Corporate banking, especially
for large multinational corporations, can be quite complex and span a great
number of accounts and institutions.

However a corporation that wanted to transact in cryptocurrency would
have to address the fundamental issue that inflows and outflows from
their accounts are commercially sensitive information. The amount of
money that a corporate pays in payroll correlates with staffing and
their operating expenses in specific regions and divisions in their
company. The accounts receivable correlates with invoices it collects
and its commercial interactions with its clients and its lines of
business. Public metadata of any of these transactions is deeply private
information that is usually some of the most protected information
inside of a corporation. It is usually only shared with auditors and its
direct banking relationships. Both of these parties are professionally
and legally bound to confidentiality. A company electing to transact in
cryptocurrency would leak confidential information like a sieve by
choosing this public mode of payment.

The technical answer that one might propose to this problem is that the
corporate should create a network of wallets and shuffle the payments between
the wallets in random amounts and times to obscure the provenance of funds. This
solution is needlessly complicated for a traditional corporate treasurer who
shouldn't perform this level of financial obfuscation and needless overhead for
their normal daily activities. For cryptocurrency to pose any value to the
commercial banking sector this question requires a good answer.

If we step back, this conundrum begs a more profound question. Why are we making
what was once a non-problem into a difficult problem. The privacy problem has
been solved since banking was invented in Florence in the 13th century. A
mixed-visibility network with some access to authorities and privacy otherwise
works very well. What does cryptocurrency offer except to create new problems?

\section{Security}

The common advice around the custodianship of cryptocurrency is that one should
``be your own bank" and ``if you don't hold your keys, they aren't your
coins". Both of these idioms related to the cryptocurrency are a \textit{bearer
instrument}, if one holds the private keys to a set of funds then they are
effectively in control of the assets. Just as if you physically hold euros or
pounds in your wallet they are indeed your bills. The problem becomes when these
funds are held by an exchange account which holds your funds before they
withdraw. These exchanges are not banks, are not legally bound to hold deposits
and most likely are not in your jurisdiction. Most cryptocurrency exchanges
provide no legal recourse for lost funds and the funds held are not insured
under any deposit insurance scheme.

\begin{infobox}
 \textbf{Lost private keys account have resulted in 20\% of the supply of
  bitcoins being irretrievably lost.}
\end{infobox}

In addition these exchanges are some of the most targeted entities on the planet
for hackers. In 2019 twelve major exchanges were hacked and \$292 million
equivalent was stolen in these attacks. These events have only increased in
severity and frequency over time and in conjunction with bubble economics
involved.

While some best practices can mitigate this risk, the fundamental design
of bitcoin-style systems is that an end user is responsible for their
own keys and wallets by safeguarding their cryptographic secrets. This
can be done through several strategies. So-called cold wallets are
wallet keys that are stored in physical objects such as paper and not
connected to electronic devices. Other systems such as hardware wallets
allow users to secure and encrypt their keys on a dedicated hardware
device.

Cybersecurity is one of the biggest problems of our era and companies with the
large information security budgets and dedicated teams regularly fail. A system
which requires every depositor to have the same level of security as a chief
information security officer and constantly be aware of threat vectors and
potential attacks on your keys is an enormous cognitive overhead. At face value
this seems like an unnecessary burden on an average user who simply wants to
hold funds and be protected against fraud in their daily transactions. Providing
your own banking-level information security is extremely hard.

In the course of human life many situations occur which require third
parties to be able to access or reset our accounts. In the simple case
if one forgets a PIN or loses a credit there is a simple mechanism by
which one can go to the banking branch, prove their identity, and
restore access to their funds. In a more extreme case of an untimely
death a person's funds will be passed along to their spouse or children
through inheritance and wills. The successors can petition the bank for
access to the funds by presenting a death certificate and gaining
control of the deceased accounts. Being your own bank makes both of
these cases either impossible or needlessly complex. The human mind is
also fragile and subject to decay, mental disorders and memory loss. If
one forgets the passwords to their hardware wallet or it is physically
destroyed they will lose access to their accounts. This has already
occurred to even some of the most sophisticated investors. This could be
mitigated by an elaborate setup of data backups, safety deposit boxes or
multiparty wallet setups but these technical solutions are again an
unnecessary burden of complexity for most users.

There are many news stories of ransom, kidnapping, and murder of holders of
cryptocurrency assets who attempted to personally safeguard their wallets. In
cybersecurity the term rubber-hose cryptanalysis refers satirically to the
extraction of cryptographic secrets from a person by coercion or torture. A
digital attack vector is simply unnecessary if criminals could simply extract
the keys by kidnapping and torturing the owner, and then laundering the funds
from anywhere in the world.

Of course the natural solution to this would simply be that most users shouldn't
be their own bank, instead they should use a ``cryptobank'' which holds their
funds and provides them access. However this ultimately is just recreating the
same centralised authority system which cryptocurrency advocates attempted to
replace. Providing cryptocurrency security for the masses either introduces more
social problems for which the technology has no answer, or ultimately results in
a centralization which undermines its own ideological goals. After all we
already have banks and payment systems that already work.

\section{Compliance}

The fundamental reality of international commerce is that the management
of money has been vital to a nation's sovereignty and its ability to
manage it's economic growth and security. The movement, storage and
handling of money is a regulated activity and most countries have
regulations on the international movement of funds. Showing up at an
airport in Berlin with undeclared cash in excess of €10,000 will land
one in quite a bit of trouble. If the value proposition of
cryptocurrency is international money movement or extranational stores
of value then the technology will have to conform to existing
regulations at entry points and exit points.

As a point of reference it is helpful to consider how money transfers currently
work. Nations with advanced economies will have a domestic settlement system
which allows banks within a regulatory regime to quickly transfer funds between
entities. The United Kingdom has FasterPayments, Australia has BPAY, and the
United States has CHIPS (Clearing House Interbank Payments System). These
systems act as netting engines between the banks where trades are netted against
each other instead of the full amount of all trades being cleared on every
transaction. A trade goes through two steps. The first is \textit{clearing}
which is the confirmation of information between the payer and payee and the
second is \textit{settlement} which is the actual transfer of funds. For
financial institutions to transfer funds they will have what is called a
\textit{vostro account} of the other bank which records the amount of funds held
by the current bank on behalf of the other. Conversely the other bank will have
a its nostro account which is an account held by the other bank which holds current
banks money. Transfers between the banks will be debited and credited in their
respective vostro accounts thus allowing them to transfer money between each
other.

\index{vostro account}

\index{CHIPS}

International wire transfers are done on the SWIFT (Society for
Worldwide Interbank Financial Telecommunication) network which forms the messaging
systems by which banks communicate messages about international transfers. The
SWIFT network doesn't move money itself but simply is a messaging protocol for
institutions to communicate the intent of transfers to happen. In addition banks
can only work directly with overseas banks with whom it holds accounts, this is
known as a \textit{correspondent account}. If a bank does not have a
correspondent banking relationship, they will have to route the wire through a
third party bank who does. This amounts to having a vostro account of a foreign
bank, or going through a chain of correspondent banks who do.
\index{SWIFT}
\index{wire transfers}

Every hop along this chain incurs compliance checks with domestic laws
and often involves multiple human and technical touch points inside or
the organisation to complete the wire. At each step along this process
they each party involved in the transaction chain is legally required to
carry out AML (anti-money laundering) and sanctions checks to ensure
that the transfer complies with domestic laws and international
treaties. If these checks complete, the transfer is completed and the
money will be credited to the target account. The fees associated with
this transfer are deducted from the total amount and represent the
operational costs of performing all of these compliance checks along the
way.

\index{sanctions}
\index{anti-money laundering}

\begin{infobox}
 \textbf{
   Cryptocurrency is purposefully built to evade regulation and make compliance
   impossible. This is incompatible with it existing under the rule of law.
  }
\end{infobox}

The bottleneck along this process is never the technical transmission of the
messages. Like any modern electronic messaging systems they are almost
instantaneous. Any human touchpoint will be subject to operating hours of the
bank and days on which they are open for business which is often only business
hours and work days.

Normal fintech companies like Wise (previously known as TransferWise) have come
up with alternative solutions to international payments for small amounts that
everyday customers send often. Since most retail transactions are small (less
than \$5000 per day) Wise's internal system matches users attempting to send
small amounts in one currency block with corresponding users sending amounts in
the opposite currency block.  Wise uses these pools of funds to net out
aggregate transactions via local bank transfers.

The inability to move money from a country is ultimately one of domestic internal
infrastructure development and external international relations, not of
technical limitations. And the proposed use case for cryptocurrency as a mode of
international remittances is fundamentally limited in scale because of a lack of
a coherent compliance story. Even if we were to hypothetically use
cryptocurrency as a theoretical international settlement medium, this system
hasn't removed financial institutions from the equation. Both the entry and exit
points of this system would have to perform the same kind of checks of outgoings
and incoming money flows as required by an enormous number of international
agreements.

\index{remittances}

In this hypothetical scenario we've simply shifted the custodial, compliance, and
identity management responsibilities to a different centralized entity which
performs exactly the same activities and ultimately is subject to the same legal
liabilities. And insteady of settling in a currency pair, there is no two
currency pairs with a useless and voltile intermediary in between. Using
cryptocurrency for remittance hasn't disintermediated anything, it's just
shifted the intermediaries and introduced another level of indirection for no
reason.

A system that aimed to replace the existing international transfers would be
subject to the similar set of rules regarding international transfers and
capital controls and it is naive to think hundreds of treaties would be
renegotiated on behalf of digital currencies. Of course the counterargument is,
like all cryptocurrency arguments, an ideological one. Compliance is a non-issue
because nation states simply should not exist and should not have capital
controls. This is the ideological goal that is inexorably embedded in
cryptocurrency and which ultimately makes it unscalable and untenable technology
for any real-world application where sanctions, laws, and compliance are
inescapable part of doing business in financial services.

\index{anarchy}
