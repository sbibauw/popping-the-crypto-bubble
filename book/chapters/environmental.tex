\chapter{Environmental Shortcomings}

The technical inefficiencies of cryptocurrencies are the mark of a technology
which is over-extended and not fit for purpose. However, more concerning is the
environmental footprint these technologies introduce to the world. Bitcoin and
currencies that use the \textit{proof-of-work} method, require massive
consumption of energy to produce coins. This feature is central to their
operation and is the mechanism that allegedly ``builds trust'' in the network.
No network participant has any privileged status except in the amount of energy
they expend to maintain consistency of the network itself. The amount of energy
spent in the global block lottery results in an expected direct return per watt
which is statistically predictable. In a nutshell the premise of mining is to
prove how much power you can waste, and the more power you can waste, the more
resources you get for yourself. \cite{gerard2017attack}

The environmental impact of any technology has to be considered within a
framework its environmental shortcomings because the stakes of our era are so
high. Climate change is happening now, the earth is getting hotter and the polar
ice caps are melting. In Greenland the glaciers are receding six times faster
than expected.

\index{proof-of-work}
\section{Wasteful Mining}

The bitcoin network automatically adjusts the difficulty of mining so that each
block takes an average of 10 minutes to mine. If a miner performs a fixed number
$H$ of hashes every 10 minutes, this means that each hash has a $1/H$ chance of
mining a block. If a miner performs N hashes every minute, the number of blocks
they can expect to find per minute follows a binomial distribution with maximum
$N$ and with probability $1/H$. We can then compute the expected value of the
binomial distribution, calculated by multiplying the number of trials by the
probability of successes. The expected return is the expected value of computing
a block times the block reward amount (currently 6.25 bitcoins). This expected
return can then be computed as a function of a single variable: the number of
hashes a miner can compute per second.  This unit is dimensionalized in
hashes/second and often denoted in megahashes ($10^6$ or Mhash/s), gigahashes
($10^9$ Ghash/s) or terrahashes ($10^{12}$ Thash/s). At time of writing in order
for a data centre to consistently produce an expected return of 1 bitcoin per
day would require approximately 140,000 Thash/s.  These computations are
typically spread over many racks of compute devices running in parallel.

\index{bitcoin!energy consumption}

Off-the-shelf computers can be used for building cryptocurrency mining
rigs, however the performance of these devices is suboptimal compared to
dedicated mining hardware. A top of the line CPU such as the Intel i9
processor can perform 5.2 Mhash/s with an expected return of 0.00000001
bitcoin per week or at time of writing \$0.00012 per week. Mining using
standard CPUs found in most computers is generally infeasible. However a
single specialized \textit{graphics processing unit} (GPU) such as an Nvidia
GeForce RTX 2070 uses a model of execution with a great deal more
parallelism, and can have up to 42 Mhash/s and has an expected return of
0.0051 bitcoin per month or \$60.30 per month. One of these graphics
cards retails between \$500 and \$1000 per device and consumes 214 watts
of power.

\index{graphics processing unit}

As many people learned during the California gold rush, sometimes the most
profitable activity is often not mining itself but selling picks and shovels to
the miners. Globally there is a cottage industry of services and providers
selling in cryptocurrency mining wares and this global trade has drastically
driven up the price and lowered the supply of high end graphics processing
hardware. Cryptocurrency's volatility has led to the usual economics during
speculative bubbles. bitcoin has created a veritable arms race of mining
equipment which attempts to optimize hashes per watt. This created an enormous
network of data centers around the world all clustered around areas of cheap
power where the input capital per watt can yield an optimal return on
investment.  Regions such as Siberia in Russia, Texas in United States, and
Xinjiang in China have seen upticks in cryptocurrency mining activity due to
their climates proximity to cheap fossil fuel power.

\index{United States}
\index{Siberia}
\index{China!Xinjiang}

However the environmental impact of all this is simply that we are drawing more
power from the grid, burning more fossil fuels for the purpose of maintaining
this cryptocurrency network and lining the pockets of cryptocurrency miners. For
the bitcoin network which may have only 5\% of activity corresponding to
economic transactions \cite{vigna2019most} , this would result in a truly
staggering amount of economic and environmental waste if we compute the amount
of carbon emissions required to sustain this entire scheme.

Since the statistics involved in the proof of work system are readily
quantifiable it is possible to estimate the amount of energy required to
sustain the bitcoin network. de Vries et al. estimates that the bitcoin
network consumes 87.1 tWH (terawatt hours) of electrical energy annually
as of September 30, 2019. \cite{de2020bitcoin}

\begin{infobox}
 \textbf{ Bitcoin consumes nation-state levels of energy to process a minuscule
  amount of transactions considerably slower than other method.}
\end{infobox}

This amount of wasted energy on bitcoin is comparable to the energy consumption
of the entire nation of Argentina, a country of 50 million people. In his paper
de Vries also estimates the network has doubled its electricity consumption
between the 2018 and 2019 period. This figure represents 43\% or close to half
of the current entire global data centre electricity across the entire IT
industry as estimated by the International Energy Agency.
\cite{international2019key}

For a comparison with the traditional IT and banking sectors, the Visa
network processed 111.2 billion transactions in 2016. An internal audit
of the company reported its data centers 674,922 gigajoules, which
during the course of the year amounts to 21.4 megawatts ($10^6$ watts). All
of Google's data centers globally used an annual 5.7 terawatts ($10^{12}$
watts). This includes operations which provide Google search and video
streaming of YouTube to everyone on the internet.

The bitcoin network consumes more power than then all of the data centres of
Amazon, Microsoft, Facebook, Netflix, Google and Microsot put together.

\section{Environmental Horrors}

The per transaction costs of bitcoin is an even more alarming statistic.  The
transaction statistics from aggregated chain data indicates the bitcoin network
is performing 326,140 transactions per day or 119,041,100 transactions per year.
The per transaction energy cost is then 731.7 kWH, or equivalently the annual
power consumption of 2 American households. In comparison, the Visa network can
perform 100,000 transactions for 151 kWH and a single transaction takes 0.002
kWH.  \footnote{A time of writing the bitcoin network was performing 359,405
transactions per day.}

The yearly carbon output of this energy consumption is quantifiable as a
percentage of power derived from fossil fuel emissions, and the bitcoin
network is estimated to annually emit 51.9 megaton of carbon dioxide.
A single bitcoin transaction alone produces 270 kg of CO2.
\cite{goodkind2020cryptodamages, de2020bitcoin}

In addition to the carbon waste output, the network itself requires a constant
replacement of hardware and produces a constant stream of waste from broken and
exhausted components. \cite{DEVRIES2021105901} A substantial new change to the software protocol of a
cryptocurrency network may invalidate the previous purchases and require a
complete overhaul and repurchasing of all global mining hardware, especially
dedicated ASIC miners. The network produces 11,000,000 kilograms of electronic
waste annually or 96 grams of electronic waste per transaction. This is the
equivalent annual e-waste as several small countries and equivalent to 482,456
people living at German standard of 22.8 kg of e-waste per person/year.
Approximately 98\% of bitcoin mining equipment will become obsolete before
returning any value. \cite{peplow2019}

\index{ASIC}
\index{e-waste}
\index{Germany}

Bitcoin is a single network among hundreds which use similarly wasteful
proof-of-work models. It is difficult computation to estimate the global energy
cost and CO2 emissions across the entire cryptocurrency sector, but the total
sum of hundreds of proof-of-work currencies could conservatively be 50\% on top
of bitcoin energy requirements.  Gallersd{\"o}rfer et al.~estimates ``bitcoin
accounts for 2/3 of the total energy consumption, and understudied
cryptocurrencies represent the remaining 1/3. Therefore, understudied currencies
add nearly 50\% on top of bitcoin's energy hunger.'' The entire cost in terms of
health and climate damages caused by the continued operation of these services
is an alarming number and is deserving of further study and estimation.
\cite{gallersdorfer2020energy, de2019renewable}

\index{proof-of-work}

The question whether bitcoin has a legitimate claim on any of society's
resources is not one that has a scientific answer, it is fundamentally
an ethical question. There are many activities where humans burn massive
amounts of fossil fuels for activities which are for entertainment or
serve no economic purpose. For example, Americans burn 6.6 tWH annually
for holiday lightning. The question of whether we should sustain a
perpetually wasteful activity in perpetuity is a question the software
industry must ask.

The answers that have come outiside of the tech industry have raised some
alarming extrapolations from current trends.  In an environmental study Mora et
al.~estimate that ``Bitcoin emissions alone could push global warming above 2
°C'' \cite{mora2018bitcoin} and Goodkind et al.~suggests that
\cite{goodkind2020cryptodamages}:

\begin{quote}
Each \$1 of bitcoin value created was
responsible for \$0.49 in health and climate damages in the US and
\$0.37 in China
\end{quote}

Climate change is a runaway phenomenon that may pose an existential threat to
human civilization. Massive CO2 emissions today are a debit on the quality of
life for future generations. The problem of cryptocurrency mining is one that
needs to be addressed within a framework that takes into account the quality of
life for future generations, and in terms of a cost-benefits analysis of running
a network which is consuming nation state levels of power and whose purpose is
overwhelmingly speculative gambling.
