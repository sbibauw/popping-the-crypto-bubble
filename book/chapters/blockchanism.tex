\chapter{Blockchainism}

\index{blockchainism}

In the middle ages, a widely practiced proto-scientific tradition called
\textit{alchemy}, was concerned with finding ways to turn worthless metals into
gold. For centuries the most brilliant medieval scholars strived in their
laboratories to find a chemical process that would transmute so-called ``base
metals'' into ``noble metals,'' under an assumption that it was physically
possible but that it was inevitable they would discover it. Alchemy is one of
the first examples of the synthesis of economically motivating reasoning
bleeding over into scholarship and warping the epistemological foundations of
its practicers.

\index{alchemy}

Today \textit{blockchainism} is the new alchemy of the internet age; it is a
pseudoscience to transmute technology into trust. The ideology of
\textit{blockchainism} arises out of the proposition that blockchain technology
is a social or economical vehicle to enact change on social issues. This
phenomenon is primarily bracketed to the entrepreneurship and business
development contexts in which blockchain is the catalyst for new ventures and
corporate transformation projects. This phenomenon is perpetuated by the
over-exuberance of blockchain articles in trade journalism and a reasonably
continuous conference cycle centered around blockchainism.

\index{blockchainism}

\section{The Pseudoscience of Trust Alchemy}

The alchemy of blockchainism is a concept rooted in the mystique and
misunderstanding around the nature of bitcoin's original approach to
establishing trust between otherwise unrelated parties over an untrusted
network. Bitcoin has a partial answer to this problem in a particular way for a
specific data structure for a particular application. The core fallacy of
blockchainism is extrapolating that somehow cryptocurrency has solved trust in
generality. What ``solving trust'' means will depend on context, but this thesis
is central to many books, including \textit{Real Business of Blockchain},
\textit{Blockchain Revolution}, \textit{The Trust Machine}, \textit{The Infinite
Machine} and dozens of other popular business books.
\cite{jeffries_blockchain_2018}

Technology has always had a loose affiliation with futurism and science fiction.
Speculative fiction is the stories our civilization tells itself about where we
might end up and, ultimately, the ethical and philosophical problems we will
encounter on that journey. Science fiction, for many programmers, is an outlet
for philosophical discussion and a cultural touchstone that influences our work.
The history of science fiction literature has anticipated or inspired many of
the everyday technologies we use in our lives.

However, since the 80s, science fiction's tone has shifted toward a darker and
more bleak view of the future. The cyberpunk genre of the 80s often envisioned
dystopian futures in which institutions had broken down, leaving lawless
plutocracies. The combination of ``low life'' and ``high tech'' is the defining
attribute of this fiction. Often these genres concern the clashes between
authoritarian governments, powerful corporations, and hackers. The themes of
surveillance states, privacy, and internet politics are familiar storylines, and
in hindsight, much of the writing is quite prescient concerning the current
state of affairs. Many brilliant authors and speakers fall into this category,
and a diversity of perspectives and visions overwhelmingly enriches the field.
However, many of these visions are extraordinarily speculative, and the line
between speculation and reality is often one that becomes confused. We need to
draw a line between science fiction and scientific facts.

\index{cyberpunk}

The software industry is in a constant state of technical booms and busts.
Professor at Stanford Roy Amara once said of the software field that ``we
overestimate the impact of technology in the short-term and underestimate the
effect in the long run.'' In the past decade alone, we have seen trends rise and
fall of many buzzword trends that go by names such as artificial intelligence,
big data, internet of things, cloud computing, edge computing, and most
recently, blockchain. In most technologies, there is at least some kernel of a
novel idea that has the potential to alter the technology landscape and create
new lines of business. However, technology journalism tends to drastically
overstate the impact that any one of these ideas will have. Indeed most
entrepreneurs are strongly incentivized to enormously exaggerate the impact of
their ventures for both investors and the general public.
\cite{white_blockchain-based_2022}

\index{blockchain!as buzzword}

The rise of social media has also meant that many non-technical users have
stumbled on the surprising truth that they can significantly boost their
professional careers by becoming technical thought leaders with very little
understanding of technology or software. The rise of micro venture capital and
microfunds (funds with less than \$50 million assets in management) has given
rise to a particularly vocal set of associates in these funds whose job is
ostensibly to scream extremely loud, contrarian, and bombastic opinions about
technology trends. A large part of technology culture is a dialogue about
mundane ideas expressed through hyperbole and over-extrapolation.
\cite{baldwin_digital_2018, soatok_against_2021}

\index{venture capital}

This model presumes that the advent of cryptocurrency has given rise to
breakthrough technology and a global network in which arbitrary data can be
globally written, verified of integrity, and shared with relevant parties
without intermediaries. In this 'game-changing' paradigm shift, any existing
process that requires a single authoritative source of truth has now found the
ultimate vehicle for storing that single source of truth sans the authority
component. The blockchain (often referred to in singular form) will decentralize
power and disintermediate the global economy unlocking new opportunities and
building international reciprocity and trust. The seductive marketing around
this cliche is that absent cryptocurrency, the blockchain itself could convey
the same disruptive power as bitcoin for any domain. \cite{schneier_theres_2019}

The model is simultaneously a massive oversimplification and over-extrapolation
that stems from a fundamental misreading of the technology. At face value, it
has an illusory coherence and loose affiliation with the original intent and
ideology behind cryptocurrencies. Public blockchains are generally an awful form
of a database for most applications. The assumptions baked into their design are
entirely at odds with almost all business and government applications. For many
involved with blockchainism, the idea of a project is as far as their knowledge
and expertise will ever carry them. The feasibility details are simply an
implementation detail to be resolved by others. These details are often
addressed with a cliche ``blockchain not bitcoin'' or hand-wavy appeals to the
``the underlying blockchain technology is here to stay.''
\cite{walch_deconstructing_2019, soatok_against_2021}

This mentality is a partial path back to sensibility for most but confusingly
carries this blockchain semantic baggage. The problem most of the ventures will
quickly discover is that generally, most human processes that one would attempt
to digitize an implicit degree of control, governance, and privacy. For example,
companies operating in Europe are required under the General Data Protection
Regulation (GDPR) to allow users to erase personal personally identifiable
information when citizens requested by citizens. Blockchain is semantically
ambiguous, but the general ideas are multiparty networks that are decentralized,
trustless, and immutable. Suppose, hypothetically, an organization sets up a new
blockchain project where the servers are centrally controlled. In that case,
access is granted by a central authority, and data is revocable; then, one has
created the exact opposite technology and confusingly labeled it the same as its
polar opposite. This contradiction is the bizarre marketing-driving logic of
blockchain solutions looking for problems. \cite{babu_behind_2020,
halpin_deconstructing_2020}

\index{GDPR}

\section{The Blockchain Meme}


The form of technology that many of these ventures may build is not novel at
all; cryptographic ledgers and databases that maintain audit logs have been used
since the early 1980s. In practice, the term blockchain is often used to refer
to regular databases is a widely noted phenomenon. Rauchs et al. refer to this
phenomenon as the \textit{blockchain meme} and mention this phenomenon in a
whitepaper on this topic. \cite{rauchs_2nd_2019}

\begin{quote}
Unclear terminology and marketing hype have contributed to the
``blockchain meme'': 77\% of live enterprise blockchain networks have
little in common with multi-party consensus systems apart from
incorporating some of the same technology components (e.g.~cryptography,
peer-to-peer networking) and using similar nomenclature. The
``blockchain meme'' nevertheless acts as a powerful catalyst to overcome
corporate inertia and spur wide-reaching organisational change, both
within and across organisational boundaries.
\end{quote}

\index{blockchain!meme}

The blockchain meme has been found associated with all manner of outlandish
endeavors. In the United Kingdom, the government finance minister has publicly
suggested that blockchain may be a solution to Britain's Brexit problems
regarding trade along the Northern Ireland border. One of the most egregious
examples of blockchain opportunism is the introduction of California A.B. 2004.
This bill would create a pilot program for using ``verifiable health
credentials'' to report COVID-19 test results publicly. The bill was opposed by
both the ACLU and EFF for several core issues regarding privacy rights.

\index{United Kingdom}
\index{coronavirus}

Suppose we sort through trade journalism and press releases from 2018. In that
case, blockchain is proposed by many seemingly sensible people as the solution
to everything from human trafficking, refugee crises, blood diamonds, and
famines to global climate change. This is all even though most technologists
have minimal experience working with vulnerable groups or understanding the
political complexities. This thinking that blockchain somehow has the answers to
our problems has infected consultants, executives, and now even our politicians.
The one group of people who are not asked about the efficacy of blockchain is
programmers themselves, for whom the answer is simple: just use a normal
database.

\index{blockchain!supply chain applications}

Blockchainism is a siren-song of utopianism whose whispers are so soft that
every person will hear precisely that which they dream of. The charitable
interpretation of this phenomenon is that this simply an inefficiency in human
language that results from civilization collectively defining new terminology
and expanding understanding of technology. However the terminology itself lends
credibility to a domain which largely consists of gambling and financial frauds.
Blockchain consultants have become strange bedfellows with international
charities and non-government organisations advancing humanitarian causes.

Blockchainism is a siren song of aimless utopianism whose whispers are so soft
that every person will precisely hear what they dream of. A Rorschach test in
which everyone will see something different. The charitable interpretation of
this phenomenon is that this is simply an inefficiency in human language that
results from civilization collectively defining new terminology and expanding
its understanding of technology. However, the terminology itself lends
credibility to a domain that primarily consists of gambling, illicit financing,
and financial frauds. Blockchain consultants have become strange bedfellows with
international charities and non-government organizations advancing humanitarian
causes.

\index{blockchain!utopianism}

The reach of this philosophy is vast, and advocates of blockchainism are invited
to speak at the United Nations and the World Bank and attend meetings at the
World Economic Forum in Davos. On the back of this access, humanitarian groups
and NGOs have pumped millions of dollars into this sector for technologists to
look into the problem of applying The Blockchain to all manner of problems.
Humanitarian work is often thankless, and the ability to add ``innovative
solutions'' such as blockchain may generate publicity but ultimately comes at
the opportunity cost of doing high-impact but much less glamorous projects.
