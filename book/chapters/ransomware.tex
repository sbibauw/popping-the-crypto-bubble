\chapter{Ransomware}

\index{ransomware}

The majority of bitcoin use outside of speculation is not in payments, but
financial black market activities and malware. Malware is a form of software
which infects computer systems and directs them to behave in ways that are
contrary to user intention.  Viruses are a common form of malware which consist
of executable software which infects systems and instructs them to destroy or
deviate from normal running behavior. The virus will often spread itself to
other computers before executing its malicious payload. Another type of malware
is surveillanceware which operates by allowing a remote operator to observe the
private user behavior on a computer or smartphone and intercept private
communications and data on the device.

\section{The Cyberpandemic}

Malware is not a new phenomenon, and it has existed since the 90's and has seen
massive proliferation since the rise of widespread internet connectivity.
However, what is a new phenomenon is \textit{ransomware} which is a form of
malware which infects a target's computer, potentially encrypting or threatening
to delete their files in exchange for a ransom to be paid to the hackers. The
invention of bitcoin and other cryptocurrencies has allowed the feasibility of
this new form of malicious software to collect the funds of its victims around
the world.  While the software is not a new invention, if early ransomware
demanded a payment it would have had to either demand a small cash payment in
person potentially which would expose the hacker to tracking and law
enforcement. The alternative would be to require funds to be issued via bank
transfer, however this could be reversed by the institution and would reveal the
identity of the hacker's target account to receive the funds.

Cryptocurrency provided the perfect answer to allowing hackers to anonymously
prey on their victims and extort cash payments from them while minimising their
exposure of being caught by law enforcement. These attacks can be automated and
the ransomware will simply propagate itself and automatically funnel funds back
to the issuer with very little intervention on the hackers part. If there is a
killer app for cryptocurrency it is providing an efficient payment rail for the
criminal activity of ransomware. Ransomware could not function effectively
at scale without cryptocurrency.

\section{Crypto's Killer App}

Ransomware began using cryptocurrency in September 2013 with the spread
of an encrypting ransomware known as CryptoLocker. The software would
infect target computers and encrypt various local files and demand a
bitcoin with three days to unlock the captured files. The attack was
eventually isolated to a Russian network of computers and shut down. The
hackers in question were likely involved with this network but no
arrests were made.

The attack occurred in 2017 with the rise of a new payload known as WannaCry.
The payload was delivered and transmitted over the internal in a traditional
virus payload and infected nearly 230,000 Windows systems worldwide. The attack
demanded its victims pay \$300 per system paid in bitcoin to the hackers wallet
account. The attack hit many government and industrial systems including
Deutsche Bahn, FedEx, Russian Ministry and several health care systems. Notably
the United Kingdom NHS health care systems which infected some 70,000 devices
and effected 16 hospitals. The attack cost the British taxpayers £92 million to
recover and caused significant disruption to patient care and people's lives.
\cite{nytRansomware}

CryptoLocker was a form of ransomware that used a trojan to target Windows PCs
in 2013 and spread via infected email attachments would lock local and mounted
drives with public key cryptography. Bitcoin payments were required by a certain
deadline, otherwise the price would rise significantly. In total, the damage is
estimated to be in the range of \$30 million. Over 22 Texas municipalities were
hit by a ransomware attack in August 2019. The mitigation costs along with
recovery are estimated to be \$12 million. Although hospitals and businesses are
typically targets, more recently local governments have become the main focus
of these cybercriminals.

In late 2019 there was an attack on University of California San Francisco
research department performing COVID-19 vaccine development which locked servers
used by epidemiology and biostatistics departments. The university staff
attempted to negotiate with the attacker for over six days and came to the
ransom costs at \$1.14 million. COVID-19 has accelerated the ransomware business
targeting universities, labs and hospitals with vaccine research data. Such
ransomware groups tend to originate from Russia, China, and Eastern Europe and
possibly from state-level actors in these regions. \cite{xia2020don, mehrotra2020}

\index{coronavirus}

In 2019, Demant, a large manufacturer of hearing aids was hit with a ransomware
attack expected to cost up to \$95 million in losses.  The largest ransomware
attack known as NotPetya was responsible for a massive attack on global shipping
and commerce. The global shipping company Maersk and courier service FedEx were
eachhit each with a \$300 million loss due to the bitcoin ransomware attack.
The total effect was a devastating \$10 billion in losses. This also affected
pharmaceutical firm Merck and food production company Mondelēz Maersk shipping
holds responsibility for 76 ports and nearly 800 ships which carry cargo nearly
a fifth of the world shipping capacity.  NotPetya was a combination of two
leaked exploits, EternalBlue, an NSA tool and MimiKatz, made from a French
researcher and repurposed to a devastating capability to be able to infect
unpatched computers and use passwords to infect patched computers. One of the
most sinister aspects of NotPetya was that any payment to pay off the \$300
bitcoin ransom had no effect. All drives were encrypted and the key thrown away.
International cybersecurity researchers eventually uncovered that this was a
cyberattack meant for Ukraine from the Russian military. The effect on Ukraine
hit multiple hospitals, power companies, airports, and Ukrainian banks, payment
systems, and almost every federal agency. It effectively halted the Ukrainian
government from functioning at all.
\cite{mcquade2018untold}

\index{terrorism}

\section{The Oncoming Storm}

Traditional kidnapping and ransom based criminal enterprises have the unpleasant
requirement of physical presence. Physical cash forces the locality and
physicality of the criminal to the crime that ransomware frees them from. If a
criminal mugs you in the street and demands withdrawal of cash at an ATM, at
most they will get \$300 in a transaction that you can immedietely reverse with
a simple call to the bank.

With cryptocurrency enabling ransomers, it allows these criminals to proliferate
behind the scenes and with very little chance of getting caught.  The normal
financial system has innate measures to prevent this kind of indiscriminate
targeting of the innocent.

In 2021 the number of ransomware attacks has exploded.

The average ransom fee requested has increased from \$5,000 in 2018 to around \$200,000 in 2020.

The total ransomware costs are projected to exceed \$20 billion in 2021.

\begin{enumerate}
  \item January 2021 -- Travelex (\$2.3 million)
  \item April 2021 -- DC Police Department (\$4 million)
  \item May 2021 -- Brenntag (\$4.4 million)
  \item July 2021 -- CWT Global (\$4.5 million)
  \item May 2021 -- Colonial Pipeline (\$4.4 million)
  \item March 2021 -- Acer (\$5 million)
  \item May 2021 -- JBS USA Holdings (\$11 million)
  \item March 2021 -- CNA Financial (\$40 million)
\end{enumerate}

Ireland’s Health Service Executive was extorted for \$20 million, but the
government refused to play.

Thousands of unreported attacks that are payed out in secret through offshore
entities.

\index{Colonial Pipeline}

TODO: expand
