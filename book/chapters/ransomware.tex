\chapter{Ransomware}

\index{ransomware}

Most bitcoin use outside of speculation is not in payments but in financial
black market activities and malware. Malware is a form of software that infects
computer systems and directs them to behave in ways that are contrary to user
intention. Viruses are a common form of malware that consist of executable
software which infects systems and instructs them to destroy or deviate from
normal running behavior. The virus will often spread to other computers before
executing its malicious payload. Another type of malware is surveillanceware,
which allows a remote operator to observe the private user behavior on a
computer or smartphone and intercept private communications and data on the
device.

\section{The Cyberpandemic}

Malware is not a new phenomenon; it has existed since the 90s and has seen
massive proliferation since the rise of widespread internet connectivity.
However, its most novel and destructive form is \textit{ransomware}, a form of
malware that infects a target's computer, potentially encrypting or threatening
to delete their files in exchange for a ransom to the hackers. The invention of
bitcoin and other cryptocurrencies has allowed the feasibility of this new form
of malicious software to collect the funds of its victims around the world.
While the software is not a new invention, if early ransomware demanded a
payment, it would have had to demand small physical cash payments in person or
by mail, potentially exposing the hacker to tracking and law enforcement. The
alternative would be to require funds to be issued via bank transfer; however,
the institution could reverse this transaction and reveal the identity of the
hacker's target account to receive the funds.

\index{ransomware}

Cryptocurrency provided the perfect answer to allowing hackers to anonymously
prey on their victims and extort cash payments from them while minimizing their
exposure to being caught by law enforcement. Hackers can automate these mass
attacks, and the ransomware will virally propagate itself and automatically
funnel funds back to the issuer with very little intervention on the hackers'
part. If there is one killer app for cryptocurrency, it is that it provides an
efficient payment mechanism for extortion. Ransomware could not function
effectively at scale without cryptocurrency.

\section{Crypto's Killer App}

The first example of cryptocurrency-based ransomware was in September 2013 with
the discovery of a ransomware payload known as CryptoLocker. The software would
infect target computers, encrypt various local files, and demand a bitcoin
payment to unlock the captured files within three days. The attack was
eventually isolated to a Russian network of computers and shut down. The hackers
in question were likely involved with this network; however, there were no
arrests.

The attack occurred in 2017 with the rise of a new payload known as WannaCry.
The payload was delivered and transmitted over the internal in a traditional
virus payload and infected nearly 230,000 Windows systems worldwide. The attack
demanded its victims pay \$300 per system paid in bitcoin to the hackers' wallet
account. The attack hit many governments and industrial systems, including
Deutsche Bahn, FedEx, the Russian Ministry, and several public health care
systems. Most notably, the United Kingdom's National Health Care systems saw
some 70,000 devices infected, affecting 16 hospitals. The attack cost the
British taxpayers £92 million to recover and caused significant disruption to
patient care and people's lives. \cite{popper_ransomware_2020}

CryptoLocker was ransomware that used a trojan to target Windows PCs in 2013 and
spread via infected email attachments to lock and encrypt local storage devices.
Bitcoin payments were required by a specific deadline; otherwise, the price
would rise significantly. The damage is estimated to be in the range of \$30
million. A ransomware attack hit over 22 Texas municipalities in August 2019.
The mitigation costs and recovery are estimated to be \$12 million. Although
hospitals and businesses are typically targets, local governments have become
the main focus of these cyber criminals more recently.

In late 2019 there was an attack on the University of California San Francisco
research department performing COVID-19 vaccine development, which locked
servers used by epidemiology and biostatistics departments. The university staff
attempted to negotiate with the attacker for over six days and came to the
ransom costs of \$1.14 million. COVID-19 has accelerated the ransomware business
targeting universities, labs, and hospitals with vaccine research data. Such
ransomware groups originate from Russia, China, and Eastern Europe and possibly
from state-level actors in these regions. \cite{xia_dont_2020, mehrotra_ucsf_2020}

\index{coronavirus}

Not all attacks result in payouts to Hackers. In 2021 Ireland's Health Service
was extorted for \$20 million, but the government refused to pay.
\index{terrorism}

\section{The Oncoming Storm}

Traditional kidnapping and ransom-based criminal enterprises have the unpleasant
requirement of physical presence. Physical cash forces the locality and
physicality of the criminal to the crime, whereas ransomware frees them from
this locality restriction. If a criminal mugs a victim in the street and demands
withdrawal of cash at an ATM, they will at most get the maximum ATM withdrawal
amount (often \$300) transaction that the victim can immediately reverse with a
simple call to the bank.

With cryptocurrency enabling ransomers, it allows these criminals to proliferate
behind the scenes with very little chance of getting caught. The regular
financial system has innate measures to prevent this indiscriminate targeting of
the innocent.

In 2021 the number of ransomware attacks exploded in number. The average ransom
fee requested has increased from \$5,000 in 2018 to around \$200,000 in 2020.

The total ransomware costs are projected to exceed \$20 billion in 2021. As of
2021, the most lucrative hacks have been the following attacks:

\begin{enumerate}
  \item January 2021 -- Travelex (\$2.3 million)
  \item April 2021 -- DC Police Department (\$4 million)
  \item May 2021 -- Brenntag (\$4.4 million)
  \item July 2021 -- CWT Global (\$4.5 million)
  \item May 2021 -- Colonial Pipeline (\$4.4 million)
  \item March 2021 -- Acer (\$5 million)
  \item May 2021 -- JBS USA Holdings (\$11 million)
  \item March 2021 -- CNA Financial (\$40 million)
\end{enumerate}

\index{Colonial Pipeline}
