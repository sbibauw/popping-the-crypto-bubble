\chapter{Introduction}

Cryptocurrency is all a giant scam, albeit a complicated scam that uses
technobabble, heterodox economics and populist anger to obfuscate its
functioning. It is a pitch perfect scam for the post-truth era of social media
where trust in institutions and experts is at an all-time low.

The overarching idea of cryptocurrency is based on a complex set of myth making
yet built around a simple unifying idea: to reinvent money from first principles
independent of existing power structures. It has emerged as one of the most
intriguing and destructive socio-technical phenomena of the early 21st century.
Rarely in the history of technology has there been any invention that has been
the nexus of discussion along many dimensions of human experience and divides
technologists on the ethical implications of its existence. Cryptocurrency is a
phenomenon that captured the zeitgeist in economics, technology, politics, law,
ethics, culture, and monetary policy.

Cryptocurrencies are undeniably an economic bubble, but it is a bubble on a
scale that we have yet to observe before in human history. Nevertheless, the
Nobel laureate Robert Shiller gives us a framework for understanding this
phenomenon in his work Narrative Economics \cite{shiller_narrative_2017}, which
describes the epidemiology of bubbles that perpetuate "stories that motivate and
connect activities to deeply felt values and needs". Cryptocurrency is primarily
driven by the twenty-first century narrative of distrust in institutions, yet
the undeniable need for the continuity of the financial services traditionally
provided by these financial institutions.

All software is political, but some software is more political than others.
Cryptocurrency can best be understood as a reaction to 2008 the global financial
crisis and the synthesis of technolibertarianism ideology with growing distrust
in financial institutions. Within this context, this technology has influenced
the financial services sector and had a profound influence on popular economic
discourse. It is a technology that remains extremely divisive in its approach to
disruption and its political ends. The divisions on cryptocurrency divide
individuals based on a philosophical question: Does one worry more about the
abuse of centralized power, or does one worry more about anarchy?

To those that fear anarchy, cryptocurrency is the fantasy of a financial system
free of democratic oversight whose efficacy is based purely on technology and
free-market forces. To those that fear centralized power, cryptocurrency is a
hedge against the overreach and corruption of big government and a financial
services sector whose greed and excesses were made clear during the financial
crisis. This philosophical divide falls loosely along with existing political
ideologies; the cryptocurrency school is often ideologically adjacent to
existing right-wing economic schools of thought, it is, however, not entirely
contained within that political sphere. It is a phenomenon that many individuals
have co-opted for many ends as a result of its general misunderstanding.

In this critique, we will survey the state of cryptocurrency, its political
imaginaries, and the outcomes it has achieved in the last decade.  In both the
``fake it til' you make it'' culture of Silicon Valley and libertarian
ideologies, it is widely accepted to ignore externalities of technology until a
critical mass is reached, and the benefits can be realized in hindsight.

Nevertheless, cryptocurrency in the present moment has profound technical,
privacy, economic, ethical, and environmental shortcomings that need to be
discussed. The constellation of these shortcomings, lack of innovation, and
potential for vast public harm form the core case against cryptocurrency.

\index{anarchy}
\index{narrative economics}
\index{Silicon Valley}
\index{financial crisis}
\index{techno-obscurantism}
\index{Shiller, Robert}
