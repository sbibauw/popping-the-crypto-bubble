\chapter{History of Cryptocurrency}

\begin{quote}
The concentration of wealth is natural and inevitable, and is periodically
alleviated by violent or peaceable partial redistribution. In this view all
economic history is the slow heartbeat of the social organism, a vast systole
and diastole of concentrating wealth and compulsive recirculation.
\begin{flushright}
  \textit{The Lessons of History} --- Will and Ariel Durant
\end{flushright}
\end{quote}

\index{bitcoin!history of}

The introduction of bitcoin as a technology is undoubtedly a limited innovation.
Even if the technology is limited in its applications and scale, it presents a
novel approach to classical computer science problems that spawned a new
approach to digitizing financial products. However, for most of the general
public, the perception of cryptocurrency is often at the level of:

\begin{quote}
Bitcoins have something to do with computers, are very expensive, and are going up.
\end{quote}

The technology behind cryptocurrency is the product of many people and the
fusion of different advances from the last twenty years. Nevertheless, its
history is described by a single overarching truth:

\begin{infobox}
 \textbf{
    Cryptocurrencies were intended as a peer-to-peer medium of payment but have
    since morphed into a product whose purpose is almost exclusively as a
    speculative investment.
  }
\end{infobox}

\section{Cypherpunk Era}

Despite its most ardent acolytes' claim, bitcoin was not an artifact of divine
revelation. Like most technology, it does not exist in a vacuum and was the
product of a long sequence of trends going back to the early days of the
internet. Bitcoin was not even the first digital currency and was preceded by
multiple attempts along the same idea dating back to the early 80s. The
provenance of bitcoin is best understood through the lens of understanding the
various internet subcultures that gave rise to the political ideologies and
component technologies behind it.

The first hints of this idea go back to David Chaum's cryptography paper
\textit{Blind Signatures for Untraceable Payments} which outlines a theoretical
basis for a system of making an electronic payment system using digital
signature algorithms. [@chaum1984blind] The paper presents that idea as a
mathematical formulation and does not provide an implementation. Later in 1989,
Chaum would start a company that attempted to take these ideas into production
in the financial services sector. The company sold three years later, and its
technology was eventually folded into an open-source project called GNU Taler.

At the same time, the United States was coming out of the Cold War, and the
United States Department of Commerce and the State Department were increasingly
concerned about the geopolitical implications of allowing exports of strong
encryption standards to the world. At this time, Phil Zimmerman, the inventor of
a common encryption standard known as PGP, challenged the United States munition
controls on encryption in what would be known as the first Crypto Wars. The
legal precedent, in this case, had massive implications for the early
development of internet browsers and internet technology which were starting to
rely on strong encryption standards to enable secure communication and
eventually e-commerce. This trend created an underground movement on the early
internet known as cypherpunks, who were technological activists who advocated
for the unrestricted use of cryptography and privacy-enhancing technologies as a
vehicle for social and political change.

\index{United States}
\index{cypherpunk}

The first digital currency was launched in 2006 in the United States under a
company named eGold. The company allowed early internet users to purchase
fractional ownership amounts in offshore physical gold holdings and used this
centralized register to make instantaneous transfers to other eGold customers.
Businesses like MoneyGram and Western Union had been operating e-money services
for many years; however, eGold differed in that it was not denominated or backed
by a national currency. This service operated under what the law defines as a
money transmitter business that facilitated payments between third parties by
creating a central register that recorded credits and debits. The business
quickly rose to prominence during the early internet eGold was prosecuted by the
United States federal government under the Patriot Act and was found to violate
existing money transmitter laws. In 2007 the enterprise was shut down, and the
federal government seized its assets. \cite{popper_untold_2015}

\index{money transmitter}
\index{mining}
\index{eGold}
\index{e-money}
\index{gold}
\index{Moneygram}
\index{Western Union}

From 2006 to 2013 company called \textit{Liberty Reserve S.A.} operated out of
Costa Rica and offered an offshore anonymous money transmission
service. The service allowed users to deposit money into a virtual dollar
account via wire transfer or credit. Customers could then transfer these funds
to other Liberty Reserve account holders without validation of the identities of
the account holders or any legal restrictions. The FBI raided the offices, and
the company was shut down in 2013 for violating the United States money
laundering laws.

\index{FBI}
\index{Patriot Act}
\index{Liberty Reserve}

In the 21st century, most money is digital and is represented as numerical
values in databases holding balance sheets for bank deposits. The auditing and
accounting of this money is a regulated part of obtaining banking licenses, and
this process of digitization of products and digital straight-through processing
has been proceeding since the 1980s. To most consumers, this is transparent;
however, in the early 2000s, most consumers became aware of the digitization of
their money in the form of increasing online banking. These now-common services
gave customers a real-time view of their balances and transactions and
increasingly allowed consumers to issue and receive payments. However, in the
early days of e-commerce, there was still apprehension around receiving and
making payments over the internet with credit cards. To fill this gap, PayPal
emerged as a service to support online money transfers, which allowed consumers
and businesses to transact with a single entity that would process and transmit
payments between buyers and sellers without the need for direct bank-to-bank
transfers. This was a particularly well-timed business that capitalized on the
rise of online shopping services such as eBay and Amazon.

\index{PayPal}

Contemporaneous to the e-money digital transformation was the digital
file-sharing scene, which in 1993 grew out of early file-sharing systems on
Usenet and found its way into the mainstream with the development of  Napster.
Napster was a global peer-to-peer file-sharing network that allowed users to
share the newly invented MP3 files with other users without paying for the
original recordings of the music. This service was eventually shut down for
copyright infringement but spawned an entire generation of new open-source
protocols such as Gnutella, Freenet, and BearShare. The most successful of
these, BitTorrent, proved more challenging to shut down because of the lack of a
single entity to target. The BitTorrent protocol was based on a data structure
known as a Merkle tree which allowed large files to split up into individual
pieces, transmitted over a network, and reconstructed in parts while maintaining
the integrity of the entire data file. This core data structure would be
instrumental in advancing peer-to-peer networks to share hashes of incomplete
data while maintaining integrity through cryptography.

\index{e-money}
\index{Usenet}
\index{Merkle tree}
\index{Napster}
\index{BitTorrent}

On October 31st, 2008, a person or group of people under the pseudonym Satoshi
Nakamoto published \cite{nakamoto_bitcoin_2008} a paper outlining a new system on a
cryptography mailing list. The nine-page essay outlines what is described as a
peer to peer electronic cash that could operate independently of central
authorities.

\index{Nakamoto, Satoshi}

\begin{quote}
A purely peer-to-peer version of electronic cash would allow online
payments to be sent directly from one party to another without going
through a financial institution. We propose a solution to the
double-spending problem using a peer-to-peer network.
\begin{flushright}
-- Bitcoin Whitepaper
\end{flushright}
\end{quote}

Computer science had previously grappled with what was known as the
\textit{double-spend problem} since the first networked computers and databases
were invented. In short, this problem is concerned with ensuring that a digital
representation of value is not copied and consumed by multiple sources that
require it. The digital banking and credit card industries had dealt with a
similar set of problems in which digital ledgers recording payments would need
to be reconciled or communicated to multiple parties to ensure that digital
representations of units of value were kept in sync with deposits. The existing
solution to this problem had always been centralized trust authorities which
held a legal obligation to maintain authoritative digital records and would
handle discrepancies by introducing time delays for compliance checks and manual
mechanisms to handle disputes and inconsistencies.

\index{double spend problem}

The mechanism described by bitcoin proposed a novel solution for the
double-spend problem, which did not require a central trust authority. Instead,
it relied on a class of computer program known as a \textit{consensus algorithm}
and clever use of a digital signature scheme to maintain a consistent record of
a ledger database across multiple computers without needing an authoritative
central source of trust. A digital signature is a technique in cryptography
where a user uses a set of cryptographic keys (a public and a private key) to
sign a piece of data to verify the integrity and origin of the data. The public
key can be shared with the world and authenticated by associating it with
another identity. The private key is kept secret and is used to generate digital
signatures, which only the user who possesses the private key would be able to
generate. This technique was extensively studied and widely used in existing
internet infrastructure such as SSL, which protects online banking and other
secure services.

\index{cryptography}

A consensus algorithm is a class of computer program in which multiple computers
use a set of steps to write a shared set of data (such as a database) so that
all users can access a consistent view of the data from all computers. Consensus
algorithms are a branch of a field of computer science called distributed
systems which concerns itself with different approaches to building consensus
algorithms and different techniques of sequencing read and writes to shared data
stores. All consensus algorithms are constrained by a fundamental result known
as the CAP theorem, which states that any algorithm may have at most two out of
three properties: Consistency, Availability, and Partition Tolerance.

\index{consensus algorithm}

\begin{itemize}
\tightlist
\item
  \textit{Consistency} - A readers of data from the service see the same data at
any given point of time
\item
  \textit{Availability} - All readers can retrieve the data
you stored in a system even if a subset of the service goes down.
\item
  \textit{Partition Tolerance} - No set of failures less than total network failure
result in the system responding to queries incorrectly.
\end{itemize}

The bitcoin network chooses the availability and partition tolerance properties
of this triplet.

The bitcoin algorithm took a particularly interesting approach to consensus in
that it attempted to create a censorship-resistant network in which no
participant was privileged. The consensus process was eventually consistent and
tied the addition of writes to the solution of a computational problem in which
computers that participated in the consensus algorithm would need to spend a
given amount of computational work to attempt to confirm the writes. This
approach, known as \textit{proof of work} created what is known as a random
sortition operation in which a network participant would be selected randomly
and probabilistically based on how much computational power (called
\textit{hashrate}) was performed to attempt consensus.

\index{proof of work}
\index{hashrate}

This architecture created a computational game mechanic in which computers in
this network (called \textit{miners}) competed to perform consensus actions and
successfully confirming a block of transactions gave a fixed reward to the first
"player" to commit a set of transactions. The rules of this game were defined by
the shared software that all participants ran, which defined the network
protocol. The critical ideas encoded in the protocol are the predetermined
release schedule, fixed supply, and support for protocol changes with support
from a majority of participants.

\index{miner}

One of the core algorithms used in most blockchains is a hash function. A hash
function is a classic cryptography algorithm in which data is repeatedly
scrambled in a process that is difficult to reverse and produces a unique
fingerprint of the data unique to the given input. The output of this function
is an inordinately large number which is often encoded in the hexadecimal
(base-16 number system).

\index{blockchain!data structure}

\begin{verbatim}
Hello World
d2a84f4b8b650937ec8f73cd8be2c74add5a911ba64df27458ed8229da804a26
\end{verbatim}

Minor changes to the input of a hash function alter the output. For instance,
the swapping of the case of the ``w'' in the statement changes the output
drastically.

\begin{verbatim}
Hello world
1894a19c85ba153acbf743ac4e43fc004c891604b26f8c69e1e83ea2afc7c48f
\end{verbatim}

\index{cryptography!hash function}

The hash function's output follows some well-studied statistical distributions,
and the probability of an output prefixed by a fixed number of zeros can be
predicted to occur within a certain amount of hashes. This process allows the
difficulty mechanism of obtaining a particular output to be scaled and gives
rise to the adjustable puzzle that miners are forced to solve to perform block
confirmation. This mechanism allows the difficulty of bitcoin mining to be
artificially adjusted proportionally to the rewards.

This consensus algorithm's underlying data kept in sync was a specific
constructed data structure known as a \textit{blockchain}. The blockchain is a
ledger data structure that holds an authoritative record of all proposed spend
activities of a digital numerical unit of value. A bitcoin is a decimal value
whose spend activities were enforced by the consensus algorithms and could be
transmitted to other accounts, recorded on the blockchain, and mapped to
addresses corresponding to public cryptographic keys. This design created a
distributed digital ledger that recorded debits which could be continuously
updated over time as the bitcoins were created and spent.

\index{blockchain}
\index{distributed ledger}
\index{blockchain!accounts}
\index{blockchain!definition of}

The censorship resistance of this algorithm was the critical improvement over
existing ecash systems which previously had a single legal point of failure, in
that the central register or central node would have to be stored in a single
server that could be targeted by governments and law enforcement. In this
trustless peer-to-peer (P2P) model, all computers participated in the network,
and removing any one node would not degrade the availability of the whole
network. Just as previous P2P networks had routed around intellectual property
laws, bitcoin routed around money transmitter laws.

On January 3rd, 2009, the first block\footnote{This is referred to as the
genesis block.} in the bitcoin blockchain was created and contained a simple
message in the Times of London concerning the bank bailouts during the subprime
mortgage crisis:

\begin{quote}
\begin{verbatim}
The Times
03/Jan/2009
Chancellor on brink of second bailout for banks
\end{verbatim}
\end{quote}

The early history of bitcoin saw the technology was primarily used as Nakamoto
intended, as an anonymous global digital payment network. Early-adopter
technologists used Bitcoin as an anonymous way to pay for goods and services and
as a tip system for authors of online content. One of the first notable
exchanges was for two pizzas in exchange for 10,000 bitcoins on March 22nd,
2010. The first evidence of a price bubble was in June 2011, when the exchange
value for one bitcoin moved from \$30 down to \$2 in December 2011 after a
prominent bitcoin exchange website hack. In 2013 the technology began to receive
mainstream attention from the press, and this era represents the philosophical
transition of the technology from use as a hypothetical digital currency into an
asset for investment. The rapid price movement was then uncorrelated with
traditional assets and proved an attractive investment for a class of traders
looking for new speculative opportunities. After this point, the single defining
feature of bitcoin no longer became its utility for anonymous payments but
solely for its price action as a speculative investment.

The original author (or authors) of the technology withdrew from participation
in the network and any communication, and other software developers and
companies took up development after that. This era marks a rapid expansion of a
cottage industry of startups and early adopters who would build exchanges,
mining equipment, and a marketing network to proselytize the virtues of this new
technology. The culture around the extreme volatility of the asset created a
series of memes within the subculture of HODL (a portmanteau of the term
"holding," standing for "hold on for dear life"), which encourages investors to
hold the asset regardless of price movement. This investment philosophy became
central to the cryptocurrency sector and was a statement of faith in the asset
class. The implicit promise of bitcoin and any cryptocurrency in this era is
that if of easy money for nothing, the idea that if one invests early, one can
get rich when the value "goes to the moon." Effectively a digital form of the
classic get-rich-quick scheme for the internet era. This marked the start of a
new era in crypto.

\section{Grifter Era}

\index{HODL}

In addition to bitcoin, a series of similar technologies based on the same ideas
emerged in the 2011-2013 era. The first movers were Litecoin, Namecoin,
Peercoin, and a parody token known as Dogecoin based on an internet meme. These
projects (called \textit{altcoins}) built on the bitcoin model but tweaked the
implementation of the protocol to allow for different network behaviors and
incorporated different economic models. As of August 2018, the number of
launched cryptocurrency projects exceeded 1600. In 2015 a significant extension
to the bitcoin model was launched as the ethereum blockchain, a project which
aimed to build a "world computer" in which programmable logic could be expressed
on the blockchain instead of only simple asset transfers. This project would
become the second most traded token and would popularize the notion of smart
contracts. In addition to fully visible transaction models of previous tokens,
chains such as Monero and ZCash would incorporate privacy-enhancing features
into the design, allowing participants to have blinded transactions that would
obscure the endpoint details for illicit transactions with no public audit
record.

\index{altcoin}
\index{Monero}
\index{Litecoin}
\index{ZCash}
\index{Dogecoin}
\index{ethereum}

Unlike in the original bitcoin paper, the idea of "one computer, one vote" was
supplanted by the economic reality that large groups of computers could more
efficiently compute the hashes required to confirm blocks and thus reap
financial rewards from the network. Early entrepreneurs realized that they could
gain an advantage over traditional server farms if they built faster and more
specialized hardware to compute these hashes. These entrepreneurs began to build
ASIC (Application Specific Integrated Circuit), which were custom hardware
circuits that could do the computations required for the bitcoin network more
efficiently than traditional CPUs offered by companies like Intel and AMD. This
economic circumstance led to a technical arms race in which dedicated hardware
became required to mine bitcoin, and larger groups would build clusters of
computers that would pool the rewards acquired by mining together. These
\textit{mining pools} as they would become called, became a centralized and very
lucrative business for early investors. In particular, the Chinese company
BitMain and others began to centralize most of the computational resources, and
by 2019 70\% of all bitcoin mining was concentrated in mainland China.

\index{mining pool}
\index{China}
\index{ASIC}

The underutilization of coal-fired power production and Chinese capital
restrictions on renminbi outflows offered a unique opportunity for enterprising
Chinese citizens to move capital outside of the mainland beyond government
controls. In 2018 the Chinese government officially declared cryptocurrency
mining an undesirable activity. Bloomberg also reported that year that \$50
billion of capital flight occurred from the Chinese state using the Tether
cryptocurrency as the mode of flight. \cite{leising_crypto_2020}

\index{China!renminbi}
\index{China!capital flight}

In 2017 a new trend emerged when market participants realized that secondary
tokens could be launched on top of the ethereum blockchain. This trend gave rise
to a controversial new market for ICO (Initial Coin Offerings), which were
offerings of digital tokens to fund project development. In September 2017, an
ICO named Kik raised \$100 million within several days. Many other ventures
would raise unconventionally high amounts of money for early-stage ventures over
the 2017-2018 time period. Between January to June 2018, over \$7 billion was
raised to fund ICO projects. Most of these projects failed within a very short
period and exhausted their funds to pursue business models that would prove
intractable and economically unviable. Many of these businesses turned out to be
fraudulent or thinly veiled \textit{exit scams} in which the entrepreneurs simply
abscond with the crypto assets raised and never build the product claimed in the
prospectus. These cases are still being litigated to this day.

\index{ICO}
\index{exit scam}

This period also saw the introduction of stablecoins such as Tether, aiming to
be a stable crypto asset whose price was allegedly pegged to the United States
dollar and theoretically backed by a reserve of other assets. After the 2019
era, there would be a period of market volatility and market consolidation of
cryptocurrencies in which many unfounded ideas would fall off and leave a
handful of 20 projects which would dominate trading volume and developer
mindshare.

\index{Tether}

In 2021 China outright banned all domestic banks and payment companies from
touching crypto assets and banned all mining pools in the country. At the same
time, the United States continued to be hit by an onslaught of cyberterrorism
and ransomware attacks that began to attack core national infrastructure and the
country's energy grids.

\index{China!mining ban}
\index{ransomware}
