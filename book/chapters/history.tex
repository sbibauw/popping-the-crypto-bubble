\chapter{History of Cryptocurrency}

\index{bitcoin!history of}

The introduction of bitcoin as a technology is undoubtedly a limited innovation.
Even if the technology is limited in its applications and scale, it presents a
novel approach to classical computer science problems that spawned a new
approach to digitisation of financial products. However for most of the general
public the perception of cryptocurrency is often at the level of:

\begin{quote}
Bitcoins have something to do with computers, are very expensive and
are going up.
\end{quote}

The technology behind cryptocurrency is the product of many people and the
fusion of different advances from the last twenty years of development of
the internet. This history is described by a single truth:

\begin{infobox}
 \textbf{Bitcoin was intended as a peer-to-peer medium of payment but has since
  changed into a product whose purpose is almost exclusively as a speculative
  investment.}
\end{infobox}

Bitcoin was not an artifact of divine revelation. Like most
technology it does not exist in a vacuum and was the product of a long
sequence of events and trends going back to the early days of the
internet. Bitcoin was not even the first digital currency and was
preceded by multiple attempts along the same idea dating back to the
early 80s. The provenance of bitcoin is best understood through the lens
of understanding the various internet subcultures that gave rise to the
political ideologies and component technologies behind it.

The first hints of this idea go back to David Chaum's cryptography paper
\textit{Blind Signatures for Untraceable Payments} which outlines a theoretical
basis for a system of making an electronic payment system using digital
signature algorithms. \cite{chaum1984blind} The paper purely presents that idea
as a mathematical formulation and does not provide an implementation.  Later in
1989 Chaum would start a company which attempted to take these ideas into
production in the financial services sector. The company sold three years later
and it's technology was eventually folded into an open source projects.

At the same time the United States was coming out of the Cold War and
the US Department of Commerce and the State Department were increasingly
concerned about geopolitical implications of allowing exports of strong
encryption standards to the world. At this time the Phil Zimmerman, the
inventor of a common encryption standard known as PGP challenged the US
munition controls on encryption in what would be known as the first
Crypto Wars. The legal precedent on this case had massive implications
for the early development of internet browsers and internet technology
which were starting to rely on strong encryption standards to enable
secure communication and eventually e-commerce. This trend created an
underground movement on the early internet known as Cypherpunks (a pun
on the Cyberpunk science-fiction genre), who were technologists who
advocated for free use of cryptography and privacy-enhancing
technologies as a vehicle for social and political change.

\index{United States}
\index{cypherpunk}

The first digital currency was launched in 2006 in the United States under the
company named eGold. The company allowed early internet users to purchase
fractional amounts of ownership in offshore physical gold holdings and used this
centralised register to make instantaneous transfers to other eGold customers.
Businesses like MoneyGram and Western Union had been operating e-money services
for many years, however eGold differed in that it was not denominated or backed
by a national currency.  This service operated under what the law defines as a
money transmitter business that facilitated payments between third parties by
the creation of a central register that recorded credits and debits. The
business quickly rose to prominence during the early internet eGold was
prosecuted by the US federal government under the Patriot Act and was found to
be in breach of existing money transmitter laws. In 2007 the company was shut
down and its assets were seized by the federal government.
\cite{popper2015untold}

\index{money transmitter}
\index{mining}
\index{eGold}
\index{e-money}
\index{gold}
\index{Moneygram}
\index{Western Union}

During 2006-2013 company called \textit{Liberty Reserve S.A.} operated out of
the island of Costa Rica and offered an offshore anonymous money transmission
service. The service allowed users to deposit money into an virtual dollar
account via wire transfer or credit. Customers could then transfer these funds
to other Liberty Reserve account holders without validation of the identities of
the account holders or any legal restrictions. This allowed anonymous
international payments and money laundering, the offices for the company were
raided by the FBI and the site was shut down in 2013.

\index{FBI}
\index{Patriot Act}
\index{Liberty Reserve}

In the 21st century most money is digital, most money is represented as values
in databases holding balance sheets. The auditing and accounting of this money
is a regulated part of obtaining banking licenses and this process of
digitisation of products and digital straight-through processing has been
proceeding since the 1980s. To most consumers this is transparent, however in
the early 2000s most consumers became aware of the digitisation of their money
in the form of increasing online banking. These now-common services gave
customers a real-time view of their balances and transactions and increasingly
allowed consumers to issue and receive payments. However in the early days of
ecommerce there was still apprehension around receiving and making payments over
the internet with credit cards. To fill this gap PayPal emerged as a service to
support online money transfers which allowed consumers and businesses to
transact with a single entity who would process and transmit payments between
buyers and sellers without the need for direct bank-to-bank transfers. This was
particularly well timed business that capitalised on the rise of online shopping
services such as eBay and Amazon.

\index{PayPal}

Contemporaneous to the e-money digital transformation was the digital file
sharing scene, which in 1993 grew out of early file sharing systems on Usenet
and found its way into the mainstream with the development of Napster service.
Napster was a global peer-to-peer file sharing network which allowed users to
share the newly invented MP3 files with other users without paying for the
original recordings of the music. This service was eventually shut down for
copyright infringement but spawned an entire generation of new open source
protocols such as Gnutella, Freenet, BearShare. The most successful of these,
BitTorrent, proved more difficult to shut down because of the lack of a single
entity to target.  The BitTorrent protocol was based on a data structure known
as a Merkle tree which allowed large files to split up into individual pieces,
transmitted over a network and reconstructed in parts while maintaining the
integrity of the entire data file. This core data structure would be
instrumental in advancing peer to peer networks to share hashes of incomplete
data while maintaining integrity through cryptography.

\index{e-money}
\index{Usenet}
\index{Merkle tree}
\index{Napster}
\index{BitTorrent}

On October 31, 2008 a person or group of people under the pseudonym
Satoshi Nakamoto published\cite{nakamoto2019bitcoin} a paper outlining a new system on a
cryptography mailing list. The nine-page essay outlines what is
described as a peer to peer electronic cash that could operate
independent of central authorities.

\index{Nakamoto, Satoshi}

\begin{quote}
A purely peer-to-peer version of electronic cash would allow online
payments to be sent directly from one party to another without going
through a financial institution. We propose a solution to the
double-spending problem using a peer-to-peer network.
\begin{flushright}
-- Bitcoin Whitepaper
\end{flushright}
\end{quote}

Computer science had previously grappled with what was known as the
\textit{double spend problem} since first networked computers and databases were
invented. In short this problem is concerned with how to ensure that a digital
representation of value is not copied and consumed by multiple sources which
require it. The digital banking and credit card industries had dealt with a
similar set of problems in which digital ledgers recording payments would need
to be reconciled or communicated to multiple parties in order to ensure that
digital representations of units of value were kept in sync with deposits. The
existing solution to this problem had always been centralised trust authorities
which held a legal obligation to maintain authoritative digital records and
would handle discrepancies by introducing time delays for compliance checks and
manual mechanisms to handle disputes and inconsistencies.

\index{double spend problem}

The mechanism described by bitcoin proposed a novel idea to the double spend
problem which did not require a central trust authority and instead relied on a
class of computer program known as a \textit{consensus algorithm} and a clever
use of a digital signature scheme to maintain a consistent record of a ledger
database across multiple computers without the need for an authoritative central
source of trust. A digital signature is a technique in cryptography where a user
uses a set of cryptographic keys (a public and a private key) to sign a piece of
data to verify the integrity and origin of the data. The public key can be
shared with the world and authenticated by associating it to other an identity.
The private key is kept secret and is used to generate digital signatures which
only the user who possesses the private key would be able to generate. This
technique was extensively studied and widely used in existing internet
infrastructure such as SSL which protects online banking and other secure
services.

\index{cryptography}

A consensus algorithm is a class of program in which multiple computers
use a set of steps to write a shared set of data (such as a database) in
a way such that all users can access a consistent view of the data from
all computers. Consensus algorithms are a branch of a field of computer
science called distributed systems which concerns itself with different
approaches to building consensus algorithms and different techniques of
sequencing read and writes to shared data stores. All consensus
algorithms are constrained by a fundamental result known as the CAP
theorem which states that any algorithm may have at most two out of
three properties: Consistency, Availability, and Partition Tolerance.

\index{consensus algorithm}

\begin{itemize}
\tightlist
\item
  \textit{Consistency} - A readers of data from the service see the same data at
any given point of time
\item
  \textit{Availability} - All readers can retrieve the data
you stored in a system even if a subset of the service goes down.
\item
  \textit{Partition Tolerance} - No set of failures less than total network failure
result in the system responding to queries incorrectly.
\end{itemize}

The bitcoin network chooses the availability and partition tolerance properties
of this triplet.

The bitcoin algorithm took a particularly interesting approach to consensus in
that it attempted to create a censorship resistant network in which no
participant was privileged. The consensus process was not only eventually
consistent but also tied the addition of writes to the solution of a
computational problem in which computers which participated in the consensus
algorithm would need to spend a given amount of computational work to attempt to
confirm the writes. This approach known as \textit{proof of work} created what is known
as a random sortition operation in which a network participant would be selected
randomly and probabilistically based on how much computational power (called
\textit{hashrate}) performed to attempt consensus.

\index{proof of work}
\index{hashrate}

This created a computational game mechanic in which computers in this network
(called \textit{miners}) competed to perform consensus actions and successfully
confirming a block of transactions gave a fixed reward to the first ``player''
to commit a set of transactions. The rules of this game were defined by the
shared software that all participants ran which defined the protocol of the
network. The key notions encoded in the protocol are the predetermined release
schedule, fixed supply, and support for protocol changes with support from a
majority of participants.

\index{miner}

One of the core algorithms used in most blockchains is known as a hash
function. A hash function is a classic cryptography algorithm in which
data is repeatedly scrambled in a process that is difficult to reverse
and produces a unique fingerprint of the data unique to the
given input. The output of this function is an inordinately large number
which is often encoded in the hexadecimal (base-16 number system).

\index{blockchain!data structure}

\begin{verbatim}
Hello World
d2a84f4b8b650937ec8f73cd8be2c74add5a911ba64df27458ed8229da804a26
\end{verbatim}

Small changes to the input of a hash function alter the output. The
swapping of the case of the ``w'' in the statement for instance changes
the output drastically.

\begin{verbatim}
Hello world
1894a19c85ba153acbf743ac4e43fc004c891604b26f8c69e1e83ea2afc7c48f
\end{verbatim}

\index{cryptography!hash function}

The output of the hash function follows some well-studied statistical
distributions, and the probability of an output prefixed by a fixed number of
zeros can be predicted to occur within a certain amount of hashes. This allows
the difficulty mechanism of obtaining a certain output to be scaled and gives
rise to the adjustable puzzle that miners are forced to solve in order to
perform block confirmation. This allows the difficulty of bitcoin mining to
be artificially adjusted proportional to the rewards.

The underlying database kept in sync by this algorithm was a specific
constructed data structure known as a \textit{blockchain}. The blockchain is
ledger data structure that holds an authoritative record of all proposed spend
activities of a digital numerical unit of value. A bitcoin is represented as a
decimal value whose spend activities were enforced by the consensus algorithms
and could be transmitted to other accounts, recorded on the blockchain, and
mapped to addresses which corresponded to public cryptographic keys. This
created a distributed digital ledger which recorded debits which could be
continuously updated over time as the bitcoins were created and spent.

\index{blockchain}
\index{distributed ledger}
\index{blockchain!accounts}
\index{blockchain!definition of}

The censorship resistance of this algorithm was the key improvement over
existing systems which previously had a single legal point of failure, in that
the central register or central node would have to be stored in a single server
which could be targeted by governments and law enforcement. In this trustless
peer-to-peer (P2P) model all computers participated in the network and removal
of any one node would not degrade the availability of the whole
network. Just as previous P2P networks had routed around intellectual property
laws, bitcoin routed around money transmitter laws.

On January 3rd, 2009 the first block\footnote{This is referred to as the
genesis block.} in the bitcoin blockchain was created and contained a simple
message in the Times of London concerning the bank bailouts during the subprime
mortgage crisis:

\begin{quote}
\begin{verbatim}
The Times
03/Jan/2009
Chancellor on brink of second bailout for banks
\end{verbatim}
\end{quote}

Early history of bitcoin saw the technology was largely used as Nakamoto
intended, as an anonymous global digital payment network.  Bitcoin was used by
early-adopter technologists as an anonymous way to pay for goods and services
and as a tip system for authors of online content. One of the first notable
exchanges was for two pizzas in exchange for 10,000 bitcoins on March 22, 2010.
The first evidence of a price bubble was in June 2011 in which the exchange
value for 1 bitcoin moved from \$30 down to \$2 in December 2011 after a hack of
a major bitcoin exchange website.  In 2013 the technology began to receive
mainstream attention from the press and this era represents the philosophical
transition of the technology from use as a hypothetical digital currency into an
asset for investment. The rapid price movement was then uncorrelated with
traditional assets and proved an interesting investment for a class of investors
looking for new speculative opportunities. After this point the single defining
feature of bitcoin no longer became its utility for anonymous payments but
solely in terms of its price action as speculative investment.

The original author (or authors) of the technology withdrew from participation
in the network and any communication thereafter, and development was taken up by
other software developers and companies.  This era marks a rapid expansion of a
cottage industry of startups and early adopters who would build exchanges,
mining equipment, and a marketing network to proselytize the virtues of this new
technology. The culture around the extreme volatility of the asset created a
series of memes within the subculture of HODL (a portmanteau of the term
``holding'', standing for ``hold on for dear life'') which encourages investors
to hold the asset regardless of price movement. This investment philosophy
became central to the idea cryptocurrency sector and was a statement of faith in
the asset class.  The implicit promise of bitcoin and any cryptocurrency in this
era is that if you invest early you can get rich when the value ``goes to the
moon''.

\index{HODL}

In addition to bitcoin a series of similar technologies based on the
same ideas emerged in the 2011-2013 era. These first movers were
Litecoin, Namecoin, Peercoin, and a parody token known as Dogecoin based
on an internet meme. These projects (called \textit{altcoins}) built on the
bitcoin model but tweaked the implementation of the protocol to allow
for different network behaviors and incorporated different economic
models. As of August 2018 the number of launched cryptocurrency projects
exceeded 1600. In 2015 a major extension to the bitcoin model was
launched as the ethereum blockchain, a project which aimed to build a
``world computer'' in which programmable logic could be expressed on the
blockchain instead of only simple asset transfers. This project would
come to be the second most traded token and would popularize the notion
of smart contracts. In addition to fully visible transaction models of
previous tokens, chains such as Monero and ZCash would incorporate
privacy enhancing features into the design which would allow
participants to have blinded transactions which would obscure the
endpoint details of transactions allowing participants to trade
anonymously on illicit transactions without a public audit record.

\index{altcoin}
\index{Monero}
\index{Litecoin}
\index{ZCash}
\index{Dogecoin}
\index{ethereum}

Unlike in the original bitcoin paper, the idea of ``one computer, one vote'' was
supplanted by the economic reality that large groups of computers could more
efficiently compute the hashes required to confirm blocks and thus reap
financial rewards from the network. Early entrepreneurs realised that if they
built faster and more specialised hardware to compute these hashes they could
gain an advantage over traditional server farms. These entrepreneurs began to
build ASIC (Application Specific Integrated Circuit) which were custom hardware
circuits which could do the computations required for the bitcoin network more
efficiently than traditional CPUs offered by companies like Intel and AMD. This
led to a technical arms race in which dedicated hardware became required to mine
bitcoin and larger groups would build clusters of computers which would pool the
rewards acquired by mining together. These \textit{mining pools} as they would be
called became a centralised and very lucrative business for early investors. In
particular the Chinese company BitMain and others began to centralise most of
the computational resources and by 2019 70\% of all bitcoin mining was
concentrated in
mainland China.

\index{mining pool}
\index{China}
\index{ASIC}

The underutilisation of coal-fired power production and Chinese capital
restrictions on renminbi outflows offered a unique opportunity for enterprising
Chinese citizens to move capital outside of the mainland beyond government
controls. In 2018 the Chinese government officially declared cryptocurrency
mining as an undesirable activity. Bloomberg also reported that year that \$50
billion dollars of capital flight occurred from the Chinese state using the
Tether cryptocurrency as the mode of flight. \cite{leising2020}

\index{China!renminbi}
\index{China!capital flight}

In 2017 a new trend emerged when market participants came to realise that
secondary tokens could be launched on top of the ethereum blockchain. This gave
rise to a controversial new market for ICO (Initial Coin Offerings) which were
offerings of digital tokens to fund project development. In September 2017, an
ICO named Kik raised \$100 million within several days. Many other ventures
would go on to raise unconventionally high amounts of money for early stage
ventures over the 2017-2018 time period. Between January to June 2018 over \$7
billion was raised to fund ICO projects. Most of these projects failed within a
very short time period and exhausted their funds pursuing business models which
would prove intractable and economically unviable. A great number of these
businesses turned out to be fraudulent or thinly veiled \textit{exit scams} in
which the entrepreneurs simply abscond with the crypto assets raised and
never build the product claimed in the prospectus. These cases are still being
litigated to this day.

\index{ICO}
\index{exit scam}

This period also saw the introduction of stablecoins suchs as Tether which would
aim to be a stable asset whose price was pegged to the US dollar and
theoretically backed by a reserve of other assets. After the 2019 era there
would be a period of marked volatility and market consolidation of
cryptocurrencies in which many unfounded ideas would fall off and leave a
handful of 20 projects which would dominate trading volume and developer
mindshare.

\index{Tether}

In 2021 China outright banned all domestic banks and payment companies from
touching crypto assets and banned all mining pools in the country. At the same
time the United States continued to be hit by an onslaught of cyberterrorism and
ransomware attacks that began to attack core national infrastructure and the
country's energy grids.

\index{China!mining ban}
\index{ransomware}
