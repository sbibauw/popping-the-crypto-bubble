\chapter{Valuation}

Any discussion of cryptocurrency ultimately trends towards the central question we would ask of any financial product: What should a crypto token be worth?

The well-known value investor Warren Buffet presented the opinion that most investment professionals arrive at when applying conventional financial valuation models to the question of cryptoassets:

\begin{quote}
Cryptocurrencies basically have no value and they don't produce anything. They
don't reproduce, they can't mail you a check, they can't do anything, and what
you hope is that somebody else comes along and pays you more money for them
later on, but then that person's got the problem. In terms of value: zero.
\end{quote}

\begin{enumerate}
\item Market value --- The price an asset is traded at in a public market.
\item Fundamental value --- The price of an asset derived from an analysis based upon the present value of estimated future cash flows.
\item Exchange value --- The proportion in which one commodity is exchanged for other commodities.
\item Present value --- Present value is the current value of a future sum of a stream of cash flows given a specified rate of return.
\item Sign value --- Sign value denotes and describes the intangible value according to an object because of the prestige or social status that it imparts upon the possessor.
\item Use value --- Use value is a feature of an asset that can satisfy some human requirement, want, or need.
\item Terminal value --- Terminal value is the value of an asset beyond the forecasted period when future cash flows can be estimated.
\end{enumerate}

\section{Asset Classification}

We must bracket our discussion based on a context-specific description of the
underlying asset when discussing abstractions like value. The process by which
we rationally value a Picasso painting and Apple stock is inherently
incommensurate, as one is a piece of art and the other is a financial asset
based on the cash flows of a corporation. We cannot compare financial apples to
oranges.

Thus, to step back and even begin to analyze this question of the value of
cryptoassets, we need to determine what type of asset a crypto token is. There
is no single narrative for what crypto is. However, we can examine the bulk
behavior of people involved with the asset and find comparables for it within
existing markets. While the original intent of cryptoassets might have been as a
cash-like asset for use in a digital payment system, this is overwhelmingly not
the use observed in the large today. Discussion about crypto is often deeply
confusing because its mode switches between referring to it as a currency, a
commodity, and a speculative asset in the same context, sometimes even in the
same sentence. Often, this mode-switching is an intentional intellectual bait
and switch and a form of sophistry to confuse and misdirect.

Transactions on speculative crypto tokens like bitcoin and ethereum are
considerably more expensive than credit card networks and wire services, do not
scale to national level transactions volumes, and lack the most basic consumer
payments protections found in nearly every traditional payment system. No
economy trades in crypto, no large-scale commerce happens in the currency, and
no goods or services are denominated in crypto because of its hyper volatility.
Crypto payments are uniformly worse than any other payment mechanism except
perhaps for illegal purchases.

Although the payments narrative has failed, the initial experiment and ambiguity
around the technology generated paper profits for a sufficiently large enough
number of insiders to cause them to pursue alternative narratives in hopes of
enticing more retail investors to bail out the first failed scheme. In this new
reimagined form, crypto tokens are presented as a speculative asset, and a means
to generate short-term returns rather than technology with intrinsic utility.

Thus we come to the comparable of commodities. Commodities are economic goods
used in commerce. The value of a commodity is derived exclusively from its
use-value. Use value or intrinsic value is a feature of an asset that can
satisfy some human requirement, want, or need or serve a useful purpose. It is
an asset whose demand is generated by organic economic activity rather than
artificial demand or narrative.

Commodities can either be maintenance-free or require a maintenance fee to
sustain their value as a commodity. Pork is an example of a perishable
commodity; it has a finite time horizon and requires an ongoing cost of
maintenance to sustain its value before it can be sold based on its use-value as
food. Metals like nickel or gold are effectively maintenance-free; after they
are produced, they continue to exist on an infinite time horizon and require no
upkeep to sustain their value. Every commodity has an intrinsic industrial or
economic use, which generates a demand for its application.

\section{Theory of the Greater Fool}

Crypto tokens have no such use or organic demand and exist purely to speculate
on detached from any pretense of use-value. Cryptoassets are speculative
financial assets with neither use-value nor fundamental value nor are
non-monetary; thus, cryptoassets cannot be commodities or currencies. The demand
for a crypto asset is not generated by any use-value but from narrative and the
greater fool theory. A financial asset that behaves like a commodity—by virtue
of lack of underlying cashflows—but whose demand is derived purely from its
self-referential exchange value or sign value, rather than use-value, is
sometimes referred to as a pseudo-commodity in academic literature.

We thus conclude that crypto tokens are a pure speculative financial asset that
presents as an exchange-traded investment. Therefore, its valuation rationally
should be assailable by the traditional quantitative finance models by treating
it like any other securities contract. In security pricing models, the
representative investor's expectation today of his expectation tomorrow of
future payoffs must be equal to his expectation today of a future payoff. If
today we expect that we will expect the price to vary at some point in the
future, then rationally, this variation must be incorporated into the price
today. If we are trading a risky asset that will be liquidated at date T + 1 but
is traded at dates 1 to T, only realizable gains on the asset are a function of
its income between dates 1 to T. In a security valuation model, the intrinsic
value is the present value of all expected future net cash flows associated with
the asset calculated via discounted cash flow valuation.

Unlike equities or derivatives contracts, a crypto token does not derive its
value from underlying assets. Equities generate an income from exposure to the
economic activity of a company, while derivatives represent a contractual or
formulaic claim on some underlying asset. A company's stock will naturally
reflect the expectation of future earnings, which flow to the shareholders
through dividends from these earnings or indirectly via stock buybacks or
mergers. A company's stock provides exposure to the market the company serves
with its business activity and is a fractional legal claim on the accumulated
assets of the company. This analysis makes cryptoassets extremely pathological
in the space of financial assets since it has no underlying asset and represents
no claim on literally anything.

By virtue of crypto being a closed system with a finite fixed supply of tokens
issued, the trading of crypto tokens is necessarily a zero-sum game. Since a
token generates no external income, any wins made by one participant speculating
on a token are necessary losses from another participant. Since there is a net
negative cash flow associated with the maintenance cost of mining and
facilitating transactions, then the zero-sum game transforms into a negative-sum
game. The expected return in the limit of iterated trading of crypto is
negative, and therefore an economic game is in the same category as games of
chance in a casino in which a rational player should expect a negative payout.

A crypto token can only be modeled by a game of chance distilled into a
tradeable token, a pure momentum position based on recursive speculation and
untethered to any fundamentals or economic activity. The nature of this type of
game is that its value only goes up so long as the hype monotonically increases
and the market can find increasingly greater and greater fools to buy the coins,
pushing the price higher. Therefore, investing in this scheme is inextricably
linked to an irrational expectation that ever more fools will continue to
believe in the scheme indefinitely in the future. All short-term gains are
ultimately paid out from recruiting later investors into the scheme to pay out
the paper wins of early investors. This type of scheme has a payout structure
that, to a first approximation, is identical that of a Ponzi scheme, albeit an
\textit{open Ponzi scheme}.

The statistician Nassim Taleb also makes a salient observation about the
underlying fragility of purely-narrative-driven financial assets and open Ponzi
schemes in his Bitcoin Black Paper. In the common analysis of games of chance,
there is an event in some games known as an absorbing barrier, which is an event
when a failure event at which no possible gains are realizable. Gambling in a
casino has an absorbing barrier when the gambler exhausts his funds, and markets
have an absorbing barrier when a company ceases becoming insolvent or is not
available to trade. Traditional maintenance-free commodities (such as metals or
gemstones) do not have an absorbing barrier globally because their value is tied
to their physical existence. Since a crypto token is not a physical item, there
is a non-zero chance of regulatory, sabotage, or technical catastrophes that
would cause the network's failure and destruction of all stored value. Crypto
assets like bitcoin have an absorbing barrier associated with their asset
mortality that must be non-zero in probability. \cite{taleb_bitcoin_2021}

Since crypto token produces no income and is a negative-sum game with an
absorbing barrier, the asset cannot be valued on its future cash flows because
there is no yield generated. Trading an asset of this type can generate a return
for individual holders based only on the greater fool theory; the value of a
crypto token is then only what the next fool will pay for it. Thus the only
purpose in buying it now is to find someone who will pay more for it in the
future. Therefore it is an asset that needs to be traded ad infinitum to a
greater and greater pool of fools, all of whom are willing to pay out early
investors for more than they paid. The only allowable solutions under the
discounted cash flow model are that the asset's market value must grow
exponentially, without remission, and with total certainty requiring an infinite
chain of future buyers. The terminal value is then contingent on an infinite sum
of the present valuations chain in terms of an infinite chain of fools. Since
human economic activity and existence are finite, this is not a realistic model,
and as such, cryptoassets cannot have a non-zero fundamental value.

In the rational bubble model, the value paid by increasingly-greater fools must
increase exponentially as a function of the discount rate and the probability of
failure. If the asset does not sustain this growth or stabilizes, then the sales
to the infinite chain of fools cannot overcome the cash outflows, and the yield
of the entire scheme must become negative. Bubbles need to be fed, and bubbles
cannot sustain themselves with stable inflows. Under the rational bubble model,
a crypto asset can only be a "monetary black hole" investment on long time
scales requiring increasing capital to be fed to sustain its ultimately unstable
and transient existence.

Since cryptoassets cannot technically function as a currency, they are
speculative assets with no income, non-zero maintenance costs, negative-sum, and
a non-zero probability of failure from its absorbing barrier. Therefore by
backward induction on its future expected value, we must say that its present
value is zero, and its fundamental value must be zero. The market value of a
crypto asset will fluctuate along with any number of allowed paths as it reverts
to its fundamental value of zero; however, any rational investor should regard
it as presently worthless.

Crypto assets are quantitatively a completely irrational investment, and
theoretically treating them as a sensible asset class necessitates irrational
assumptions of infinities or introductions of absurdities that contradict all of
established economic thought. We are thus left with the most obvious conclusion:
crypto is a bubble much like tulips, beanie babies, and another non-productive
curio that humans have manically speculated in the past. It is a financial
product whose only defining property is random price oscillations along a path
that inevitably leads to its ruin.
