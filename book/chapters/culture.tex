\chapter{Cryptocurrency Culture}

The intellectual center of cryptocurrency culture is the premise to reinvent
money from first principles independent of existing power structures. The
cryptocurrency phenomenon can therefore be viewed as a political struggle over
the fundamental question of ``who should exercise power over money'' in the world
idealized by its acolytes. There is a great insight to learn about the movement
from their manifestos. How a group describes their path to utopia gives a great
deal of insight into their mind and values.

There are three ideological movements and distinct subcultures within the
cryptocurrency movement that gives rise to the intersection that defines
cryptocurrency culture. Accordingly, each facet of cryptocurrency culture is
defined by reverence for a concept that it holds sacred and forms the basis for
its conception of utopia. How these groups frame their advocacy for
cryptocurrency is a normative process that exhibits their implicit
prioritization. The political economy of cryptocurrency contains three distinct
subcultures:

\begin{itemize}
\tightlist
\item
  \textbf{Cryptoanarchism} - A political movement that sees technology as the
    mechanism to dismantle the state.
\item
  \textbf{Austrian economics} - A economic philosophy that believes in
    unconventional views about money and its independence from government
    intervention.
\item
  \textbf{Technolibertarianism} - A culture that believes in the inefficiency of
    institutions and the need to replace them with software.
\end{itemize}

Each of these cultures describes a different end state for the technology which
originates in these worldviews and describes the creative and destructive
mechanisms by which to arrive at the end state. The Nobel laureate and Keynesian
economist Paul Krugman sees cryptocurrency as a destructive force and bluntly
describes \cite{krugman_technobabble_2021, krugman_bitcoin_2018,
krugman_brutal_2021} the phenomenon as a speculative bubble:

\begin{quote}
[Bitcon is] a bubble wrapped in techno-mysticism inside a cocoon of libertarian ideology
\end{quote}

While Steve Bannon, the Trump political campaign strategist reactionary populist
leader, describes bitcoin in terms of a creative force to achieve his
revolutionary ends:

\index{Bannon, Steve}
\index{Trump, Donald}

\begin{quote}
[Bitcoin is] disruptive populism. It takes control back from central authorities.
It's revolutionary.
\end{quote}

\section{Cryptoanarchism}

\index{cryptoanarchism}
\index{anarchy}
\index{libertarianism}
\index{bitcoin!right-wing politics}

Contemporary anarchism emerged after World War II as a counter-cultural movement
that perceived the failure of governments to solve enduring social problems as
endemic to the existence of the state itself. The problems of poverty,
environmental destruction, wars, and gender inequality originated from the
state's social contract and its failures to address these problems. This
movement flourished after the war and found sympathy in a broad mixture of
philosophical, political, and literary sources.

\cite{Cryptoanarchism} or \cite{cyberanarchism} is a political ideology the aim
of which is to achieve the protection of privacy, political freedom and economic
freedom through the use of cryptography and crypto assets. Cryptoanarchism sees
itself as a reaction to the overreach of governments and the state into the
private and financial lives of citizens and asserts the need for so-called total
freedom.

\begin{enumerate}
  \item Total anonymity of individuals in the digital spaces
  \item Total freedom of speech without censorship or moderation
  \item Total freedom to trade without regulation or protections
\end{enumerate}

In this tradition, early internet mailing lists developed an interpretation of
anarchist ideas contemporaneously to the rapid technological evolution of the
90s and the internet revolution. This movement came to be known as
\textit{cryptoanarchy} and is a political ideology that proposes that the
internet and strong cryptography will ultimately diminish the sovereignty of
nations and destroy the capacity of governments. The movement's founding
documents are the 1988 \textit{The Crypto Anarchist Manifesto} and
\textit{Cypernomicon}, which were documents by early Intel engineer Tim May. The
document beings with an allusion to Karl Marx:

\index{cryptoanarchy}

\begin{quote}
A specter is haunting the modern world, the specter of crypto anarchy.
Computer technology is on the verge of providing the ability for
individuals and groups to communicate and interact with each other in a
totally anonymous manner. [...] Arise, you have nothing to lose
but your barbed wire fences!
\end{quote}

The manifesto is a vehemently anti-state document and describes the political
landscape of the 20th century as an escalating conflict between the internet and
the state. The ideology sees the rise of the internet as supplanting the
privileges of the state ``to snoop, wiretap, eavesdrop, and control its
citizens.  This diffusion of power from the state to the internet-enabled
individual (called \textit{cyberspace}) is the heart of its proposed conflict
theory. Cryptography is seen as the core mechanism of building out this
conceptual cyberspace and ensuring its supremacy over the state.

\begin{quote}
When you want to smash the State, everything looks like a hammer.
Strong crypto as the ``building material'' for cyberspace
\end{quote}

John Barlow, an American political activist and founder of the Electronic
Frontier Foundation, drafted a seminal essay entitled \textit{Declaration of The
Independence of Cyberspace} \cite{barlow_declaration_2019}, which declared the
domain of cyberspace to be a sovereign entity. The essay describes the premise
that cyberspace is a distinct political entity and is outside the domain of
other nation-states.

\begin{quote}
Governments of the Industrial World, you weary giants of flesh and steel, I come
from Cyberspace, the new home of Mind. On behalf of the future, I ask you of
 the past to leave us alone. You are not welcome among us. You have no
sovereignty where we gather.
\begin{flushright}
- Declaration of The Independence of Cyberspace
\end{flushright}
\end{quote}

The weak form of this ideology proposes a view of how to protect the individual
against state surveillance. This document accurately predicted government
overreach and the use of surveillance infrastructure on its populace, as
revealed by the Snowden leaks. We now live in an age where digital surveillance
is pervasive, both by governments and corporations. In this conflict worldview,
all the traditional institutions have entered into a holy alliance to exorcise
the specter of strong cryptography. The individual is seen as self-sovereign and
in conflict with the authoritarian state.

\index{self-sovereignty}
\index{Snowden, Edward}

The extreme form of this ideology is a synthesis of anarchism with technology as
the means to ``smash the state.'' Cryptography and digital currencies are seen
as vehicles to wrestle back the state's monopoly on violence by creating
anonymous \textit{assassination markets} in which financial incentives to
assassinate politicians will inevitably destroy governments.

\index{assassination markets}

\begin{quote}
16.23.7. As things seem to be getting worse, vis-a-vis the creation of a
police state in the U.S.--it may be a good thing that anonymous
assassination markets will be possible. It may help to level the playing
field, as the Feds have had their hit teams for many years (along with
their safe houses, forged credentials, accommodation addresses,
cut-outs, and other accouterments of the intelligence state).
\end{quote}

This inflammatory document explores the implications of digital internet money,
which is privately issued, anonymous, and allegedly can overturn central banks'
power. Central to the ideology is the belief that the core functions of the
state can and will inevitably be replaced with software, and the technical
supremacy of these functions will result in the irrelevancy and dissolution of
the state.

Cryptoanarchism is adjacent to a less extreme political narrative known as
technolibertarianism, an extension of libertarianism that adopts an additional
assumption of the ``defense of the absolute freedom of the internet.''
Traditional libertarian ideology maintains that the role of government is
strictly limited to protecting citizens from aggression, theft, breach of
contract, fraud, and enforcing property laws. Instead of the complete
destruction of the state, the technolibertarianism ideology seeks to minimize
the role of government and presents the internet as a great liberating mechanism
on which to build society.  Software is thus seen as a force of liberation whose
purpose is to dismantle bureaucracy. In this worldview, cryptography provides
privacy for the weak and enforces transparency for the powerful.
\cite{assange_cypherpunks_2016} The ideology posits a view where code is
incorruptible and can be trusted while people and institutions cannot.
\cite{krugman_brutal_2021}

\begin{infobox}
 \textbf{
    Cryptocurrency is a technology built around and inseparable from right-wing
    philosophy.
  }
\end{infobox}

Technolibertarianism remains a popular school of thought for many information
technology professionals. This ideology is prevalent in the venture capital and
software engineering culture. Central to this ideology is two modes of action:
\textit{disruption} and \textit{decentralization}. Disruption involves
dismantling existing institutions and power structures by undermining their
mechanisms. While decentralization involves reconstructing these structures in a
form for which power is subsequently diffused to many parties using network
technologies. The term disruption has entered the mainstream lexicon and
describes new ventures who attempt to undermine existing institutions and
business models. The notion of decentralization is central to the blockchain
philosophy. The technolibertarian perspective is that digital currencies are
disrupting and decentralizing the bureaucratic institutions of banks and
supplanting the sovereignty of nations to control the money supply.

\index{technolibertarianism}

This perspective of government disruption resonates with American
ultraconservative political ideology, whose core tenet is that the state is
intrinsically authoritarian and any form of government intervention breaches the
individual's freedom. In this worldview, the state itself is the primary source
of violence, theft, oppression, and all forms of coercion. Freedom and liberty
are synonymous with ``freedom from the state.''

A characteristic of anarchistic politics is the practice of living the movement
as if the desired society already existed. In this practice, one adapts their
present behavior to embody the ideals of the future outcome before it occurs.
This notion of prefigurative politics is central to the cryptocurrency movement,
and the ideation to ``live now in the future we are building'' is a standard
rhetorical device in cryptocurrency marketing and rhetoric.
\cite{jeong_bitcoin_2013}

\index{prefigurative politics}

Part of the contract of living under a democracy is giving up individual
liberties to secure others' collective rights. However, within the
cryptoanarchism worldview, technology has removed this compromise and stems from
the central tenet of their philosophy that freedom is ``freedom from the
collective.'' Technology and software do not offer an escape from government;
they can only offer another competing governance system with its own power
dynamics and coercive institutions. Technolibertarianism is, at its core, simply
a preference for that system over democratic institutions.
\cite{bogost_cryptocurrency_2017, golumbia_politics_2016}

\section{Technolibertarianism}

Developers are as vast and varied as any arbitrary sample of people, and little
about the profession of writing software implies any particulars about culture.
Nevertheless, distinct subcultures within the software have shaped history and
had an outsized influence on the craft of software itself.

The \textit{hacker culture} is a term to describe a subculture and general
mentality toward engineering that emphasizes demonstrations of technical
aptitude and cleverness and a set of ethics around information sharing. This
usage contrasts with the colloquial use of the word hacker to refer to actors
who penetrate computer systems for criminal activity. This hacker philosophy
traced its roots to the Massachusetts Institute of Technology and was
contemporaneous to the early internet development. American journalist Steven
Levy first canonized the hacker ethic in his 1984 book \textit{Hackers: Heroes
of the Computer Revolution}. Levy defines the five hacker values as sharing,
openness, decentralization, free access to computers, and societal improvement.
Additionally, proponents of the hacker ethic often espouse a distaste for
authority and the belief that bureaucracies are inefficient or fundamentally
flawed.

This set of principles and the early internet is an overarching faith in the
transformative power of computers and a belief in meritocracy. The belief is
that one's programming skills are the ultimate determination of one's value as a
hacker and can and should be separated from identities such as age, sex, race,
orientation, position, and qualification.

At the same time, the free software movement arose, which was a social movement
that aimed to create free computer software independent of the corporate
offerings of the day. The notion of freedom was a term of art that referred to
the right ``to copy the information and adapt it to one's own uses.'' This notion
of freedom was jokingly defined as ``free as in free speech, not as in free
beer.''

The MIT researcher Richard Stallman led this project. This project ultimately
evolved into the GNU collaborative initiative to develop free software. This
movement created a legal initiative to license software under a set of
provisions known as the GNU General Public License (GPL), which ensured that
software (and software derived from that software) remained available to inspect
(open source) and available for public use. The defining feature of open source
software is that anyone can freely inspect it and share it for others to
understand its inner workings. The GNU movement was inherently an influential
political initiative and enormously transgressive for its era.

In 1991 the renowned software developer Linus Torvalds released the Linux
kernel. The combination of the GNU software and the Linux kernel would form the
GNU Linux (often shortened just as Linux) operating system. This operating
system was enormously influential in the history of computing.

Beyond the operating system, the open-source movement thrived, and the sharing
of code under permissive software licenses became the norm. Platforms such as
Github allowed rapid collaboration on open source projects by participants
globally.

At the same time, parallel questions were also being asked about digital assets
and what ownership meant in a world of bytes instead of atoms. The technology to
copy and disseminate files freely became available was effectively a solved
problem by 2010. These technologies marked the move toward censorship-resistant
platforms, where information could be shared resiliently against removal by
external actors. The mantra ``information wants to be free'' was an expression of
the hacker mentality that all people should be able to access any information
they desire without constraints. Many software engineers began to believe that
an ever-growing corpus of human knowledge cannot and should not be contained
within any legal framework.

This era also saw many leaks of private government information and dumps of
classified intelligence done by activists such as Julian Assange, a cypherpunk
who started the organization Wikileaks. Wikileaks successfully leaked sensitive
military and intelligence information from whistleblowers within the United
States government and was able to disseminate this information using
censorship-resistant internet protocols.

\index{cypherpunk}

Central to the contemporary developer mindset is that all technology has no
moral character. The design of a device or structure is neither good nor bad; it
simply depends on its use. In other terms, the tool is independent of the
consequences of using it. The internet can disseminate mathematics lectures just
as easily as it can disseminate hate speech. The overarching presupposition is
that whether the design of the software has ethical intent or purpose is not
even a well-formed question. In this worldview, the question simply has no
answer. The perspective of the universal amorality and absolute neutrality of
technology is an extreme view commonly held by technologists; however, this view
that technology is entirely amoral is not widely held outside the industry.

As the internet became more mainstream and intertwined with everyday human
experience, the radicalism of the old hacker culture sublimated into a more
moderate developer culture with a generation of software developers. In place of
the anti-authoritarian proclivities of their predecessors, many software
engineers have embraced a new order of economic growth and prosperity within the
software engineering profession. The craftsmanship of software and hacker
mentality faded into a new vision of market-oriented mentality which prioritized
rapid iteration and customer experimentation.

For tech workers unhappy with the outcome that has unfolded, these individuals
often find their positions simply as puppets of the corporate milieu in which
they find themselves. Without much sway over the corporate hierarchy or change
to alter the course of their careers. Some tech workers feel their work has led
to the overall degradation of civilization but cannot see an actionable path
out.

The anti-authoritarianism of past hackers has faded into the learned
helplessness of the current generation. The hacker culture's idealism of the
internet's transformative power has transformed into a culture of corporate
serfdom and subversive opportunism. Idealism and practical utopianism have
become a parody, and the promise to ``make the world a better place'' has become
a punchline to industry jokes.

A malaise has descended over Silicon Valley as many have come to terms with the
unexpected dystopia that has unfolded in the wake of disruption. In the absence
of advancement in the field, many developers have retreated into
technolibertarian fantasies that center around pipe-dream decentralized
technologies as a panacea to the world's problems. While these dreams may not be
realistic or internally consistent, they at least offer a respite from the
ambient nihilism of the tech hegemony. It is grasping at something, literally
anything, that could be better than the status quo, which seems to be unfolding
into an undesirable tech-led Dickensian feudalism.

The cryptocurrency ecosystem derives its intellectual structure from this
escapist alternative fantasy world. This fantasy presents a future vision in
which decentralized code, artificial intelligence, and universal access to
technology dissolve the coercive influence of modernity's corrupt bureaucracies.
The libertarian ideology has consumed the hacker notion of decentralization,
transforming from an egalitarian vision of progressive inclusion and access to
an ideological one in the service of centralized wealth creation, privilege, and
the power of capital. \cite{narayanan_what_2013}

\index{Silicon Valley}

\section{Austrian Economics}

Austrian economics is a heterodox perspective on economics that embraces a
radical view on laissez-faire markets and the non-interventionist of government
in matters of the economy. It emerged from a series of debates in the London
School of Economics in the 1920s regarding whether centrally managed communist
economies could be so well-managed that they would outpace the market economies
of democratic states. Austrian economics gained some prominence in the late-19th
and early-20th centuries and was the intellectual product of philosophers and
economists Ludwig von Mises, Friedrich von Hayek, and Murray Rothbard.

\index{Hayek, Friedrich}
\index{Mises, Ludwig von}
\index{Rothbard, Murray}

The school of Austrian economics differs from orthodox economics in its
methodology. Instead of proceeding from an empirical framework of observations
and measurements, Austrian economics is a presuppositional framework that
attempts to create a model to describe all human economic activity by purely
deductive reasoning. The framework presupposes core axioms of human behavior,
which form the philosophical foundation of the rest of the theory. The central
tenet, called the action axiom, is the synthetic a priori statement ``human
action is purposeful.''

From this tenet, the Austrian view attempts to describe a framework for rational
choices in the presence of human desires, resource scarcity, ranking of desires,
time preference for resources, and the synthesis of the activities that give
rise to economizing behavior. The Austrians call this line of reasoning
\textit{praxeology}, a pure axiomatic-deductive system that its founder Mises
claims can be knowable and derived independent of experience, in the same sense
that mathematics can be known.

\index{praxeology}

\begin{quote}
{[}Economic{]} statements and propositions are not derived from
experience. They are, like those of logic and mathematics, a priori.
They are not subject to verification and falsification on the ground of
experience and facts. They are both logically and temporally antecedent
to any comprehension of historical facts. They are a necessary
requirement of any intellectual grasp of historical events.

\begin{flushright}
Ludwig von Mises - Human Action: A Treatise on Economics
\end{flushright}
\end{quote}

\index{Mises, Ludwig von}

Mainstream economics arises out of the empiricism philosophy in which all
knowledge is derived from experience, and true beliefs derive their
justification from measurements, observations, and coherence to scientific
models which make falsifiable claims. In contrast, mainstream economists believe
that knowledge is difficult to obtain in social sciences, some truths that are
knowable, and it is possible to posit models that can be falsified. Mises,
however, asserts that no economic truths are knowable empirically and that
praxeology is the only method for deriving economic knowledge
\cite{golumbia_cyberlibertarianism_2013, golumbia_cyberlibertarians_2013}.

\begin{quote}
No laboratory experiments can be performed with regard to human action.
We are never in a position to observe the change in one element only,
all other conditions of the event remaining unchanged [...] The
information conveyed by historical experience cannot be used as building
material for the construction of theories and the prediction of future
events [...] Neither experimental verification nor experimental
falsification of a general proposition is possible in its field.
\begin{flushright}
Ludwig von Mises - Human Action: A Treatise on Economics
\end{flushright}
\end{quote}


\index{Mises, Ludwig von}

The Austrian framework is unconventional in the larger realm of scholarship and
is not widely taught in universities. The presuppositional nature of the
Austrian framework makes it more akin to philosophy than mainstream economics,
which proceeds from explicit models and predictions that can be derived from
data and which are falsifiable. However, some of the individual ideas from
Austrian economics have found integration in mainline schools of thought.

The Austrian school diverges drastically from the other schools in the role of
government in managing the economy. Most other economic schools see the
government as playing an essential interventionist role in managing the economy
and money supply, especially in times of recessions and shocks. However,
Austrian economists see a minimal case for a government role in managing the
market economy. Hayek proposed replacing central-banking controls with a
free-market setting to set interest rates from within the private sector. In
this view, central banks, regulation, and market intervention contain logical
contradictions which achieve the same effect as central economic planning. He
claims these policies empower the state over the individual and disrupt the
ability of the market to transmit information and achieve price information
leading to cycles of instability.

In the Austrian perspective, centralized market intervention turns economic
influence into political power and financial rewards based on non-public
information. This process, in turn, is worse than centrally planned economies as
it does not allow for accurate price formation of assets and ultimately leads
society into a loss of freedom, tyranny, and a state of serfdom. To many
hardline Austrians, all other views are lumped under the amorphous term
``Marxism'' if they involve divergence from the complete free-market orthodoxy. In
this worldview, intervention is inherently undemocratic in that it allows a
select few individuals to exert undue power over the lives of the rest of a
nation and beyond. Hayek argued that fascism is itself inexorably linked to
central economic planning and called this state of being ``serfdom''
\cite{golumbia_bitcoin_2015, golumbia_cyberlibertarianism_2013}.

\index{Hayek, Friedrich}

The central political call of some Austrian economics is to set up a system that
is immune by construction to government meddling. Menger and Mises all argue
that the gold standard is the mechanism that can form the basis for sound money
and argue strongly against the modern fiat central bank system. In this
worldview, the fiat system of money is unstable and will inevitably
self-destruct because it is based on unsound economics and contains logical
contradictions.

Much of these contradictions stem from an analysis of inflation, the rate of
increase in prices for goods and services relative to the current purchasing
power. Many Austrians prescribe what is known as the \textit{monetarist theory}.
This idea is that inflation is not rooted in economic reasons within the economy
but rather only arises from central banks' activities. While it is true that if
we double the number of dollars in circulation, then the classical
interpretation is that this dollar will purchase half as much. Uncontrolled
expansion of the money supply can potentially be one source of inflation, but it
is not necessarily the source of all inflation. \cite{frisch_theories_1983} The
sophistry and confusion are where the monetarist theory devolves into a
right-wing conspiracy theory, which is the redefinition of the word inflation to
mean precisely any event which introduces new money into circulation. In the
Austrian frame of view, inflation has pejorative connotations and is seen as an
undesirable force. In contrast, most mainstream economists view low inflation
($<2\%$) as a desirable feature of a managed economy as it discourages hoarding
and encourages investing in productive assets.

\begin{infobox}
 \textbf{Inflation is not expansion of the money supply}
\end{infobox}

\index{inflation}
\index{Federal Reserve!conspiracy theories}
\index{central bank!conspiracy theories}

The intellectual edifice built by the Austrians to argue for the commodity-based
money was a convenient framework for early bitcoin advocates to adopt. The roots
of the movement within libertarianism meant that advocates for gold could see
sufficient similarities in the artificial scarcity mechanism of
cryptocurrencies. Just as the gold supply on Earth is limited, the number of
bitcoins is similarly constrained by a fixed supply. In the bitcoin
neo-metallism worldview, the decentralized nature of cryptocurrency is
theoretically immune to intervention and driven purely by market forces and thus
fulfills the Austrian notion of a basis for so-called ``sound money'' or ``hard
money.''

\index{hard money}
\index{neo-metallism}

A \textit{gold standard} is a commodity-based currency regime in which money can
be exchanged for gold at a fixed price. In 1699 Sir Isaac Newton was made the
Master of the Mint, and after the accession of Queen Anne in 1702, England was
under a bimetallic regime with silver and gold backing the British pound. Issac
Newton set an exchange rate of 21 silver shillings to 1 pound sterling and
quickly discovered there were difficulties in balancing market rates and the
standard exchange rate. Eventually, with gold prices remaining relatively stable
and silver prices volatile, England switched to a full gold standard which
started the longest-running international gold standard.

\index{gold standard}

The heterodox economists that argue for a gold standard and sound money are
fundamentally pushing for a scenario in which the fixed price of gold constrains
governments, institutions, and people. They argue that this limits government
power by restricting the ability of governments to print money at will. The main
argument against a gold standard is that the supply of gold is relatively fixed;
thus, an economy that is growing can rapidly outpace the money supply and, as a
result, create economic contractions and periodic deflationary pressure. Under
deflationary pressure, prices fall, so it is always rational for individuals to
hoard rather than spend money as much as possible, as it will be worth more
tomorrow than it is today. This self-reinforcing process takes more money out of
the supply, risking deeper deflationary spirals that threaten economic
productivity --- if no one wants to buy anything, the market cannot fund
economically and socially productive activities. Left unchecked, this type of
deflationary cycle is enormously destructive to economies.

When money is fixed to the price of gold instead of the price of goods and
services, fluctuations in the price of goods replace gold's market price.
Between nations, those running external trade deficits face increasing
deflationary pressure. Central banks cannot have too little gold in reserves,
but they can never have enough. This inescapable fact creates a massive
imbalance between nations' financial stability. \cite{roche_understanding_2011}

Historians and economists often cite the gold standard as the reason for the
Great Depression spread from the United States to the rest of the world and may
have influenced the start of the second world war. Fed Chairman Ben Bernanke
said of the gold standard \cite{bernanke_essays_2004}:

\begin{quote}
The gold standard-based explanation of the Depression is in most
respects compelling. The length and depth of the deflation during the
late 1920s and early 1930s strongly suggest a monetary origin, and the
close correspondence (across both space and time) between deflation and
nations' adherence to the gold standard shows the power of that system
to transmit contractionary monetary shocks.
\end{quote}

Modern economics widely regarded commodity-based money as a terrible idea, a
barbarous relic best relegated to the ash heap of history.
\cite{shea_survey_2012} The United States National Bureau of Economic Research
has shown that between 1800 to 1933, there were 15 business cycles. During the
gold standard era, recessions would occur every 3½ years, while since 1972
(after the abolishment of the gold standard), there have been about eight
recessions occurring roughly every six years. Boom and busts are natural, but
governments have the tools to mitigate the volatility and frequency of these
cycles under a modern fiat system.  \cite{mcleay_money_2014}

Nevertheless, after nearly a century of the success of fiat money, the
conspiratorial side of the Austrian movement still views the fiat banking system
as a plot to acquire the gold of citizens and replace it with a system of
illusory value. In their worldview, gold, and only gold, is seen as the asset of
any real persistent value, and the last forty years are seen as a massive
conspiracy for the wealthy to ``acquire our gold.'' This perspective is not
mainstream; nevertheless, it is common in certain ultra-conservative American
political circles \cite{bartlett_forget_2017}. However, recently the bitcoin
movement will often adopt this same narrative, except with the story being
``government trying to steal our bitcoins at all costs.''

Curiously, although cryptocurrency advocates are rather keen on embracing
Austrian economics, the same could not be said the other way around. Within the
contemporary Austrian discourse, bitcoin and cryptocurrencies are often seen as
untenable pipe dreams, and they argue strongly against the notion of bitcoin as
sound money. The blog for the Mises Institute, an Austrian advocacy group,
offers their counter perspective:

\index{Mises, Ludwig von}

\begin{quote}
Some experts maintain that bitcoin will displace the existent fiat money
and will usher in a new era of free banking, which will finally put to
rest the menace of inflation. Unfortunately, this is a pipe dream.
Electronic money will not replace fiat paper money. The belief that it
can stems from a failure to understand the nature and function of money
and how it emerges on the market.
\begin{flushright}
-- mises.org
\end{flushright}
\end{quote}

Nevertheless, cryptocurrency advocates have repackaged the Austrian arguments
and rebased them with bitcoin or other cryptocurrencies as their center. Trade
books central to the bitcoin movement (The Bitcoin Standard) proceed from an
exclusively Austrian perspective to posit the notion of bitcoin as a basis for a
new global reserve currency to displace the United States dollar and an alleged
improvement on gold.

The presuppositional nature of the Austrian framework effectively places it
outside the realm of reason, and this feature forms the basis of integration
into what the industry calls \textit{cryptoeconomics}. The libertarian ideology
and incommensurability of praxeology with other economic schools make the
bitcoin philosophy nearly impossible to debate in conventional terms since it
redefines terminology and contradicts most orthodox economic assumptions. Most
debates about bitcoin in mainstream economics will then simply reduce to debates
about the legitimacy of the nation-state itself and the axioms of praxeology.
The veracity of cryptoeconomics conveniently requires no evidence, testing, or
falsification; it is simply a matter of faith in its underlying assumptions. For
many, the incoherence of this intellectual edifice is irrelevant, as Austrian
economics simply serves as a pseudointellectual posthoc rationalization for
crypto investing and resonates with their existing political ideology.

\index{cryptoeconomics}

\section{Financial Nihilism}

While the ideologies and ideas around bitcoin are varied, the most common
worldview held by most crypto investors is simply a complete lack of any
worldview. In normal philosophy, this perspective is called nihilism: the belief
that all values are baseless and that nothing can be known or communicated.
Philosophical nihilism is associated with extreme pessimism and a radical
skepticism that condemns all of existence and belief in nothing and no purpose
other than an impulse to destroy. Financial nihilism is a ``philosophy'' on
investing and markets where value does not exist or is inherently unknowable.
The claims of this ideology stem from an underlying belief that markets have no
purpose and cannot or do not exist to do capital intermediation on assets
because of an underlying belief that capitalism or markets simply can not work.
With regards to crypto, this vacuous ideology posits that there is no reason or
purpose to anything involving cryptocurrency and that the sole purpose of the
entire enterprise is simply to make the ``number go up'' as a means to demonstrate
the absurdity of all financial structures.

\index{financial nihilism}

In contrast to the previous ideologies, financial nihilism posits no utopian,
political, economic, or social enterprise to cryptocurrency other than to
personally realize financial gains by encouraging others to invest in the same
tokens one is invested in to drive up the price. All is permitted, and the
legality or ethics of any actions which drive the price up are inconsequential
or meaningless to analyze as they have no moral or economic structure. The goal
is always ``number go up,'' and any means justify the ends.

Thus, cryptocurrency is often seen as an absurdist joke or trolling of the
financial system to illustrate the folly of existing institutions or as a
political statement about the inherent emptiness of human existence in a
hypercapitalist society. This mindset stems from a low-information worldview in
which traditional financial instruments such as stocks and bonds are simply
another form of meme and narrative with an equally empty intellectual structure.

\index{meme stocks}

The political economy of the cryptocurrency movement is as varied as the people
involved; however, the movement has spread the seeds of fringe right-wing ideas
to a crowd that would not have otherwise been exposed to heterodox ideas about
economics and monetary policy.
