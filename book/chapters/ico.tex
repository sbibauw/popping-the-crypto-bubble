\chapter{Initial Coin Offerings}

During 2017-2019 there was a massive secondary bubble on top of the
cryptocurrency bubble in which fledgling blockchain companies used the ethereum
blockchain as part of crowd sale activities to sell custom tokens representing
alleged ownership in new enterprises. The history of the Initial Coin Offering
(ICO) bubble is the most explicit witness to the madness of crowds, and that
truth is indeed stranger than fiction.

The namesake of ICO (a mutation of the term \textit{IPO}) comes from the
traditional terminology of an \textit{initial public offering}, an event in
which a company converts private shares of its stock into a product that the
general public may purchase in exchange.

\index{bubble}
\index{ICO}

The first ICO was in 2013 for a small project called Mastercoin. The project
raised \$2.3 million by selling a custom digital token for a specified exchange
amount of bitcoin and ethereum per new token issued. This model eventually
kicked off a deluge of similar investments reaching its peak in 2018. These
projects were highly controversial from a legal, governance, and technical
perspective and were overwhelmingly dominated by outright scams and securities
law violations. The Financial Conduct Authority, a regulatory body in the United
Kingdom, reports that 78\% of listed ICOs were outright scams
\cite{financial_conduct_authority_guidance_2019}. US regulators estimate the
cumulative amount of money investors raised under ICOs at \$22.5 billion. The
companies that raised successfully under this ICO funding model fall into three
categories: blatant scams, self-aware scams, and outright naivety.
\cite{de_jong_what_2018}

\index{United Kingdom}
\index{Financial Conduct Authority}

For ICO exit scams, the strategy is straightforward. One constructs a
fantastical prospectus that makes wild claims about a product or business. One
implies or outright states that investment will increase in value over time and
incur massive returns for early investors. Then one raises the money and then
hops on a plane to a country without an extradition treaty and launders the
money into the local currency.

This is the simplest and most common form of ICO business model. The best
example of this is the April 2018 Vietnamese scam for two companies named Ifan
and Pincoin. The two firms are alleged to have misled approximately 32,000
investors and stolen upwards of \$660 million. \cite{benedetti_digital_2018}

For projects that are not outright exit scams, the name of the game is quite
simple, sell off a minority of the total supply in a presale and retain a
majority of that stake. This \textit{presale} could be sold to accredited
investors such as venture funds and high net worth individuals, given a specific
discount on the initial public offering price. After this presale, a public sale
is offered to the general international public. This offering is usually
marketed around an ICO whitepaper, a prospectus, and a technical document
outlining the alleged proposed vision for the product. Typically paid advisors
and influencers are given a percentage of the presale tokens in exchange for
their help in touting the token. The token is then sold to the public and then
traded. The capital raised may be recycled back to purchase more of the
company's own token, thereby synthetically inflating the price.

\index{ICO!presale}

The entrepreneurs and early investors can then simply release whatever
information they desire to manipulate the price and dump their bags on the
retail market when most opportune for them to profit. The lynchpin in this
scheme is investor information asymmetry and the ability to manipulate this
thinly traded market for personal gain without legal consequences associated
with a regular equity sale. \cite{boreiko_serial_2020}

Institutional investors, such as venture capitalists, can be legally isolated
from the actions of the entrepreneurs they invest in. If the venture they invest
in engages in a dubious activity without their knowledge, the fund will not be
directly liable for the executives' actions if they are detached from the
decision-making. Given their supposed sophistication and performed due
diligence, they are legally considered capable of taking risky positions in
early companies. Indeed, many venture capitalists coerced startups into raising
in this model by offering preferential terms, as this model promised
faster-outsized returns by offloading tokens on secondary markets and exploiting
retail investor information asymmetry. This market configuration inherently
creates a moral hazard for entrepreneurs who are encouraged to raise under a
structure that exposes them, but not institutional backers, to legal risk.

Backroom deals during the ICO bubble have been jet fuel for producing
exceptional returns. These funds can leverage their capital to obtain
significant positions in these ICO presales to gain advantages over the public.
From a macro perspective, this bubble has been a massive wealth transfer from
the participants in general sales of tokens to those in the presale of tokens.
From a purely economic perspective, this is entirely rational behavior for the
funds involved. These venture funds exist to take high-risk, high-return
positions in companies that will generate returns for their limited partners.
However, the investor information asymmetry and the fact that the retail
investors are often on the other side of these trades raises some fundamental
ethical questions.

In 2020 the mobile messaging company Telegram attempted to launch a token called
\textit{Grams} in the most prominent proposed coin offering to date. The company
tried to raise \$1.7 billion for a presale before an emergency injunction by the
SEC \footnote{Securities and Exchange Commission v. TELEGRAM GROUP INC.}
declared that the token was unregistered security and was illegally being sold
to US persons. \cite{noauthor_securities_2020-1} The legal documents around this
case give us the most transparent insight into these deals' underlying economic
structure and backroom machinations. The Financial Times reported that many of
the most significant Sand Hill Road funds (Sequoia Capital, Benchmark Capital,
and Kleiner Perkins) were subscribed to this presale.
\cite{cornish_silicon_2018}

\index{Telegram}
\index{unregistered securities}

The secondary economic question pertains to the fact that the overwhelming
majority of these companies have produced nothing of value. The lack of any
marketable blockchain artifacts raises some existential questions about the
utility of this sector.

The question remains where did all this money go? Not all of it was spent on
Lamborghinis, parties, and cocaine (although a fair amount was). While it is
true that these companies have created jobs, however, this kind of job creation
is equivalent to paying employees to dig a ditch and then fill it back up again.
The parable of the broken window is an economic thought experiment regarding
whether a child breaking a window is a net win for the economy simply due to the
window having to be replaced. The activity of replacing the window has unseen
costs that, when netted over all the participants, are in aggregate negative
over the opportunity costs of other productive activities. ICOs, simply put, are
a society-level misallocation of capital that incurs a massive opportunity cost
in the number of productive things and companies that could be built with said
capital.

\index{Lamborghini}

Within ICO companies, this form of non-productive development is often
performative and in the same economic category as the broken window metaphor.
The company needs only maintain the illusion of credibility long enough to
maintain the liquidity of the initial token, allowing early investors to dump
their bags. If the products produced in the prospectus never go to market, the
employees are paid, but there is no economic output from this activity that
contributes to the economy. Token companies act like boutique bakery that raises
billions of dollars by selling shares in a bread business but never bakes a
single loaf of bread.

The simple fact remains that no company that raised funds under an ICO model has
taken any profitable Product to market. We must ask, what is it about this
fundraising scheme that makes it so pathological and attracts such low-quality
ventures and entrepreneurs?

For coins that are neither exit scams nor thinly-veiled pump and dump schemes,
there is another class of projects with slow-burn failures. This class of
ventures stems from the inability to deliver on unrealistic business defined by
the whitepaper. This kind of venture may be entirely staffed with entrepreneurs
and employees acting in good faith; however, the pressure from investors to
increase the value of the token or to launch a high-growth business can often
result in undesirable outcomes for all parties involved.

\begin{itemize}
\item Immutable
\item Decentralized
\item Trustless
\item Secure
\item Tamper-proof
\item Disintermediated
\item Open/Transparent
\item Neutral
\item Direct transfers of value
\end{itemize}

After the fundraising, the problems these companies face are simply to secure
the funds and exchange a subset of cryptocurrencies for real money. Many
companies raised funds before even setting up a corporate entity or opening a
bank account for their business and faced the daunting prospect of where to
register and bank. Several jurisdictions became ICO-friendly to encourage
innovation, encourage job growth, collect taxes, and expand the possibilities of
having home-grown domestic startup success stories. The most popular choices for
jurisdictions were the Swiss canton of Zug and the island of Malta. The Swiss
banking culture of client confidentiality encouraged many ICO companies to
incorporate in the Zug region and then use the Swiss or Lichtenstein banking
system to convert their bitcoin and ethereum into Francs and enter the
traditional financial system. These funds could then be distributed to British
offshore trusts, often set up in Gibraltar, to hide the funds from taxation and
lawsuits.

\index{Zug}
\index{Lichtenstein}
\index{Gibraltar!trusts}
\index{Malta}
\index{SWIFT}

\begin{infobox}
 \textbf{
    ICOs are a high-risk investment in startups that are sold directly to the
    public and circumvent the regulation on sales of equity by offering
    unregistered securities and often hide funds in offshore accounts in
    Switzerland.
  }
\end{infobox}

However, the actual building of the business was often quite challenging for
these startups. Typically startups go through a series of rounds of funding in
which investors buy equity in the company in exchange for increasing sums of
capital. The sums usually scale with the viability of the business, maturity of
the business model, and opportunity for growth. These are customarily denoted
pre-seed, seed, Series A, Series B, etc. The average Series A for an American
startup is around \$13 million. However, these ICO funds raised capital 10-100
times that of a typical Series A round. All of this money was raised for a
company that had nothing: no existing business, no product, no growth, no
service, and no customers. Startups are challenging even for the most seasoned
entrepreneurs; when first-time founders are handed hundreds of millions of
dollars and simply told to deliver, the reality of this situation can often
produce an impossible setup with perverse incentives. \cite{howell_initial_2018}

The behavior seen by many of these ICO startups was characterized by executive
infighting, lawsuits over corporate governance, drastic turnovers in staff, and
class action suits by investors. Unconventionally these companies had such large
cash reserves that they would often themselves spin-out investment vehicles to
invest in other companies building on the company's proposed token or protocol.
This token "turtles all the way down" approach created a cottage industry of
blockchain consultants, blockchain lawyers, and blockchain developers who were
more than happy to burn through the money raised. According to Bloomberg, "56\%
of crypto startups that raise money through token sales die within four months
of their initial coin offerings". \cite{kharif_half_2018}

The founding teams of these ICO companies were often a zoo of disjoint
personalities and backgrounds. Some ICO founding teams were entirely fictional
biographies with stock photo images pulled from the internet. The Financial
Times reported an ICO with fictional cartoon characters for all founders.
\cite{kelly_ico_2019} There was an unusual pattern of ICO-backed tech ventures
founded entirely by lawyers and social media influencers with no technical
leadership.  From a technical perspective, many of these slow-burn companies
attempted to build the software proposed in their initial whitepaper only to
find that the underlying technology stack they initially proposed was simply too
slow, immature, or impossible to support their product pitched. Many companies
overpromised the capacity of so-called smart contracts to build arbitrarily
complex financial products and were quickly hit by the hard limitations shortly
after investigating the technology. In the absence of experienced technical
leadership, many of these companies attempted to remedy the immaturity of the
software themselves and hired repeated iterations of teams unsuccessfully to
build what they had initially promised.

\index{smart contracts}

As is common in many software projects, the scope of these attempts began to
grow in scope and complexity. To justify the high growth prospects of their
initial raise, many companies started to launch their own bespoke currencies,
which attempted to create entirely new offerings from the ground up. With
absurdly large cash reserves, lack of coherent project plans, and inexperienced
management, many software engineers took it on themselves to simply pilfer these
companies to fund their own projects. These software teams were often defined by
drastic turnover and a constant stream of leadership replacements to steer the
company in a new technical direction to address the lack of marketable products.

This situation may sound like an absurdist farce to outsiders, but this was the
lived experience of many engineers and founders who participated in this bubble
from 2017 to 2020. The absurdity of these ventures was essentially not given
coverage by the tech press except in the events of large corporate scandals.
Tech journalism has a propensity to cover success stories and ignore failures.
During this period, most ICO companies burned their cash reserves on frivolous
pursuits and lavish executive expenses and then died.

The ICO sector was notorious within the technology industry for lawsuits. From
the employee's perspective, the work environment was beset with omnipresent
paranoia, conspiracies, greed, and aimless technical goals that seemed
performative and detached from the concealed goal of the company: inflating the
corporate coin. This environment mirrored the entire philosophy of the
cryptocurrency ecosystem, which is simply "number go up at any cost."

\section{Crazy Coins}

The coin projects that came out of the bubble were a zoo of absurdity. The line
between projects whose originations stem from the Dunning-Kruger effect
\footnote{The Dunning-Kruger effect is a type of cognitive bias in which people
believe that they are more intelligent and more capable than they are. Low
ability people do not possess the skills needed to recognize their own
incompetence.} as compared to outright fraud is enormously blurry and
undecidable. Particularly hilarious examples ended up defining the historical
record of this absurd bubble.

\textbf{BananaCoin} was a fruity coin whose whitepaper described their currency
offering as being like the "Uber for bananas." Two Russian entrepreneurs started
the project and invited the public to "join the organic" revolution by buying a
digital token whose value was theoretically pegged to the export value of 1kg of
bananas from Laos. The token quickly collapsed since surprisingly perishable
bananas don't form an economically sound basis for a reserve currency.

\textbf{KodakCoin} was a strange pivot from the Kodak camera company, whose
business had been slowly degrading with shifts in consumer trends away from
their traditional camera and photo processing business. Kodak's executives
decided to capitalize on the two biggest buzzwords of the day, blockchain and
social media, to launch a new medium of exchange for their photo-sharing site.
The stock shot up 300\% on the announcement but quickly collapsed to below the
original announcement within months. The idea was a poorly convinced executive
fantasy and never executed.

\textbf{PotCoin} was proposed as a "banking solution for the \$100 billion
global legal marijuana industry". In the United States, the disconnect between
state legalization and federal legalization has left many financial institutions
unable to bank marijuana dispensaries without running afoul of federal laws. The
compliance costs for banks to transact with the sector are as high as the
customers of these dispensaries. Chicago Bulls basketball player Dennis Rodman
decided to promote this investment during high-profile trips to North Korea in a
genuinely bizarre turn of events. Despite the sanctions and concentration camps,
the North Korean state is sometimes portrayed in marijuana culture as a stoner's
paradise where cannabis is legal and commonplace.

\index{North Korea}

\textbf{WhopperCoin} was an official Burger King project which launched in
Russia. The premise was deceptively simple; you would eat whoppers and earn a
digital token which you could redeem for more whoppers or speculate on. The
company made it clear that "eating Whoppers now is a strategy for financial
prosperity tomorrow." This revolutionized the fast-food industry and allowed
Burger King's customers to gamble with imaginary money instead of just their
health.

During the height of the ICO bubble, even companies outside of the sector
decided that the action was too good to stay out of. The company Long Island
Iced Tea Corp, a New York-based ice tea producer, made the bold switch from
selling refreshing tea beverages to cashing in on their newfound expertise in
blockchain technology. They accomplished this by simply renaming themselves to
\textbf{Long Blockchain Corp}. The company's stock was traded on the NASDAQ as LBCC
and jumped 300\% when the corporate renaming went public. At its peak, the
company was traded at \$6.91/share; however, six months later, the stock's value
collapsed to penny stock levels, and it was subsequently delisted from the
exchange.

\index{FBI}

In 2018 the nation of Venezuela also decided to get into the cryptocurrency
business with the \textbf{Petro} coin. President Nicolás Maduro decided that
after 17 years of running his country into the ground, the right answer to the
country's economic problems was to issue a new digital currency backed by the
nation's petroleum commodities. This project was a petrol reimagining of the
BananaCoin model. The most notable point in its proposed design was the ability
to circumvent US sanctions. The government ran an ICO that raised \$735 million
in bitcoin, ethereum, and Russian rubles. The New York Times reported that the
entire project was the brainchild of an idealistic young software developer
Gabriel Jiménez who opposed the dictator but thought the currency was a viable
way to bring about reform in his home. Their interests soon diverged, and
Jiménez was forced to flee Venezuela and seek asylum in the United States.
\cite{popper_coder_2020}

\index{ICO!Petro}

\index{Petro}
\index{bolivar}

Coins associated with political leaders became a phenomenon with the launch of
\textbf{TrumpCoin} and \textbf{PutinCoin}. Both were either parodies or
absurdist political gestures to create national currencies for the "true
patriots" of both of these authoritarian figures. On the more righteous side,
\textbf{JesusCoin} made the obvious economic observation that the Church is an
authority that needs to be disintermediated to do straight-through processing to
the Lord. Through the holy power of blockchain, JesusCoin was able to give
"record transaction speeds between you and God's son." While all of these coins
seem like satire, the crypto interpretation of Poe's law led to all of them
raising six to seven-figure offerings on the back of jokes and memes.

\index{Trump, Donald}

The satirical \textbf{PonzICO} released a whitepaper
\cite{cincinnati_ponzico_2017} which gives us the most accurate description of
what drove the true philosophy and mentality that drove this bubble:

\begin{quote}
In today's age, it seems better to promote the plausibility of future
profit rather than waste energy on actually delivering.
\end{quote}

\section{Celebrity Endorsements}

On the back of the speculative bubble of coin offerings, many entrepreneurs
recruited an unlikely variety of people to promote these investments. These
included many celebrities such as rappers and Hollywood actors who used their
influence and social media presence to tout unregistered securities.

According to SEC documents, the highest-profile case involved the rapper T.I.
who promoted FLiK tokens. He allegedly encouraged his followers to buy into the
offering for a purported new digital streaming service. The SEC alleges the
rapper then secretly transferred the FLiK tokens to himself and sold them on the
retail market, reaping an additional \$3 million in profits.

\index{ICO!Flik}

In 2018 the government settled the case of Hollywood action star Steven Seagal.
Unlike in his action movies, Steven was not the hero of his story and lost the
final fight. In 2018 Steven posted a press release on his 6.7 million social
media followers claiming that "Zen Master Steven Seagal Has Become the Brand
Ambassador of Bitcoiin2Gen," which was an offering for an alleged
"second-generation cryptocurrency." The SEC alleges the partners in the venture
gave Steven an undisclosed \$157,000 as a side payment for the promotion.

\index{ICO!Bitcoiin}

In the most underhyped fight of 2018, the SEC fought Floyd Mayweather over his
promotion of a token called Centra sold as an unregistered security. Together
with musician DJ Khaled, the SEC alleges the pair were paid an undisclosed
\$50,000 and \$100,000 to tout the token to their 21 million Instagram
followers. In a curious move, Floyd then disclosed his motivation for the
promotion to all of his followers in a series of bizarre tweets which appear to
implicate himself:

\index{ICO!Centra}

\begin{quote}
I'm gonna make a \$hit t\$n of money on August 2nd on the {[}\ldots{]}
ICO
\end{quote}

The SEC settled the case in 2018, and Floyd was ordered by the SEC to pay for
the damages in the offering.

These cases were simply the highest-profile events that involved celebrities of
note. There are thousands of other cases that authorities did not pursue. The
thread running through all of these stories is that in our era, social media can
be used to influence the general public to purchase financial products, however
absurd and ill-conceived, purely on the clout of celebrity.

\section{Court Cases}

The cases that the government has litigated are generally high-value cases.
These cases are matters of the public record; however, five specific cases stand
out as setting a precedent for future sales. On October 11th, 2019, the SEC
filed \cite{noauthor_securities_2020-2} an emergency action and obtained a
restraining order against two offshore entities of the Telegram messaging
company called Telegram Open Network and TON Blockchain, as they were conducting
an alleged unregistered, digital token offering by selling 2.9 billion digital
tokens called Grams and raised \$1.7 billion. These entities were alleged to
have violated the Securities Act by failing to register their offers and sales
with the SEC. On June 26th, a settlement agreement was reached, which required
the return of \$1.2 billion to investors while the parent entity is required to
pay an \$18.5 million civil penalty. \footnote{Securities and Exchange
Commission v. KIK INTERACTIVE INC.} According to the TON community description,
the entire purpose of the Ton projects involves the creation of and usage of
digital assets and is considered the "first adopted payment cryptocurrency in
the Telegram Messenger Ecosystem."

\index{ICO!Telegram}

On June 4th, 2019, the SEC filed a suit against Kik interactive for allegedly
conducting an illegal \$100 million securities offering of digital tokens. It is
alleged by the SEC that Kik sold tokens to US investors without registering
their offer and sale as required by US securities law. It was allegedly proposed
that Kik would integrate the tokens into their messaging app while creating a
new Kik transaction service and reward other companies that would adopt it. None
of these systems were built when the offer was made. In 2017 Kik's business was
struggling with operations and was running out of cash. In an effort to solve
its predicament, Kik decided to attempt a hail mary pass and engage in this
last-ditch illegal token sales as a means to turn around the company's fate. In
October 2020, the court ruled that Kik had offered unregistered security;
however, the case is still being settled at the time of writing.

On December 11th, 2019, the SEC charged an entrepreneur and his company for
defrauding investors in an initial coin offering that raised more than \$42
million from investors. The SEC alleges Eran Eyal, founder of United Data Inc.,
doing business as Shopin, planned to use the funds from the sales of Shopin
Tokens to create a universal profile for shoppers that would track customer
purchase histories across online retails and give product recommendations. All
this is allegedly on the blockchain \cite{noauthor_securities_2020}. However,
the product was never created. The SEC alleges that Eyal and Shopin
misrepresented supposed partnerships with major retail chains, and Eyal himself
had misappropriated over \$500,000 funds for personal use in rent, shopping,
entertainment, and dating services. Eyed pleaded guilty to operating three
securities fraud schemes. \footnote{Securities and Exchange Commission v. Eyal}

On May 21st, 2019, the SEC obtained a court order to stop an ongoing \$40
million Ponzi scheme. The SEC charge alleges Argyle Coin, LLC and its principal
Jose Angel Aman ran a Ponzi scheme with investor funds
\cite{noauthor_securities_2020-3}. The scheme allegedly involved the extension
of a prior fraud Aman organized using two other companies he owns, Natural
Diamonds Investment Co. and Eagle Financial Diamond Group. The SEC complaint
alleges that Aman made unregistered securities offerings in Natural Diamonds and
Eagle, promising investors that the companies would invest in whole diamonds to
cut down and sell for huge profits. Allegedly, it was claimed that the
investment was risk-free because colored diamonds backed it, and the funds were
to be used to develop the cryptocurrency business.  Instead, according to the
complaint, \$10 million of the funds were misappropriated and given to other
investors for their proposed returns and Aman's personal expenses, including
rent, purchasing horses, and riding lessons.\footnote{Securities and Exchange
Commission v. NATURAL DIAMONDS INVESTMENT CO.}

On March 16th, 2020, the SEC set in place an asset freeze and other emergency
relief to stop ongoing securities fraud committed by a former senator of
Washington state, David Schmidt, as well as Robert Dunlap and Nicole Bowdler to
sell Meta 1 Coin, a digital asset considered by the SEC as an unregistered
security offering. Allegedly by the SEC, there were preposterous claims made,
such as that the Meta 1 coin was backed by \$1 billion in art or \$2 billion of
gold audited by the accounting firm KPMG but were backed by nothing. SEC also
alleged that they told investors that the Meta 1 Coin was risk-free and would
never lose value, and could return up to 224,923\%. They had raised \$9 million,
and instead of distributing the coins, the investor funds were routed towards
paying personal expenses. The SEC complaint alleges that some of the investor
funds were used to buy exotic vehicles such as a \$215,000 Ferrari. Robert
Dunlap, a Meta 1 executive trustee, in a statement to journalists, stated: "I am
looking forward to dismantling the SEC as they are committing crimes against
Humanity in the attempted enforcement of financial slavery," and also "Meta 1's
Service and Victory For Humanity Will Be Everlasting." On an internet talk show
called CryptoVisions, Bowdler at one point tells the CryptoVision's audience
that "the Coin has been specifically architected out of the angelic realm."
Later on, in the show, Bowdler mentioned that "the Archangel Metatron and
Abraham Lincoln" revealed to her what would occur in the world's economic
structure over the next 20 years.\footnote{Securities and
Exchange Commission v. META 1 COIN TRUST} \cite{noauthor_securities_2020-4}

Every ICO raised money from investors under the explicit or implicit expectation
that the buyers will profit from the sale of that token on a secondary market
due to the efforts of the token sellers. This model appeals to entrepreneurs
because it increases the addressable investor pool to include international and
unaccredited individuals who may not otherwise be able to participate. This
token offering also does not grant any legal control or representation in the
company, allowing ICO companies to run unchecked and with no corporate
governance. It also appeals to investors because this virtual equity in a
company is immediately liquid regardless of the company's fundamentals or
economic activity. The tokens are often freely tradable, with no lock-up period,
liquidation preferences, or regulation on insider sales.

The moral hazard of ICO offerings is in the simple fact that they don't
incentivize any business development, capital formation, or economic activity in
the way a standard equity offering does. The resulting structure of raising
shadow equity via crypto tokens does not align incentives between founders,
investors, and employees and encourages a criminogenic environment in which
fraud can thrive. Companies that engage in this sale often create a
Theranos-style long firm whose premise is based on increasingly large token
sales on top of a company that is either empty or fraudulent. For these
companies, the simple fact is: \textit{the token is the product}.
\cite{momtaz_entrepreneurial_2020}

This situation begs the fundamental question of why regulators allow these
blatant scams to continue. This question is a fundamental one of democratic
institutions and goes to the economic realism of law enforcement. The American
system doesn't aspire to regulation by enforcement and acts extraordinarily
slowly. Under the Trump administration, the federal government took a hands-off
approach and hoped that some form of self-regulation would simply occur without
intervention. The SEC appears to be a house divided between a hardline
libertarian non-interventionist camp and a more proactive enforcement school of
thought. In the stalemate between the two camps, the SEC government appears to
be incapable of fulfilling its duties.

In the absence of regularity action, all that emerged is a rapidly growing
ecosystem of bucket shops that do little but hawk fraudulent securities,
proliferate investment scams, and continue to harm the public while regulators
sleep. \cite{zetzsche_ico_2017,cristina_cuervo_regulation_2020}

\index{regulators}

From the purely economic side of this phenomenon, any regulator simply only has
a limited time and resources to pursue prosecution and enforcement. Regulars are
given many additional political tools to enforce rulings; however, the primary
mechanism of action is to bring suits against the worst violations after the
fact. Under-resourced regulators will simply often go after the top 20\% of worst
cases that will result in clear legal precedence and prevent future violations,
but on the whole, the system lacks the resources to pursue every case. An
unsettling fact of our judicial systems has always been that many individuals
can and do get away with illegal activity, even in full visibility, if they are
clever or just sufficiently lucky.

It is even more unsettling that defendants may also use the funds raised in an
illegal offering to mount their legal defense with the spent funds unlikely to
be returned in the form of damages. \cite{levi_regulating_2013} Paradoxically this
creates a perverse incentive for fraudsters to grow the size of the fraud
without regard to legality under the maxim that "the bigger you are, the softer
you fall." The sad reality is that white-collar crime often pays very well in
modern corporate America.

As a metaphor, in cybersecurity, a specific type of hacking attempt called a
denial of service is an attack initiated by flooding a network with superfluous
requests to overload systems. The effect of ICOs can be seen as a metaphorical
denial of service attack on regulators. It is spamming the market with a flood
of fraudulent companies beyond regulators' capacity to intervene, proving a
viable "hacking" attack on the legal system itself. The surprising result of
this attack is that this was an effective way to circumvent regulation and the
law for many individuals.

\section{Tokens As Illegal Securities}

\index{securities}
\index{securities fraud}
\index{Securities and Exchange Commission}

The economic crises of the 1920s and 1930s led to a new variety of laws to curb
the excesses of wild speculation that had created the crises. The particulars of
investment contracts were clarified and became a term of art to describe
intangible financial products sold based on expectations of returns. This period
marked a split between tangible products and investment contracts for the former
retained "buyer beware" expectations while the new laws moved some of this
burden to the seller of the investment contracts to prevent fraudulent sales.
Most states passed legislation known as \textit{blue sky laws} which responded to the
observation that "if securities legislation was not passed, financial pirates
would sell citizens everything in \[the\] state but the blue sky," much like
crypto charlatans are doing today. These laws introduced the now-standard
registration of securities and the introduction of regulated brokers for the
selling and exchange of investment contracts.

\index{blue sky laws}

In 1946 this resulted in a Supreme Court ruling SEC vs. Howey
\cite{noauthor_sec_1946}, which defined a subset of investment contracts known
as \textit{securities} if they met a specific set of criteria which came to be
known as the Howey Test. This landmark case was over whether purchasers of
investment units in an orange grove were purchasing securities from their stake
in the managerial efforts in growing an orange grove. A product is considered a
security under US law when it shares three characteristics:

\begin{quote}
{[}A{]}n investment contract for purposes of the Securities Act means a
contract, transaction or scheme whereby a person invests his money in a
common enterprise and is led to expect profits solely from the efforts
of the promoter or a third party, it being immaterial whether the shares
in the enterprise are evidenced by formal certificated or by nominal
interests in the physical assets employed in the enterprise.
\end{quote}

Since then, the test has been applied to exotic investments from liquor
\cite{noauthor_continental_1967} to chinchillas \cite{noauthor_miller_1974} and,
most recently, to digital assets. Much of the precedent built on this definition
has been an interpretation of precisely what constitutes a contract itself and
what constitutes an expectation of profits in the form of marketing done on the
investment. Importantly, directly marketing a product as an investment for
profit is neither necessary nor sufficient cause for it to be a security.

During the 2016-2018 ICO bubble, many offerers tried to circumvent the Howey
test to sell tokens under the guise of the product being a "utility token" or
"fuel token,"; in which the purchase of the token allegedly serves as a means of
exchange on, or provides access to a function of the network. However, the
presence of external secondary markets in which these tokens were traded based
on their investment potential offered reasonable cause to infer the investment
potential of the token, even if it was not directly marketed as such. The
whitepapers and executives overseeing these offerings would deliberately avoid
terms such as "profit" and "investment" as a means to try to avoid scrutiny as
security. Tokens investments were sold with a wink and nod to potential buyers
that it wasn't an investment, yet the pretense was understood by everyone
involved.

\index{utility token}

The other method to route around the securities laws is the use of dual-purpose
tokens, which can be redeemed for services within a network and also traded
speculatively. In many of these dual-use utility token cases, the smoking gun is
the presence of prominent venture capital investors where the expressed purpose
of their investment vehicle is to return on the investment of their fund. If a
messaging app offered a token that granted the alleged "utility" of being able
to purchase in-app stickers, it is implausible that a fund of this size's intent
is to buy hundreds of millions of dollars of stickers for its own use. Instead,
they intend to use their capital and information asymmetry to gain an advantage
in trading the tokens for a return after the presale. The alleged utility is
simply a very thin legal cover to hide their real intent.

Any decentralized network brings additional operational risk to the business
itself in internalizing operations for anti-money laundering and compliant money
transmission that would typically be delegated to external payment processing
partners. There is no technical necessity for network decentralization or a
blockchain to achieve the ends of this proposal. The disintermediation
potentials posited could equivalently be achieved by centralized database
solutions far more efficiently and cost-effectively.

The implicit intent of all cryptocurrency ventures is simply to raise money from
unsophisticated investors with no oversight in a new form of regulatory
arbitrage trying to circumvent investor protections of the Securities Act. It is
an attempt to return to the wild and free-wheeling excesses of the 1920s that
ultimately led to the market crash of 1929.
