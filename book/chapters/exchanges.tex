\chapter{Exchanges}

\begin{quote}
``The beauty of doing business with a crook is that he always forgives you for
  catching him, so long as you don't stop doing business with him.''
\begin{flushright}
--- Edwin Lefèvre
\end{flushright}
\end{quote}

The vast majority of investors in the crypto market go through a centralized
business known as a cryptocurrency exchange. These businesses, somewhat like
regular exchanges, take traditional currencies such as dollars and euros at an
exchange rate for a given amount of a digital cryptocurrency. The exchange rate
between these two assets is called a \textit{trade pair}. An example trade pair
is USD/BTC, or United States dollars for bitcoin.

\section{Crypto Bucket Shops}

\index{know your customer}

Crypto exchanges act somewhat like standard stock brokerage accounts. Customers
deposit funds held and can be traded on the company's platform. The balances of
their accounts are denominated in their currencies held and can be exchanged for
other assets the exchange allows trading on. For cryptocurrencies, the exchange
acts as a temporary custodian for the funds. For national currencies, the
exchange sometimes enables the withdrawal of funds into regular bank accounts.

Customers will deposit funds with the exchange either through credit card
payments, ACH, or international wire transfers to the exchange's correspondent
banking partners. Ostensibly crypto exchanges make money by charging transaction
fees, offering margin trading accounts, and taking a percentage of withdrawals
from their accounts. However, in practice, these exchanges engage in all manner
of predatory behavior and market manipulation activities, a far more lucrative
business.

Cryptocurrency exchanges are an extraordinarily profitable, as they serve as the
primary gateway for most retail users to interact with the market. However,
these companies have a hidden risk that is transparent to many users. In the
United States, exchange activities are only lightly regulated at the state level
under state-level money transmitter businesses, a framework that offers no
consumer protection. In the events of fraud and market manipulation, customers
have little to no recourse compared to what traditional brokers and equity
market makers are required to comply with under equities laws.

\index{money transmitter}

The largest exchanges by volume have been set up outside of jurisdictions where
the bulk of their customers' cash flow originates. There are a small number of
regulated exchanges. Still, the major exchanges as a percentage of self-reported
volume are overwhelmingly dominated by unregulated exchanges in the Caribbean
Islands and Southeast Asia.

Crypto exchanges are commonly set up in jurisdictions with loose or corrupt
financial regulatory regimes. These include, but are not limited to:

\begin{enumerate}
\item Antigua \& Barbuda
\item Seychelles
\item Malta
\item Jersey
\item Gibraltar
\item Isle of Man
\item The Bahamas
\item Cayman Islands
\item Singapore
\end{enumerate}

The largest cryptocurrency exchange globally is Binance, which was formed in
China, moved to South Korea, then Malta, and currently is unknown where the
company office is headquartered globally. Although allegedly, Binance is based
in the opaque tax haven of the Cayman Islands. Binance has had consistent issuer
maintaining its banking relationships in both Europe and the United Kingdom,
which according to regulators, is because of the opaque nature of the entities.
\cite{feinstein_impact_2020}

\index{Seychelles}
\index{FATF}
\index{China}
\index{Hong Kong}
\index{market manipulation}
\index{front running}
\index{painting the tape}
\index{pump and dump}
\index{Bitfinex}

Many of the CEOs and founders of these exchanges are regularly seen in
jurisdictions on the Financial Action Task Force (FATF) blocklist, interacting
with sanctioned persons. Most personally avoid traveling to both the European
Union and the United States for fear of prosecution.

The corporate structure of offshore crypto exchanges is purposely set up to
avoid regulation, auditing, and any reporting requirements since these exchanges
regularly engage in forbidden activities in traditional markets such as wash
trading, order tampering, price manipulation, pump and dump schemes,
front-running, painting the tape, and trading against their own clients.
\cite{feinstein_impact_2020}

\section{Flavors of Market Manipulation}

\index{wash trading}

Pump-and-dump schemes are rampant in these markets
\cite{shifflett_traders_2018}, and there is no protection against the exchange
itself participating in this activity itself. The hazards of pump-and-dump
schemes result in wealth transfers from less sophisticated participants to
market manipulators. According to Dhawan and Putniņš's study
\cite{dhawan_new_2020} of cryptocurrency market distortions:

\begin{quote}
There is at least one pump on 133 days out of the 175 days in our
sample, indicating that there is almost one pump per day on average.
Such a high rate of manipulation is unprecedented in financial markets
\end{quote}

Usually, the construction of an order book is a regulated activity that informs
how to price formation on traded assets can occur. The SEC enforces a policy
known as the National Best Bid and Offer (NBBO) in the United States, which
requires brokers to execute customer trades at the best available prices across
multiple exchanges. No such rule exists for unregulated cryptocurrencies, and a
commonly seen pattern in this kind of market is the abuse of \textit{order
spoofing} or phantom bids. The exchange places a buy order and then cancels it
seconds later after a trader wants to fill the order to sell. As soon as the
order is placed, the price of the asset jumps and deceptively signals fake
demand.
\cite{cong_crypto_2020}

\index{NBBO}
\index{order spoofing}

The order of execution for trades is also an area where the exchange has
privileged information that it can use to its advantage. Many exchanges operate
their for-profit proprietary trading desk, effectively acting like an in-house
hedge fund that can trade on non-public information of its clients against its
own clients.

There is no regulation preventing any exchange employees from trading on
non-public information or prioritizing their personal trades, manipulating the
construction of the exchanges' order book, or interfering with clients' orders.
Indeed, the ability to insider trade is seen by employees as a perk of working
for a crypto exchange.

Many exchanges are subject to sudden and arbitrary "system overload" events and
market halts, which the change can use to intentionally block trades across the
entire market when it advantages their corporate positions.
\cite{ostroff_binance_2021}

Exchanges often interact with customers through mobile apps, which allow
customers to place orders directly from a mobile phone. This family of apps uses
the same dark patterns as Silicon Valley social media companies, such as
behavioral nudges and push notifications, to push consumers into buying risky
products and create "fear of missing out" on market events to encourage
customers to gamble more.

On the opposite side of direct manipulation, exchanges will offer extremely
risky leverage, positions that are taken out on loan between the customer and
the exchange. Many crypto exchanges over margin accounts allow up to 100x to
125x, figures that are utterly insane, deeply predatory, and unseen in
traditional markets. The extreme volatility of cryptocurrency makes these
accounts unimaginably risky and highly vulnerable to even small amounts of
market manipulation. Market manipulators need only shift asset prices a 1% at
this level of leverage to liquidate high leverage positions. Many exchanges
profit simply from liquidating these accounts and taking transaction fees on top
of these insanely risky positions. Several class-action lawsuits filed in the
United States allege exchange involvement. In a class-action lawsuit brought
against several exchanges in the US, the plaintiffs allege:

\begin{quote}
{[}The defendant{]} acts like a casino with loaded dice, manipulating
both its systems and the market its customers use for its own
substantial financial gain.
\end{quote}

It is challenging to secure reliable information about these exchanges and their
transaction volumes. Most of the volume between exchanges themselves may be a
complete fabrication. It is in the interest of exchanges to simulate liquidity
on their markets by a process known as \textit{wash trading}. Wash trading is when
market participants repeatedly buy and sells their products to themselves to
simulate market action to accounts they control. Wash trading is illegal in all
other regulated markets because it creates false price signals which don't
correspond to any actual economic activity.

\index{wash trading}

Similar to wash trading is \textit{painting the tape}, a form of market manipulation
whereby an \textit{economic cartel} of insiders coordinate to influence the price of an
asset by buying and selling it amongst themselves purely to create the illusion
of demand. Cartels are undesirable in markets because they give rise to
information asymmetry and warp price formation assets in ways that degrade
public trust in the market and inhibit legitimate commerce.

\index{painting the tape}
\index{economic cartel}

How many transactions on the bitcoin network are real exchanges between
independent individuals, or how many non-economic transactions are being created
to simulate volume remains largely unknown. The closest insider information on
this data is the Bitwise ETF proposal to the US Securities and Exchange
Commission, which posits that 95\% of cryptocurrency exchange volume appears to
be fake or non-economic wash trading. \cite{vigna_most_2019, hougan_economic_2019}

Exchanges that operate in the United States must collect KYC identity
information on those opening up accounts. However, regulations around customers
can often be less stringent outside the United States. An unspoken feature of
many offshore exchanges is no anti-money laundering enforcement or know your
customer obligations. This allows their customers to transact outside the law
unbounded by any domestic regulation. \cite{schar_decentralized_2021}

\index{United States}
\index{KYC}

Executives at exchanges noted that a cat and mouse game is being played,
interfacing this wild west environment with the existing banking infrastructure.
Exchanges that allow dollar withdrawals must go through a constant cycle of
shifting shell entities and opening new corporate bank accounts whenever their
previous accounts gather scrutiny or become frozen by officials. This network of
banks and correspondent banks forms the network by which these exchanges can
interface with existing domestic and international transfer systems in the
United States and Europe by routing through proxy institutions which mask the
suspicious origin of funds to and from the exchange.

Cryptocurrency exchanges are also notorious for platform risk. The highest
volume exchanges are almost exclusively the main entry point for new retail
investors and are overwhelmingly unregulated and unsupervised, with little
recourse in the courts when fraud occurs. Exchanges in Asian markets have a
mixed variety of legal reputations, and in Japan, there have been significant
issues with the exchanges acting as gateways for money laundering. In 2018,
Japanese regulators issued orders to six major exchanges: Bitflyer, Tech Bureau,
Bitpoint Japan, Btcbox, Bitbank, and Quoine, for additional scrutiny on
transactions. Many of these exchanges had some form of involvement with the
Japanese organized crime group, the Yakuza.

\index{regulators}
\index{Yakuza}

In September 2020, the United States Justice Department issued a civil
forfeiture complaint against 280 cryptocurrency accounts linked to 11 exchanges
in the Korean region. These accounts were tied to laundering efforts by the
government of North Korea to evade NATO sanctions on the country.

\index{sanctions}
\index{North Korea}
\index{North Korea!sanctions}
\index{North Korea!nuclear program}

Since exchanges hold customer funds, they are enormous targets for hackers.
Cybersecurity problems have been a constant problem for exchanges, and the
sector's history is full of high-profile hacks that have wiped all of the
deposits of many customers. To enumerate the list of hacks could be the subject
of an entire book unto itself, as they are so commonplace. Recently the largest
hack was the Japanese exchange Coincheck, where hackers stole \$530 million worth
of cryptocurrency. As of September 2020, an average of approximately \$2.5M is
stolen every day in cryptocurrency exchange hacks.

In addition to cybersecurity risk, exchanges are often highly vulnerable to
control fraud by their directors. The Candian exchange QuadrigaCX lost \$140
million in customer deposits after its director died under strange and dubious
circumstances and had not shared the keys to accounts with any other employees
in the company. After the executive's alleged death, the exchange's cold storage
cryptocurrency wallets were mysteriously drained.

\index{QuadrigaCX}

The nature of trading cryptocurrency requires investors to simultaneously manage
platform risk, counterparty risk, market risk, and asset-level risk and far
above the due diligence requirements of traditional markets and vastly beyond
the skills of most retail investors. \cite{roubini_great_2019}

Like casinos, crypto exchanges entice customers with false promises of financial
windfalls and get-rich-quick schemes. And they often omit the unspoken truth
that the intermediary company sitting between investors and sellers is often a
dodgy network of shell entities with predatory intentions and which could
disappear with a moment's notice leaving customers with no legal recourse.
