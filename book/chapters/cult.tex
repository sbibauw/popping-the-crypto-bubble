\chapter{The Cult of Crypto}

\begin{quote}
You don't get rich writing science fiction. If you want to get rich, you start
a religion.
\begin{flushright}
-- L. Ron Hubbard
\end{flushright}
\end{quote}

\index{cult}
\index{tribalism}
\index{conspiracy cult}
\index{new religious movements}
\index{bitcoin!cult of}

Crypto assets are inherently negative-sum and, as such, consistently hemorrhage
money. Thus, sustaining the scheme requires increasing the pool of greater fools
requires more than just a value proposition; it requires a self-organizing high
control group that recruits and maintains the inflow of money from its
followers. Cryptocurrency economics requires active investors to hold their
positions and onboard more investors. The system's structure requires new users
to consistently pump more money into the scheme to counter the negative-sum
economics. If numerous people simultaneously lose faith in the asset and decide
to liquidate their positions, this would result in a run that would collapse the
entire structure on which any of these investments rest. There is nothing under
these investments except a collective delusion that ``number go up.''

The high control group that has organically emerged to sustain this scheme is a
new kind of emergent internet phenomenon that exhibits many of the same
phenomena as a traditional religious cult. This kind of movement is broadly one
that provides:

\begin{enumerate}
  \item A framework for making meaning of the world.
  \item A mechanism for control of the practitioners.
  \item A mechanism for self-control.
\end{enumerate}

The classical term of *new religious movement* does not fully encapsulate the
recent decentralized phenomenon of internet-based high control groups. We lack a
proper term of art to describe these movements. However, some sociologists have
coined the new term \textit{conspiracy cult} as a more appropriate description
of the contemporary phenomenon in the internet age.

\index{conspiracy cult}

\section{The Golden Calf}

Non-theistic religious movements like Scientology are the closest to the
cryptocurrency phenomenon. Scientology is a self-help business hiding behind the
front a bizarre theology found in the science fiction writings of its founder L.
Ron Hubbard. Hubbard was transparent about the intent of his religion and was
regularly quoted as saying the entire enterprise was a cash grab and tax
avoidance scheme. Nevertheless, the movement he created persists to this very
day.

At face value, the Scientology movement presents to its followers a doctrine on
self-improvement with a set of prescriptions on actions and behavior that will
reveal the true nature of reality. These scriptures and training are defined by
a set of Operating Levels that the followers must purchase from the church in
ever-increasing amounts of money to reach some form of alleged enlightenment.
Scientology encourages existing members to proselytize Scientology to others by
paying commissions to those who recruit new members, much in the same way
pyramid schemes operate. Other countries have banned such enterprises for
socially corrosive or tax evasion. For example, the German courts have declared
that Scientology operates as a for-profit business masquerading as a religion
for tax avoidance purposes.

\index{Scientology}
\index{Germany}

The cryptocurrency movement shares many aspects of economically-based new
religious movements like Scientology. Crypto is fundamentally a belief system
built around apocalypticism, the promise of utopia for the faithful, and a
process for discrediting external critics and banishing heretical insiders.
\cite{venkataramakrishnan_inside_2021}

There is an almost zealous disdain for mainstream Keynesian economics and what
is perceived as the central banking-led political establishment. The ideology
perceives the ``other'' or the ``oppressor'' as central banks, the Federal
Reserve, financial regulators, and other traditional financial institutions.
These conspiracy theories embrace fringe economic views and propositions that
the Federal Reserve is engaged in a systematic conspiracy to rob the public via
inflation and Keynesian monetary policy.

A key differentiating factor of the crypto ideology is that it lacks a central
doctrine issued by a single charismatic leader; it is a self-organizing high
control group built from individuals on the internet who feed a shared
collective together. It is an organic movement that has arisen, evolved, and
adapted to be a more viral doctrine of maintaining faith in a perceived future
financial revolution in which the faithful view themselves as central. The
inevitability of cryptocurrency's future is dogma that is sacred and cannot be
questioned.

\index{antisemitism}

This narrative of central banking conspiracies is a particularly virulent form
of conspiracy theory because, in part, it is rooted in, and perhaps rightly so,
distrust of financial institutions in the wake of the 2008 financial crisis. The
events of 2008 saw a betrayal of the public trust and an unpopular and divisive
bailout. Public opinion of executives in the banking sector is low and phony
populism is an effective mimetic delivery mechanism for extreme political views
that wrap themselves around a legitimate public grievance.
\cite{varoufakis_yanis_2022} Much of the same financial populism seen in the
cryptocurrency ideology is also found in groups like Occupy Wall Street, albeit
from a different political bent.

\section{True Believers}

As in most high control groups, there is a set of thought-terminating clichés,
which are phrases or sayings that discourage critical thought and meaningful
discussion about a given topic \cite{montell_cultish_2021} (i.e., "The Lord
works in mysterious ways," "Everything happens for a reason"). The result of
asking any critical questions about cryptocurrencies will most often result in a
backlash in the form of one of the following thought-terminating clichés:

\begin{enumerate}
  \item "have fun staying poor" / "hfsp"
  \item "If you don't believe it or don't get it, I don't have the time to try to convince you"
  \item "we're all going to make" / "wagmi"
  \item "we're so early"
  \item "hold on for dear life" / "hodl"
  \item "the dollar is a Ponzi scheme" / everything is a Ponzi scheme"
  \item "now do the dollar"
  \item "FUD"
  \item "few understand"
  \item "bullish"
  \item "to the moon"
  \item "diamond hands"
  \item 🤡
\end{enumerate}

\index{FUD}

Other critical questions on most of the core issues result in whataboutism
concerning state-backed currencies (often the United States dollar) and the
events of the global financial crisis. The dichotomy and perceived conflict
between the traditional financial system and the perceived "crypto financial
system" are also central to the ideology and incontrovertibly and unquestionably
true for its believers. The central ideas of the ideology feed on this false
dichotomy in which the frauds of the financial crisis and scams of
cryptocurrency are an either-or choice in which one must choose the lesser of
two evils. It rejects the logically consistent position to reject the excesses
of both. \cite{golumbia_zealots_2018, wolf_libertarian_2019, golumbia_cyberlibertarians_2013}

\index{whataboutism}
\index{financial crisis}
\index{thought-terminating cliche}
\index{whataboutism}

The demographics of socially active cryptocurrency users are difficult to state
precisely. However, anecdotally, many articles have noted that cryptocurrency is
a primarily male-dominated space \cite{penny_four_2018}. One crypto fund's 2019 survey
of recent blockchain startups found that 85\% of employees and 93\% of
executives were men. A 2018 North American bitcoin conference had 86 male
speakers and one woman, which is not surprising considering the party for the
conference was held in a Miami strip club. The pejorative term \textit{crypto
bro} expresses the prevalence of a reactionary and unenlightened hypermasculine
culture that overwhelmingly dominates cryptocurrency spaces.

\index{crypto bros}

The communities and ideologies for the cryptocurrency subculture are fostered in
mediums such as Twitter, Telegram groups, 4chan message boards, Reddit, and
Facebook groups. In cryptocurrency culture, promoting a specific investment is
"shilling" for the coin. The term shilling comes from casino gambling, where
shills are casino employees who play with house money to create the illusion of
gambling activity in the casino and encourage other suckers to start or continue
gambling with their own money. For many involved with the cryptocurrency scene,
shilling and pump and schemes are their primary source of employment. Their
income and livelihood are locked up in the value of a specific portfolio, and
the promotion of this token is their full-time job. A growing set of YouTube and
Twitch streaming personalities derive income proselytizing for the movement and
promoting specific investments without disclosing their financial interests.
[@ftSchills]

\index{shills}

\index{Telegram}

For many millennials and young adults, organized religion and the nation-state
are declining as a primary form of identity. The rise of a global communication
network, free trade, and decades of relative peace has given rise to an internet
culture that increasingly dominates our lives and shapes our primary identities.
At the same time, traditional institutions such as family, unions, and religion
are declining in the influence and ability to organize communities and provide
meaning and purpose. Internet groups are where they spend their time, find
friends and seek validation. For many in this mode of thinking, it is a
seemingly natural proposition that the internet itself should be the issuer of
money, employment, and identity independent of national boundaries.

The cryptocurrency ideology provides a psychological, philosophical, and
myth-making framework \cite{faustino_myths_2021} that, for many believers,
provides sense-making for a world that seems hostile, rigged against them, and
out of their control. The crypto movement fits all the textbook criteria of a
high control group; it provides a mechanism for determining an in-crowd and an
out-crowd (nocoiners vs. coiners). It gives a framework for assessing the virtue
of other followers based on their faith (HODLing) in the cause. It offers
uncomplicated and pithy answers to complex economics and monetary policy issues.
It gives a linguistic framework of thought-terminating clichés and acronyms to
quell dissent. It provides a mechanism of social control in which one can
acquire influence and status in exchange for proselytizing and onboarding more
followers to buy tokens. It makes miraculous promises of wealth, not derived
from labor or economic activity but purely from faith. And finally, the ideology
presents an eschatological narrative of retributive justice about the end times
of the global financial system, in which the true believers will be reborn with
a new life in an anarcho-capitalist utopia.

\index{nocoiner}

The mission of proselytizing the currency takes on religious tones and requires
constant displays of faith in the cause \cite{@golumbia2018zealots}. These
communities are culturally isolated groups where dissent is met with hostility,
ostracization, death threats, doxing, and harassment.

An unexpected result of the internet era is that negative sentiment and
crackpottery spread more virulently than they did ten years ago. With the
development of platforms like Facebook, movements that would have been relegated
to the fringes of society have now found their path to the mainstream. It is
unclear what to call phenomena such as QAnon and cryptocurrency. Some media
outlets have identified the phenomenon as a \textit{collective delusion} or
\textit{conspiracy cult}. These groups defy traditional definitions of religious
movements by virtue of not being centrally organized or having single ordained
messages. The inescapable truth remains that these phenomena are speaking to a
sense of belonging and community people are not otherwise finding in their
lives. In the future, increasingly elaborate and bizarre conspiracy movements
based on economically-incentivized motivated reason will come in to fill this
void in people's lives and will likely evolve into many new forms. The
conspiracy movements of QAnon and cryptocurrency are the vanguard of a recent
phenomenon that society will be dealing with in the next century.

\index{conspiracy cult}
\index{bitcoin!collective delusion}
\index{QAnon}
