\chapter{Economic Shortcomings}

\begin{quote}
Money, again, has often been a cause of the delusion of multitudes. Sober
nations have all at once become desperate gamblers, and risked almost their
existence upon the turn of a piece of paper... Men, it has been well said,
think in herds; it will be seen that they go mad in herds, while they only
recover their senses slowly, and one by one.
\begin{flushright}
Extraordinary Popular Delusions and the Madness of Crowds -- Charles Mackay
\end{flushright}
\end{quote}

Civilization has gone through an evolution from trading in wheat, to trading in
gold, to trading in notes backed by gold, to trading notes backed by nothing, to
trading in digits backed by notes \cite{eich_currency_2018}. Each step along
this process carried a shifting set of concerns and compromises, and ultimately
shifted the custodianship of our monetary supply to different entities in our
civilization.  The role of money has a descriptive definition as having three
core properties:

\begin{itemize}
\tightlist
\item
  \textbf{A medium of exchange} - Money can be used to facilitate transactions
  between parties who agree on it for exchange for goods and
  services.
\item
  \textbf{A unit of account} - Money can be used to track gains and losses across
  multiple transactions and aggregated across these transactions. It can be
  represented as a single numerical value and used to compare the value of
  other goods.
\item
  \textbf{A store of value} - Money provides a means to store value from the
  present for reliable use in future transactions.
\end{itemize}

Because of the scalability (See: Technical Shortcomings) and unsuitability as a
store of value (See: Digital Gold) cryptocurrency simply cannot fulfill the
functional definition of money. It does not have a legal framework that
recognises it as a medium of exchange and is therefore not legal tender.  It is
not used as a mode of exchange because it cannot scale and is simply unfit for
this widespread use. \cite{roubini2018exploring, budish2018economic}

Even the term cryptocurrency is a misnomer and misleading. Many scholarly works
on the subject use the term \textit{cryptoassets} instead to seperate the false
conception that cryptocurrencies are related to monetary instruments.
Nevertheless, the term has entered the public lexicon and in this text we will
adopt its colloquial use despite the terminology confusion.

\begin{infobox}
  \textbf{Speculative crypto assets like bitcoin cannot perform the function of
  money}
\end{infobox}

\section{Negative-sum Games}

In the absence of cryptocurrency having the properties of a proper currency, we
should consider the economics of investing in it as we would other assets. In
the mathematical formalism of economics we study economic real-world economic
phenomenon as modeled by simplified models called \textit{games}.

A \textit{zero-sum game} is a specific class of game where any one player's gain
is equal to the other player's loss on any given play of the game. We model the
game as a \textit{payoff matrix} where the outcomes of the participants are given
rows and columns (A \& B below). A two person zero-sum game is a game where the
pair of payoffs for each entry of the payoff matrix sum to 0.

A game of coin flipping is as an example of a zero-sum game, Two players, A and
B, simultaneously placing a coin on the table. A player's payoff depends on
whether the coins match or not. If both of the coins land heads or tails, Player
A wins and keeps Player B's coin. If they do not match, then Player B wins and
keeps Player A’s coin.

\begin{table}
  \setlength{\extrarowheight}{2pt}
  \begin{tabular}{cc|c|c|}
    & \multicolumn{1}{c}{} & \multicolumn{2}{c}{Player $Y$}\\
    & \multicolumn{1}{c}{} & \multicolumn{1}{c}{$A$}  & \multicolumn{1}{c}{$B$} \\\cline{3-4}
    \multirow{2}*{Player $X$}
    & $A$ & $(1,-1)$ & $(-1,1)$ \\\cline{3-4}
    & $B$ & $(-1,1)$ & $(1,-1)$    \\\cline{3-4}
  \end{tabular}
\end{table}

A \textit{positive-sum game} is a term that refers to situations in which the
total of gains and losses across all participants is greater than zero.
Conversely a \textit{negative sum game} is a game where the gains and losses
across all participants sum to less than zero, and played iteratively with
increasing participants the number of losers increases monotonically.
Since investing in bitcoin is a closed system, the returns one can realise can
only be payed out from funds that payed in by other players buying in.

As an investment, cryptocurrency is fundamentally different in kind from
traditional instruments such as stocks, bonds, or physical commodities, because
it has absolutely no future income other than the money provided by the
investors themselves. Stocks track the economic growth of a business, impart
ownership and pay dividends while bonds carry the legal obligation of redemption
with interest, and commodities have consumers who will buy the asset for its
intrinsic value. Cryptocurrency is a non-productive asset. The negative sum game
of investing in cryptocurrencies is simply a game of libertarian musical chairs
in which there are no future cash flows and nothing of value is created.
Participants simply shift their holdings around to each other attempting to not
be left holding the bags when the music stops. This model goes by the name of a
\textit{greater fool} asset in which the only purpose of an investment is
simply to simply sell it off to a greater fool than one's self at a price for
more than one paid for it. \cite{bank2018v}

The only input of wealth is participants buying tokens on the market. The only
output is people selling tokens on the market. Tokens change hands but nothing
of value is created. However there is a net drain of the total wealth from
transaction costs, market fees and from miners minting new coins increasing the
supply and cashing them out. \cite{pyramid, catalini2016some} Coins are sold by
miners to pay for their power and operating expenses to maintain their
equipment.  As of 2020 the current outflow on the market was a drain of \$3.84
billion per year or around \$10.54 million per day. There will be some winners
but most participants will lose money. The only net winners on long time periods
are the cryptocurrency exchanges and market makers who capture the outflow from
the closed system. Therefore investing in bitcoin or other similar crypto
investment is a negative sum game.

\begin{infobox}
 \textbf{Bitcoin and other cryptocurrencies like it are a negative-sum game with
  more losers than winners, and as an investment has a negative expected
  return.}
\end{infobox}

\index{negative-sum}

Investing in cryptocurrencies has the same game theoretical mechanics as
investment as other negative-sum games like lotteries, casino gambling, pyramid
schemes, Ponzi funds, penny stocks, and multi-level marketing schemes. Is it
possible that some participants will make money speculating on cryptocurrencies?
Absolutely. Some participants also make money from multi-level marketing and
playing roulette in Vegas, but overwhelmingly most do not. In negative-sum
activities more money goes in than comes out.

\index{gambling}

The reporting bias between winners and losers of this rigged game ultimately
means that we'll likely never hear about the majority of losers as compared to
the minority of winners. Few will advertise their bitcoin losses on social media
because of shame and embarrassment associated with losses. This knowledge-gap
divides cryptocurrency traders into two classes. Those that don't understand the
mathematics and believe that the price going up means they are inevitably going
to become richer. And those that understand the mathematics involved and are
hoping to extract money from the fools.
\cite{shifflett2018some, makarov2020trading}

\index{bitcoin!economic model}
\index{bitcoin!as pyramid scheme}

\section{The Bubble Bath}

\index{bubble}

In the opinion of eight Nobel Prize laureates bitcoin is in their words an
economic bubble or a pure greater fool investment. In 2017 the Economics Nobel
prize winner Joseph Stiglitz condemned bitcoin and said of it:

\begin{quote}
Bitcoin is successful only because of its potential for circumvention [and] lack
of oversight. [...] It seems to me it ought to be outlawed
\end{quote}

To say cryptocurrencies are an economic bubble is partially inaccurate though,
the word ``bubble bath'' is a more apt description of the phenomenon. It is an
ever growing set of economic bubbles which inflate and pop giving rise to a new
bath of bubbles before anyone has had time to recover from the last one.

Unlike in other economic bubbles, such as the dot-com bubble of the early 2000s,
there is largely no economic activity to speak of during the cryptocurrency
bubble growth. It is a system that is largely insular, trades within its own
boundaries and remains overwhelmingly parasitic to the external financial
system. There is no economy in bitcoin. There are no fundamentals to nearly any
coin. Nothing is priced in bitcoin. No commerce is done in bitcoin. No
government recognises bitcoin as legal tender or collect taxes in it.  The price
of bitcoin simply floats otherwise randomly subject to constant market
manipulation and public sentiments of greed and fear, detached from any economic
activity.

The absence of an economy around cryptocurrency gives rise to a set of fair
thought experiments regarding macroeconomics of cryptocurrency adoption. The
dollar itself does not exist on its own, it is an integral part of the US
economy. The US economy and the dollar cannot be meaningfully separated; the
dollar is backed by a monopoly by the United States' ability to print the dollar
and protect the currency. In an abstract sense the dollar represents a unit of
the United States economy and its gross domestic product and the government can,
as part of managing its economy also manage the currency.

\index{United States}

\begin{infobox}
 \textbf{Speculation drives cryptocurrency price formation. People buy
  cryptocurrency because they believe they can sell it at a higher price later,
  in dollars.}
\end{infobox}
\index{bitcoin!speculation}
\index{bitcoin!as speculative bubble}

Cryptocurrency is not a Ponzi scheme in the traditional definition.  It bears
all the same economic structure of one except for the key differentiation of a
central operator to make explicit promises of returns. A variant known as a
\textit{decentralized ponzi scheme} (or \textit{Nakamoto scheme} or
\textit{snowball scheme}) makes claims of fantastic investment returns
independent of economic activity but does through internet promoters who are not
bound to a single legal entity (Ponzi operating company), but instead by an
anonymous network of promoters.

\index{decentralized ponzi scheme}
\index{Nakamoto scheme}
\index{snowball scheme}

The economics of cryptocurrency are fundamentally unsound and pathological in a
conventional economics. The story of is just a retelling of the same Ponzi
scheme story: money for nothing out of nothing, just get in early and don't ask
where it comes from.
