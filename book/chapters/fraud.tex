\chapter{Frauds \& Scams}

Cryptocurrency markets are undeniably rife with fraud, scams, and abuse.
Nevertheless, to understand the reasons for scams' proliferation, we need to
understand how and why fraud occurs in the modern business environment.

Trust is an essential part of any business relationship, and the foundations of
trust in any business relationship have three components: judgment, expertise,
and consistency. Positive business relationships are gained from interactions
between counterparties that reinforce the positive experiences that both parties
go through. Judgment and expertise are achieved when a party proves that they
can make successful decisions based on specialized knowledge and expertise.
Consistency is the repeated successful reinforcement of expectations.

\textit{Transitive trust} is a phenomenon in which the trust of one party is
extended to another party based on their trusted parties.

\begin{quote}
A trusts B \\
B trusts C \\
therefore A trusts C\\
\end{quote}

The primary attack vector in many frauds is exploiting some notion of
\textit{transitive trust} to use the fact that due diligence is often not
performed under the assumptions that others have already done it.

\index{transitive trust}

In advanced economies, fraud is always a possibility, but it is usually a tail
risk that occurs with a low probability compared to the bulk of routine
transactions. Fraud controls and rigorous due diligence are expensive relative
to the likelihood of the fraud and, unless otherwise required by law, are many
times discarded for cost-saving.

In high-trust sectors such as tech, a company is often trusted by its employees
simply by virtue of its existing as an incorporated entity. Starting a company
typically involves going through a process involving government and banking
approval since a company requires legal and financial legitimacy for its
existence. Transitive trust naturally exists in firms because it is assumed that
the government and banking institutions have done their due diligence to allow
it to exist at some minimum level. This transitive trust is a behavioral
heuristic that scammers and con men often exploit. \cite{balleisen_fraud_2017}

Inside the enterprise itself, lawyers and accounts act as a line of defense
against fraud. However, these professionals are not immune from corporations or
influence. It is easy for well-funded actors to penetrate this line of defense,
and once it is penetrated, the fraud can often run unchecked. Insiders to
organizations do not often check up on activities ``signed off'' activities. The
facade that an activity, however dubious, has been ``signed off on by the
lawyers'' or ``approved by the compliance department'' is often the primary
method for initiating a fraud inside of an existing enterprise. Once the fraud
is within the perimeter, the fraud can expand by layering on this trust.

\section{Fraud Triangle}

The Fraud Triangle is a framework for identifying environments and incentive
structures that may be vulnerable to fraud by their observed prevalence at the
intersection of three factors. The framework was first described by American
criminologists Donald R. Cressey and Edwin Sutherland and has become a
cornerstone of modern risk management.

\begin{itemize}
\tightlist
\item
  \textbf{Motive} - The perpetrator personally requires resources beyond what
  they can acquire by honest means. Examples:

  \begin{enumerate}
  \def\labelenumi{\arabic{enumi}.}
  \tightlist
  \item
    Debt
  \item
    Greed
  \item
    Lifestyle needs
  \item
    Vices (gambling, drugs)
  \end{enumerate}
\item
  \textbf{Opportunity} - The perpetrator has the means to bypass checks and
  controls that would otherwise prevent the fraud. Examples:

  \begin{enumerate}
  \def\labelenumi{\arabic{enumi}.}
  \tightlist
  \item
    Checks are not enforced
  \item
    Checks are not monitored
  \item
    No segregation of duties
  \item
    Single controller
  \end{enumerate}
\item
  \textbf{Rationalization} - The perpetrator requires logical or emotional
  reasons for why the fraud is not causing harm or is just temporary.
  Examples:

  \begin{enumerate}
  \def\labelenumi{\arabic{enumi}.}
  \tightlist
  \item
    ``I'm not paid what I'm worth''
  \item
    ``I intend to pay it back''
  \item
    ``Everything is a scam anyways''
  \item
    ``Nobody will miss the money''
  \item
    ``Everyone else is doing it, so why shouldn't I''
  \item
    ``The banks are doing it as well''
  \end{enumerate}
\end{itemize}

Within cryptocurrency fraud, the Rationalization and Opportunity aspects are
particularly pathological. With any consumer business (or fraudulent investment
scheme), a network effect dominates the venture's success. Often, a critical
mass of users is required to reach maturity or sustainability for the business
model. The path to sustainability can often be complex. It can create a perverse
incentive for the orchestrator to get there "by growth at any cost" and justify
any fraud that occurs to get there by the ability to offset its harm at a future
date when resources are more abundant. This phenomenon is best exemplified by
the series of tech ventures in the late 2010s, such as WeWork, Theranos
\cite{hans_how_2020}, and Uber, all mired in scandals related to rapid growth
models that ultimately masked rampant corporate wrong-doing, some cases,
outright fraud.

In the growth rationalization model, if one is launching a new currency, then
whatever it takes to bootstrap a market around it is simply the cost of doing
business, and investors can be made whole later. Moreover, if this means faking
invoices, simulating market activity, or lying to investors, this is all
justified by the returns these people will eventually see at some future date.
This fallacy central to some entrepreneurial fraud is rooted in underestimating
the probability of failure. Their enterprise is, in effect, over-leveraged on
delivering the impossible, which becomes a deeper and deeper hole to dig out of,
leading to increased desperation. Enterprises like Fyre Festival and Theranos
are textbook examples of this phenomenon.

\index{Fyre Festival}
\index{Theranos}

Libertarian conspiracy theories and the cult-like culture of cryptocurrency
present a convenient rationalization narrative that ``all money is a Ponzi
scheme'' or that all financial business models are based on fraud. This line of
thinking is the most internally consistent rationalization for engaging in
cryptocurrency fraud and is perpetuated by its culture. In this narrative, any
fraud can be justified by rationalizing their crimes against allegedly more
significant crimes perpetrated by banks or politicians.

The opportunity for cryptocurrency fraud is pervasive simply because the lack of
regulatory checks and controls on these ventures is relatively lax or
non-existent. In an environment where a single user can abscond or run away with
large amounts of investor money, seemingly with little risk to themselves, this,
not surprisingly, will create an environment that will attract less scrupulous
individuals. Cryptocurrency businesses are the perfect storm in the fraud
triangle, and crypto fraud is today's most straightforward and widespread form
of securities fraud.

\section{Ponzi Schemes}

Charles Ponzi was a Swidler in the 1920s who convinced investors of a 50\%
payoff in 45 days or 100\% in 90 days. Supposedly, the investment was in
overseas discounted postal reply coupons. In actuality, it was a massive scam
that resulted in a \$20 million loss within a year and destroyed six banks based.

The mechanism by which a Ponzi scheme function is that new investors' money is
used to pay off existing investors, effectively "robbing Peter to pay Paul." So
as long as the incoming capital flows are greater than the outgoing capital, the
scam goes on and gets larger. Incentivized recruiting of new investors leads to
a pyramid scheme structure. The most notorious version of a Ponzi scheme in
recent history is the story of Bernie Madoff's Ponzi scheme, which lasted
decades with nearly 5,000 investors and resulted in \$64.5 billion of losses. In
the world of cryptocurrency, several similar scams have been uncovered.
\cite{zuckoff_ponzis_2005}

\begin{infobox}
 \textbf{The key to sustaining a Ponzi scheme is controlling redemptions to
  create the illusion of solvency.}
\end{infobox}

\index{pyramid scheme}
\index{Ponzi scheme}

The BBC reported on the most famous cryptocurrency pyramid scheme in a series of
podcasts *The Missing Cryptoqueen*. Ruzha Ignatova was an Oxford-educated
charismatic fraudster who convinced people that OneCoin would become the largest
cryptocurrency in the world. The company was a pyramid scheme with no technology
or network and encouraged people to buy packages of digital tokens that
investors could hypothetically redeem later. By 2017, the United Kingdom,
Austria, Thailand, Italy, and Germany had warned and blocked accounts in their
respective countries. Authorities have arrested collaborators such as her
brother and lawyer, but Ruzha remains at large, likely hiding in Russia or dead.
OneCoin lasted from 2014 to 2017 and resulted in \$5 billion in damages to its
victims. Most of the investors hurt most were low-income individuals and retail
investors who had been sold on narratives of financial windfalls and financial
prosperity on the back of Ruzha's marketing.

\index{OneCoin}
\index{United Kingdom}
\index{Germany}

BitConnect existed from 2016 to 2018 and was run by an individual named Satao
Nakamoto. The scheme promised 40\% profit per month, with \$1000 initial
investment making \$50 million in 3 years. The platform was created as a trading
bot, but the platform was set up to lock investors' funds and give new investors
money from old investors. By January 16, 2018, regulators from Texas and North
Carolina issued a cease and desist to BitConnect's lending and exchange
operations, pointing to unrealistic investment payout expectations and a
multi-level marketing structure that was a Ponzi scheme.

\index{BitConnect}
\index{regulators}

\section{Pump and Dumps}

Pump and dump schemes have been a common form of market manipulation ever since
the first stock markets. They are a form of ostensibly victimless crime in which
a set of insiders manipulate market information to distort price formation to
their advantage. Pump and dump schemes were rampant leading up to the Great
Depression and subsequently became illegal in the United States in the 1930s
after the passing of the Securities Act.

\index{Great Depression}
\index{Securities Act}
\index{pump and dump}

The 2013 movie \textit{Wolf of Wall Street} fictionally depicts the misadventures of
the notorious charlatan Jordan Belfort and how his firm Stratton Oakmont
defrauded investors through boiler room tactics that led to a \$200 million loss
of investors. Pump and dump schemes involve a group of individuals communicating
with potential investors to "pump up" and drive up the price of a security with
inflated artificial demand by investors that were pressured and that at a
specific high price point, "dump" and sell the security. This event crashes the
value of the underlying security and causes a significant capital loss to the
investors who were convinced to join.

Since cryptocurrencies are generally an unregulated market, pump and dump
schemes run rampant. Unlike the days of the phone call-based boiler rooms of the
past, modern crypto boiler rooms use Telegram, Discord, and other social media
platforms to create investment groups requiring paid membership. The scammers
then run the price up of a particular crypto coin and exit before the advertised
high price, leaving many followers at a loss. In 2018, The Wall Street Journal
ran an exposé that found \$825 million in trading activity within six months
across 125 pump and dump schemes. The largest of which, Big Pump Signal, has
over 74,000 followers. \cite{scheck_how_2018}

\index{Telegram}

A study of the pump and dump schemes has found that 30\% of all cryptocurrencies
are used in 80\% of pump and dump schemes. Once used on a particular crypto
successfully, it is very likely that another pump and dump will be done on that
same coin again. More importantly, studies did show pump and dump crypto schemes
occur with low volume coins with significant wealth transfers from outsiders to
insiders and result in detrimental effects on market integrity and price
formation. \cite{xu_anatomy_2019, dhawan_new_2020, li_cryptocurrency_2019,
kamps_moon_2018, hamrick_economics_2018, hamrick_examination_2018}

\section{Giveaway Scams}

With the advent of social media, a new class of simple crypto scams has arisen
that utilizes celebrities to claim victims. The premise is simple: a receiving
party suggests one must first send a small amount of cryptocurrency to a given
address with a claim to "double" the amount sent. The sender commits the
transaction, and the receiver runs off with the funds. Since cryptocurrency
payments are not reversible or subject to regulatory protections, the receiver
will make off with the amount sent to their address. This scam is particularly
prevalent because it has virtually no overhead, can reach a mass of victims
quickly, and is unlikely to be traceable if the captured funds are laundered. It
is common to impersonate the names and likeness of cryptocurrency celebrities
like Elon Musk to push their scams in these scams.

\index{Musk, Elon}
\index{giveaway scam}

\section{Distributed Control Frauds}

Frauds need not necessarily have a single origin or perpetrator, and in today's
world, it is often increasingly difficult to identify the origin of some of the
worst scandals of our time. In corporate fraud, a common situation is when
directors of a company set up an organization with a criminogenic culture that
encourages fraudulent activity but does not explicitly direct employees to
perform the actions. A working environment that rewards the correct type of
employees via a particular system of incentives and deterrents implicitly guides
the employees to conduct the fraud themselves, leaving the company's directors
with no criminal liability. Such a setup is known as a \textit{distributed
control fraud}. \cite{davies_lying_2018}

\index{distributed control fraud}

In the United Kingdom, one of the largest financial scandals of the last century
was the Payment Protection Insurance (PPI) mis-selling scandal. Since the early
1990s, banks had begun offering a scheme in which customers could purchase
insurance plans such that if their job was made redundant, they could claim the
insurance policy which would cover their mortgage payments in the interim until
they found a new position. High street banks offered these products alongside
mortgage and credit offerings and turned into an enormously lucrative business
for the banks. This business was so lucrative for the company that branch
managers were strongly incentivized by their bonuses in terms of how many PPI
products they sold to their customers. This situation would result in
mis-selling activity in which products were sold to clients unsuited for the
policy or silently bundled with mortgages in hopes that the client would not
notice. This created an incentive for rank-and-file employees to ignore the
clients' wishes and push the products on their customers simply because their
internal performance was measured. The employees overwhelmingly did not profit
from this action and, while technically to blame, were not themselves the
beneficiaries of the fraud.

A similar situation occurred in the United States with the bank 2018 Wells Fargo
account scandal. In the same kind of setup, the local branch managers were
opening accounts for individuals in their communities without explicit
authorization from these people. This resulted in Wells Fargo opening checking
accounts, credit cards, and a variety of financial products to generate cash
flows for the bank. Just as in the PPI scandal, the rank and file employees were
directed to perform their regular duties but at a scale that was unsustainable.
The employees were not directly told to open accounts; however, the bank's
practices incentivized fraud indirectly by creating a criminogenic environment.

\index{United States}
\index{Wells Fargo}

It is difficult to label any one party as guilty in this action; the directors
are simply acting within the fiduciary interests of the organization they
represent and whom themselves did not explicitly commit fraud. No one person is
to blame, and the ethical problems are spread across the entire organization
such that no one person is likely to have the entire picture. This setup is a
near-perfect corporate crime as the legal system rarely can handle the
evidentiary standard needed to prove fraudulent intent for anyone but the
unsophisticated rank and file employees.

The essence of distributed control fraud is to create bubbles of willful
ignorance in which fear, software obfuscation, or financial incentives create an
environment in which those involved with not have complete knowledge of the
crime. \cite{black_best_2013}

The rise of companies whose business is conducted exclusively using
cryptocurrency networks presents a new form of distributed control fraud that
one uses perverse incentives encoded into the software and network itself to
induce criminal activity. Many cryptocurrency companies, especially during the
ICO bubble, offered thinly-veiled securities that are marketed and sold to the
public with a wink and a nod. Insiders to these companies can manipulate these
assets and trade on non-public information because the company has no internal
policy on these activities, and the products themselves are unregulated by
external parties. Software-based distributed control fraud is a vastly
under-reported area of fraud associated with crypto assets and may be very
difficult for law enforcement to detect or prosecute.

In many jurisdictions, directors of the company are explicitly banned from
touting the expected returns of the investment. However, if one constructs an
anonymous community in which others (outside the company) market the token's
investment opportunity, this can be sufficient to drum up market interest in the
security. A digital pyramid scheme structure can be encoded indirectly into the
computer program that dictates the network's payouts, and this can create
indirect kickbacks (airdrops) and incentives for early promoters. This
decentralized and self-organizing fraud leaves the directors' hands completely
clean as low-level employees and outside actors purely perform the actions.

These new forms of software-based distributed control frauds are a brave new
world for regulation and the courts. If left to grow unchecked, these types of
fraud will claim millions of victims while the courts are still left scratching
their heads to understand the technology and mechanisms of the fraud.
\cite{saengchote_defi_2021, zetzsche_ico_2017}
