\chapter{Smart Contracts}

\index{smart contracts}

Smart contracts are a curiously named term that has sparked a great deal of
interest off the confusion of its namesake. Like many blockchain terms, a smart
contract is a semantically meaningless term in the larger corpus of discussion,
and its usage has been defined to mean everything from the universal set of all
computer programs to a particular set of computer programs running on particular
software platforms.

A contract in Common Law jurisdictions is an interaction between two parties
recognized by the law where an offer and an acceptance are established around a
shared set of terms. Both the offer and acceptance can take many forms,
including actions such as signatures, the production of shared documents, or
digital interaction. The contract terms are interpreted in a given governing
law, which defines the environment in which the contract's terms are valid and
gives a mechanism to settle disputes. The basis of this system is the
interpretation of legal terms in the minds of judges and magistrates, which
interpret the law in a historical and situational context. Smart contracts have
nothing to do with this concept.

If there is one takeaway before we delve into the technical details, it is a
simple observation that the term is deeply flawed. However, the technology has
benefited from convenience in confusion around the name; indeed, smart contracts
are a contradiction in terms as they are neither smart nor contracts.

\begin{infobox}
  \textbf{
    Smart contracts have absolutely nothing to do with legal contracts.
  }
\end{infobox}

\section{Not Contracts}

To unpack this term, we will refer to two different types of programs
colloquially called smart contracts: bitcoin scripts and solidity scripts.
Bitcoin scripts are a very simple mechanism for including small units of logic
alongside bitcoin transactions. These can be used to orchestrate the flow of
transactions according to a minimal set of logical operations. The script itself
is interpreted in a simple "computer" called a stack evaluator, a set of simple
instructions evaluated left-to-right that can view and manipulate a mutable list
of values. Many programming languages such as Python and Ruby are based on stack
virtual machines; however, bitcoin scripts are intentionally minimal for
performance reasons and to not overcomplicate the protocol. For instance, one
can express a simple program that locks a set of bitcoin values until a given
time in the future, but one cannot write a calculator or chess engine inside of
a bitcoin script. In computer science terms, the bitcoin script is not
Turing-complete since it is a language that is not capable of expressing general
computation and has no looping operation or means to evaluate terms recursively.
Not being Turing-complete drastically restricts the set of possible programs
that can be expressed in the language. In practice, bitcoin scripts are not
widely used, and there is a class of about five commonly used tasks for which
the logic is simple enough. An example bitcoin script looks like the following:

\begin{quote}
\begin{verbatim}
OP_DUP
OP_HASH160
62e907b15cbf27d5425399ebf6f0fb50ebb88f18
OP_EQUALVERIFY
OP_CHECKSIG
\end{verbatim}
\end{quote}

\index{bitcoin!script}

The Ethereum network has taken a different approach to its protocol and embedded
an entire virtual machine (called the EVM) inside the evaluation of Ether
transactions. This method for achieving this executable blockchain logic was
outlined in the 2014 ethereum whitepaper. The EVM is a virtual machine that is
also stack-based but provides more primitives and operations to manipulate stack
data. The ethereum virtual machine serves as a target for higher-level languages
(Solidity, Viper, etc.), which take human-readable source code and translate it
into a set of machine-readable EVM instructions. These instructions are encoded
as numerical codes inside ethereum transactions sent to the public network. The
primary language that ethereum contracts are written in is Solidity, a
blockchain-specific scripting language most comparable to the standard web
scripting language Javascript.

\index{Javascript}
\index{ethereum}

Solidity logic can mediate the exchange of ethereum tokens according to
arbitrarily complex logic, including arithmetic, mathematical operations,
logical control flow, and manipulation of simple data structures such as
associative maps and lists. In a suppositional sense, Solidity, the pure
language, is Turing-complete; however, in practice, it has a restriction on its
runtime imposed by its runtime. This restriction is known as bounded evaluation,
which assigns a specific cost to each operation which incurs an economic cost to
the issuer of the transaction known as gas. This gas is capped per transaction
and effectively prevents infinite-looping events while evaluating a block of
transactions. This restriction prevents the network from being globally stalled
while evaluating a transaction's potentially arbitrarily complex program logic.

\begin{quote}
\begin{verbatim}
pragma solidity >=0.4.22 <0.8.0;

contract OwnedToken {
    TokenCreator creator;
    address owner;
    bytes32 name;

    constructor(bytes32 _name) {
        owner = msg.sender;
        creator = TokenCreator(msg.sender);
        name = _name;
    }

    function changeName(bytes32 newName) public {
        if (msg.sender == address(creator))
            name = newName;
    }

    function transfer(address newOwner) public {
        if (creator.isTokenTransferOK(owner, newOwner))
            owner = newOwner;
    }
}
\end{verbatim}
\end{quote}

\index{Solidity}

Solidity falls out of a previous tradition of open-source programming tools that
have taken a "move fast and break things" approach to development philosophy,
prioritizing time to develop over theoretical concerns of semantics,
correctness, or best practices of language design. Solidity was meant to appeal
to the entry-level Javascript developer base, which uses coding practices such
as copying and pasting from code aggregator sites like Stack Overflow. As a
result, Solidity code generally has a very high defect count and has resulted in
a constant stream of high-profile security incidents directly related to coding
errors. Some studies have put the defect count at 100 per 1000 lines.

A cottage industry has arisen around trying to prevent security flaws related to
the coding errors in smart contracts. These companies have taken two approaches:
a technical one in the form of a process known as formal verification and a
heuristic one in which third-party independent contractors audit code to attest
to the integrity of the contract's logic before it goes live. The formal
verification approach builds on a branch of computer science known as
programming language theory which attempts to model the behavior of programs by
reducing them down to a mathematical description of the values computed by
individual components.

\index{formal verification}

The degree of engineering required to prove the correctness properties of these
contracts' logic is expensive, and the return on investment for that level of
engineering would do little to mitigate the risks when the underlying platform
itself is unable to be changed. However, it is much easier in practice to
deceptively market a project as "having formal verification" or "having a code
audit," as there is very little due diligence done on these projects in the
first place. In practice, both audits and formal verification approach has thus
mainly been performative and yielded little success in preventing catastrophes
in live contracts.

The most catastrophic smart contract was undoubtedly the DAO hack. The DAO was
an experimental, decentralized autonomous organization that resembled a venture
fund loosely. A program that would allow users to invest and vote on proposals
for projects to which the autonomous logic of the contract would issue funds as
a hypothetical "investment." It was a loose attempt at building what would
amount to an investment fund on the blockchain. The underlying contract itself
was deployed and went live, consuming around \$50 million at the then exchange
rate with Ether cryptocurrency. The contract contained a fundamental software
bug that allowed an individual hacker to drain DAO accounts into their accounts
and acquire the entirety of the community's marked investment. This hack
represented a non-trivial amount of the total Ether in circulation across the
network and was a major public relations disaster for the network. The community
controversially decided to roll back the blockchain to a previous state to
revert the hacker's withdrawal of funds and restore the contract to regular
operation. \cite{securities_sec_2017}

\index{DAO}

Among the programming language ecosystem, there is a niche subset of languages
known as functional languages, which are notable for their applications in
high-assurance software applications, ones where the correctness of code can be
proven in more cases. A sizable number of practitioners in the functional
programming ecosystem have spent efforts looking at the problem of smart
contract languages. Many proposals and prototypes have been made that would make
theoretical guarantees of correctness beyond the current state of affairs.
However, none of these projects have gained any traction or born fruit, as the
fundamental problem of retrofitting these tools onto existing runtimes and
training existing practitioners in unfamiliar tools within their profession.

The grandiose promise of smart contracts was for applications that build
decentralized internet applications called dApps. These dApps would behave like
existing web and mobile applications but counter interface with the blockchain
for persistence and consume or transmit cryptocurrency as part of their
operations. There are endless proposals for ideas like decentralized clones of
existing services such as Airbnb, Uber, LinkedIn, and nearly every commercially
successful web application. The implicit assumption of these proposals is that a
decentralized application would allow the applications to be superior due to
being run "by the people," independent of corporate governance, and free of
privacy and data collection problems. Much of the smart contract narrative is
built around phony populism and the ill-defined idea that there is an upcoming
third iteration of the internet (a Web 3.0) that will interact with smart
contracts to provide a new generation of applications. In practice, none of that
has manifested in any usable form, and the fundamental data throughput
limitations of blockchain data read and write actions make that vision
impossible.

In practice, most live smart contracts fall into a limited set of categories:
gambling, tumblers, NFTs, decentralized exchanges, and crowd sales. The vast
majority of code running on the public ethereum network falls into one of these
categories, with a standard set of open-source scripts driving the bulk of the
contract logic that is evaluated on the network. However, there is a wide
variety of bespoke scripts associated with different ICO companies and high-risk
gambling products that are bespoke logic and act independently of existing
community standards and practices. Looking at ethereum chain explorers and dApp
listing sites approximates the number of live applications and their taxonomy.
There are currently 2,917 listed dApps, of which in the top four categories, 17%
fall into games, 13\% fall into gambling, 10\% fall into finance, and 10\% fall
into high-risk.

\index{tumbling}
\index{gambling}

The most common script is an *ERC20* token, a contract that allows users to
issue custom token crowd sales on top of the ethereum blockchain. These tokens
were a custom registry that mapped holdings of addresses to their balances of
the custom token. By transacting with the contract (i.e., offering party), users
could exchange their ethereum tokens to "buy into" the custom tokens offered at
a specific value. The total supply of these tokens in any one of these contracts
was a custom fixed amount, and by interacting with the ERC20 contract, the
buyers' tokens were instantly liquid and could be exchanged with other users
according to the rules of the contract. This is the standard mechanism that
drove the ICO bubble and related speculation, and this token sale contract is
overwhelmingly the most common use case for smart contracts. Quite commonly,
these ERC20 tokens were tied to a *SAFT* (Simple Agreement for Tokens), a legal
contract in US law that mapped owners in the token registry to the purchasers of
those securities inside the US legal system. This SAFT agreement was
structurally similar to a SAFE (Simple Agreement for Equity), which companies in
the US typically use for early startup fundraising to offer equity to venture
capitalists and early investors.

\index{ERC20}
\index{SAFT}

Tumblers (such as tornado.cash) are another application of smart contracts which
allow users who have engaged in darknet transactions to "clean" their money by
exchanging their dirty tokens with clean tokens through a set of random swaps
which preserve the value but obscure the provenance. These tumblers use several
evasion techniques such as random transactions, time-delayed transactions, and
swaps with other chains with private transactions to return the clean money back
to the client. In traditional financial services, this would be called a
criminal activity is known as money laundering. Traditional money custodians and
money transmitters are typically required to maintain audit logs and prevent
this form of activity as part of their licenses and regulatory oversight.

\index{money laundering}
\index{darknet}

Another class of projects is the digital collectibles and digital pets genre.
One of the most popular is *CryptoKitties*: a game in which users can buy, sell,
and breed cartoon kittens. The sale of digital cats was responsible for 6.2\% of
all ethereum traffic in 2018. The number of transactions on this one application
represented a significant fraction of all contract logic and drastically
congested the global network when many individuals began trading digital cats.

\index{CryptoKitties}

Gambling products overwhelmingly dominate the remaining set of contracts. Aside
from illegal securities offerings, Gambling is the primary domain for blockchain
apps. This script allows participants to gamble their tokens in games of chance
against other blockchain users. There are a variety of so-called dApps
(distributed applications) which offer lotteries and games in which users stake
ethereum tokens, and the game makers offer opportunities to play in exchange for
transaction fees. Some of these games are classical Ponzi schemes in which
players pay into a pool of tokens, and early investors are paid out from the
increasing pool of subsequent investors. \cite{pinna_massive_2019, hartel_empirical_2019}

The ICO bubble marked a significant increase in the interest in smart contracts
arising from outlandish claims of how cryptocurrency ventures would
disintermediate and decentralize everything from the legal profession and
electricity grid to food supply chains. In reality, we have seen none of these
visions manifest, and the technology is primitive, architecturally dubious, and
lacking in any clear applications of benefit to the economy at large. The
ecosystem of dApps is a veritable wasteland of dead projects, with none having
more than a few hundred active users at best.

\section{Not Smart}

A smart contract has none of these properties, and its very design to run on an
unregulated network prevents it from interfacing with external systems in any
meaningful fashion. This confusion around the namesake of smart contracts has
been exploited by many parties to sell products and services.

The fundamental limitation of smart contracts is best exemplified by a technical
limitation referred to as the \textit{oracle problem}. In blockchain systems, there is
a data boundary between data stored on-chain, referring to data stored natively
on the blockchain database such as transactions, wallet address, and smart
contract state. Data that is off-chain colloquially means anything not within
the blockchain database, such as the public internet, data stored in traditional
databases, and data feeds for news events or financial markets. The efficacy of
many applications of smart contracts was predicated upon the ability to query
events that would occur outside the blockchain network, which would be either
self-reported or attested to by an authoritative source. This problem poses a
fundamental contradiction in the theoretical value proposition of decentralized
dApps; if their operation was run on a decentralized platform, but the core
mechanism for operation depended on a central authority, this resulted in a
system that is functionally equivalent to an existing centrally operated
application. Stated another way, if a blockchain financial product depends on
the weather or value of the S\&P 500, the smart contract would have to extend its
trust boundary to the included National Weather Service or a Bloomberg for its
operation to function. At which point then, this entire setup provides a service
that is no better or different than running the same program on a centralized
server. \cite{doctorow_inevitability_nodate}

\index{oracle problem}

Within the domain of permissioned blockchains, the terminology has been co-opted
to refer to an existing set of tools that would traditionally be called
\textit{process automation}. In 2018 so-called enterprise "smart contracts" were
the buzzword du jour for consultants to sell enterprise projects. These
so-called enterprise smart contracts had very little to do with their
counterparts in public blockchains and were existing programming tools such as
Javascript, Java, and Python rebranded or packaged in a way that would
supposedly impart the "value of the blockchain" through undefined and
indeterminate means. Indeed one of the popular enterprise blockchain platforms,
Hyperledger, provides a rather expansive definition of the term:
\cite{noauthor_hyperledger_2017}

\begin{quote}
A smart contract can describe an almost infinite array of business use
cases relating to immutability of data in multi-organizational decision
making. The job of a smart contract developer is to take an existing
business process that might govern financial prices or delivery
conditions, and express it as a smart contract in a programming language
such as JavaScript, Go, or Java. The legal and technical skills required
to convert centuries of legal language into programming language is
increasingly practiced by smart contract auditors.
\end{quote}

This paragraph is very embodiment of the blockchain meme. It is an extension of
the terminology to include an "infinite array of business cases" and encompasses
any computer program that writes to any database is a semantic extension of a
term to be meaningless. Smart contracts simply are not useful for any real-world
applications. To the extent that they are used on blockchain networks, they
provide strictly inferior services or are part of gambling or money laundering
operations that are forced to use this a flawed system because it is the only
platform that allows for illicit financing, arbitrage securities regulation, or
avoids law enforcement.
\index{blockchain meme}
